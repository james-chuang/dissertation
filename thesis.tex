\documentclass[12pt,letterpaper,oneside]{book}
\pagestyle{plain}

\usepackage{sectsty}

\chapternumberfont{\normalsize}
\chaptertitlefont{\normalsize}
\sectionfont{\normalsize}
\subsectionfont{\normalsize}

\usepackage{bu_ece_thesis_minimal}

\usepackage{setspace}
\doublespacing

\usepackage{amsmath}
\usepackage{amssymb}
\usepackage{mathspec}
\setmainfont{FreeSans}[
    Path=fonts/,
    BoldFont=FreeSansBold,
    ItalicFont=FreeSansOblique,
    BoldItalicFont=FreeSansBoldOblique
]
\setmathsfont(Digits)[Path=fonts/]{FreeSans}
\setmathrm[Path=fonts/]{FreeSans}

\usepackage[paper=letterpaper,
            layout=letterpaper,
            lmargin=1.5in,
            rmargin=1.0in,
            tmargin=1.5in,
            bmargin=1.25in,
            footskip=0.50in]{geometry}

\usepackage{enumitem}

\usepackage{graphicx}
\usepackage{float}
\usepackage{wrapfig}
\usepackage{caption}
\captionsetup{font={normalsize, stretch=1.3}, singlelinecheck=off}
\usepackage{sidecap}
\usepackage{rotating}

\usepackage{xcolor}
\definecolor{blue}{HTML}{114477}
\definecolor{grey}{HTML}{aaaaaa}
\definecolor{lightpurple}{HTML}{beaed4}
\usepackage[colorlinks=true,
            linkcolor=black,
            urlcolor=blue,
            citecolor=grey,
            backref,
            linktoc=all,
            breaklinks=false,
            pdftitle=Title\ of\ thesis,
            pdfcreator=James\ Chuang]{hyperref}

\usepackage[round, sort]{natbib}
\setcitestyle{aysep={,}, citesep={;}}
\usepackage[]{chapterbib}
\sectionbib{\section}{section}

\usepackage{tikz}
\usetikzlibrary{shapes}

\usepackage{lipsum}

\setcounter{secnumdepth}{3}
\setcounter{tocdepth}{3}

\begin{document}
\bibliographystyle{apalike}

\frontmatter


\title{The title is wasda}

\author{James Chuang}

% Type of document prepared for this degree:
%   1 = Master of Science thesis,
%   2 = Doctor of Philosophy dissertation.
%   3 = Master of Science thesis and Doctor of Philisophy dissertation.
\degree=2

\prevdegrees{B.S., Johns Hopkins University, 2013\\
	M.S., Boston University, 2018}

\department{Department of Biomedical Engineering}

% Degree year is the year the diploma is expected, and defense year is
% the year the dissertation is written up and defended. Often, these
% will be the same, except for January graduation, when your defense
% will be in the fall of year X, and your graduation will be in
% January of year X+1
\defenseyear{2019}
\degreeyear{2019}

% For each reader, specify appropriate label {First, Second, Third},
% then name, and title. IMPORTANT: The title should be:
%   "Professor of Electrical and Computer Engineering",
% or similar, but it MUST NOT be:
%   Professor, Department of Electrical and Computer Engineering"
% or you will be asked to reprint and get new signatures.
% Warning: If you have more than five readers you are out of luck,
% because it will overflow to a new page. You may try to put part of
% the title in with the name.
\reader{First}{Fred Winston, PhD}{Professor of Genetics\\Harvard Medical School}
\reader{Second}{Ahmad Khalil, PhD}{Assistant Professor of Biomedical Engineering}
\reader{Third}{L. Stirling Churchman, PhD}{Assistant Professor of Genetics\\Harvard Medical School}
\reader{Fourth}{John T. Ngo, PhD}{Assistant Professor of Biomedical Engineering}
\reader{Fifth}{Wilson Wong, PhD}{Assistant Professor of Biomedical Engineering}

% The Major Professor is the same as the first reader, but must be
% specified again for the abstract page. Up to 4 Major Professors
% (advisors) can be defined.
\numadvisors=2
\majorprof{Fred Winston, PhD}{{Professor of Genetics\\Harvard Medical School}}
\majorprofb{Ahmad Khalil, PhD}{{Professor of Biomedical Engineering}}
%\majorprofc{First M. Last, PhD}{{Professor of Astronomy}}
%\majorprofd{First M. Last, PhD}{{Professor of Biomedical Engineering}}

\maketitle
\cleardoublepage

% The copyright page is blank except for the notice at the bottom. You
% must provide your name in capitals.
\copyrightpage
\cleardoublepage

% Now include the approval page based on the readers information
\approvalpage
\cleardoublepage

% The acknowledgment page should go here. Use something like
% \newpage\section*{Acknowledgments} followed by your text.
\newpage
\section*{\centerline{Acknowledgments}}
\lipsum[1]
\vskip 1in
\noindent
James Chuang
\cleardoublepage

% The abstractpage environment sets up everything on the page except
% the text itself.  The title and other header material are put at the
% top of the page, and the supervisors are listed at the bottom.  A
% new page is begun both before and after.  Of course, an abstract may
% be more than one page itself.  If you need more control over the
% format of the page, you can use the abstract environment, which puts
% the word "Abstract" at the beginning and single spaces its text.

\begin{abstractpage}
\lipsum[1]
\end{abstractpage}
\cleardoublepage

\tableofcontents
\cleardoublepage


\cleardoublepage

\mainmatter
\chapter{Introduction}

\section{A brief introduction to transcription}

In eukaryotic cells, transcription of protein-coding genes is carried out by the protein complex RNA polymerase II (Pol II), and broadly occurs in three sequential stages of transcription initiation, elongation, and termination \citep{shandilya2012}.
During each of these stages, the Pol II complex is associated with distinct sets of factors which modulate the activity of Pol II and carry out co-transcriptional processes such as RNA capping, RNA splicing, histone modification, RNA cleavage, and RNA polyadenylation.
Given how fundamental transcription is to gene expression, it is unsurprising that every stage of transcription is highly regulated.

To get a rough idea of just how tightly transcription is regulated, it is useful to consider a back-of-the-envelope calculation of the specificity of transcription initiation in the human genome.
That is, what proportion of the human genome at which transcription could initiate does transcription initiation actually occur?

The number of positions at which transcription could theoretically initiate is simply the size of the genome: The human genome is approximately three billion base pairs in length (BNID 111378, \citet{griffin2009}), and since each base pair can be transcribed from each of its two strands, there are $6 \times 10^9$ available positions.

The number of positions at which transcription \textit{does} initiate can be estimated from the number of genes transcribed by Pol II and the number of positions that Pol II initiates from for each gene.
At last count, the human genome contains about twenty thousand protein-coding genes \citep{encode2012}.
To be conservative in our estimate with regards to specificity, we will assume that all twenty thousand genes are expressed.
We also know that protein-coding genes are only a subset of the genes transcribed by Pol II: Pol II also transcribes multiple classes of non-coding genes, including enhancers and long non-coding RNAs \citep{kaikkonen2018}.
Compared to protein-coding genes, the number of non-coding genes is less certain.
If we assume that there are five non-coding genes for each coding gene, this brings our estimate of the number of genes transcribed by Pol II to $1.2 \times 10^5$ genes.

As you will see from yeast transcription start site data in later chapters, transcription initiation for a single gene generally occurs at multiple nucleotides, generating multiple major transcript isoforms per gene.
Assuming that there are five major transcription start sites (TSSs) per gene, the proportion of the human genome at which transcription initiation occurs is
\begin{align*}
    \frac{\left(1.2 \times 10^5 \; \text{genes}\right) \left(5 \; \frac{\text{TSSs}}{\text{gene}} \right)}
         {\left(6 \times 10^9 \; \text{possible TSSs} \right)}
    &= 1 \times 10^{-4}.
\end{align*}
Our rough estimate says that, when presented with ten thousand positions to choose from, RNA polymerase starts transcription from only one!\footnote{A similar conclusion is reached by examining ENCODE CAGE-seq data: At the time of writing, ENCODE reports roughly 150,000 TSS peaks across 30 cell types/cell lines. Assuming the signal is concentrated at 5 nucleotides per peak, then $\frac{\left(1.5 \times 10^5 \text{peaks} \right) \left(5 \frac{\text{nt}}{\text{peaks}} \right)}{6 \times 10^9 \text{nt}} = \frac{1}{8000}$.}

Many factors are known to contribute to this remarkable specificity.
Most notably, transcription initiation requires the presence of specific DNA sequence motifs, which increase the probability of Pol II binding to DNA together with necessary initiation factors \citep{haberle2018}.
That factors known to associate with Pol II during transcription initiation control transcription initiation is unsurprising.
A less obvious fact is that some transcription \textit{elongation} factors, including histone chaperones and histone modification enzymes, also play a role in restricting where transcription initiation is allowed to occur \citep{kaplan2003, cheung2008, hennig2013}.
Evidence suggests that these elongation factors are likely required to maintain normal chromatin structure over transcribed regions, and that the disruption of normal chromatin structure allows Pol II to initiate transcription in regions which are normally inacessible \citep{}.
Chapters \ref{chapter:six} and \ref{chapter:five} of this dissertation describe our studies of \textbf{Spt6} and \textbf{Spt5}, two of the transcription elongation factors involved in this process.
One phenotype observed when these factors are disrupted is \textbf{intragenic transcription}, transcription appearing to arise from within protein-coding sequences.
In chapter \ref{chapter:stress}, I describe our efforts to understand how intragenic transcription might play a role in the cellular response to various stress conditions.
The remainder of this introduction provides a brief overview of the considerations taken into account in order to make the data analyses behind this dissertation as transparent and reproducible as possible.

\section{Reproducible data analysis for genomics}

My role in the projects in this dissertation is a mix of \href{https://blog.insightdatascience.com/data-science-vs-data-engineering-62da7678adaa}{\textbf{data scientist}} and \href{https://blog.insightdatascience.com/data-science-vs-data-engineering-62da7678adaa}{\textbf{data engineer}}: I build pipelines for processing (usually genomic) datasets, taking raw data through processing, statistical analysis, and data visualization.
This mostly entails surveying available tools, selecting the tools most suitable for the task, and coding solutions to problems when existing tools are insufficient.

The analysis of complex datasets like those in genomics presents challenges to achieving transparency and reproducibility when reporting methods and results.
In building the data analysis pipelines behind the results of this dissertation, I have tried to meet these challenges by following best practices that would be standards for publication in an ideal world.
All of my data analyses are open source (\href{https://github.com/winston-lab}{github.com/winston-lab}), and are designed to be reproducible by others: For all publications, an self-contained archive is uploaded which includes everything needed to go from raw data to the figures and results of the publication (e.g. \url{https://doi.org/10.5281/zenodo.1409826}).
This level of accessibility is greatly facilitated by building data analyses using Snakemake \citep{koster2012}, one of several available frameworks for workflow management \citep{voss2017, ditommaso2017}.
Snakemake's scalable execution and its ability to specify dependencies in virtual environments allow workflows to truly be reproducible: data analyses can be re-run on personal computers, computing clusters, or cloud environments, and the exact versions of the software used when initially running the data analysis will automagically be deployed.

Open sharing of data and code like this is essential to the scientific process.
When analysis pipelines routinely consist of tens of steps with tens of parameters each, seeing the data and code is the only way for those interested to know exactly how the data were handled.
Altogether, this allows for more informed evaluation of results from the literature, as well as the possibility of finding and correcting errors in analysis.

\clearpage
\bibliographystyle{apalike}
\begingroup
    \singlespacing
    \bibliography{references/introduction,references/spt6,references/spt5}
\endgroup


\cleardoublepage

\chapter{Genomics of transcription elongation factor Spt6}
\label{chapter:six}

\section{Abstract}

Spt6 is a conserved transcription elongation factor thought to replace nucleosomes in the wake of transcription.
\textit{Saccharomyces cerevisiae} \textit{spt6} mutants have elevated levels of intragenic transcripts, transcripts appearing to initiate from within gene bodies.
In this work, we apply two high resolution genomic assays of transcription initiation to catalog the full extent of intragenic transcription in the \textit{spt6-1004} mutant, and show for the first time on a genome-wide scale that the intragenic transcripts observed in \textit{spt6-1004} are largely explained by new transcription initiation.
We also assay chromatin structure genome-wide in \textit{spt6-1004}, finding a global depletion and disordering of nucleosomes.
Our results also show an unexpected decrease in expression from most canonical genic promoters.
By comparing features of intragenic and genic promoters, we reveal several similarities between the two.
Altogether, we propose that the transcription changes in \textit{spt6-1004} are explained by a competition for transcription initiation factors between genic and intragenic promoters, which is made possible by a global decrease in nucleosome protection of the genome.

\section{Collaborators}

\begin{description}[align=right, labelwidth=5cm, noitemsep]
    \item [Steve Doris] optimized TSS-seq and ChIP-nexus protocols
    \item [] generated TSS-seq and ChIP-nexus libraries
    \item [Olga Viktorovskaya] generated MNase-seq libraries
    \item [Magdalena Murawska] generated NET-seq libraries
    \item [Dan Spatt] Northern, Western, and ChIP experiments
\end{description}

\section{Introduction to Spt6 and intragenic transcription}

The conserved transcription elongation factor Spt6 interacts directly with RNA polymerase II \citep{close2011, diebold2011, liu2011, sdano2017, sun2010, yoh2007}, histones \citep{bortvin1996, mccullough2015}, and another elongation factor called Spn1/Iws1 \citep{diebold2010b, li2018, mcdonald2010}.
The classification of Spt6 as a transcription elongation factor is based on its association with elongating Pol II \citep{andrulis2000, ivanovska2011, kaplan2000, mayer2010, krogan2002}, and its ability to enhance elongation both \textit{in vitro} \citep{endoh2004} and \textit{in vivo} \citep{ardehali2009}, though Spt6 has also been shown to regulate initiation in a small number of cases \citep{adkins2006, ivanovska2011}.
Evidence suggests that as Spt6 travels with elongating Pol II, it acts as a histone chaperone, reassembling nucleosomes after their displacement from DNA due to transcription \citep{duina2011, ivanovska2011}.
Consistent with its histone chaperone function, Spt6 influences chromatin structure \citep{bortvin1996, degennaro2013, ivanovska2011, jeronimo2015, kaplan2003, perales2013, vanbakel2013}; Spt6 is also required for some histone modifications, including H3K36 methylation \citep{carrozza2005, chu2006, yoh2008, youdell2008}, and, in some organisms, H3K4 and H3K27 methylation \citep{begum2012, chen2012, degennaro2013, wang2017, wang2013}.

\begin{wrapfigure}[10]{r}{3in}
    \centering
    \includegraphics[width=3in]{figures/six/six_spt6_western.pdf}
    \caption[Western blot for Spt6 in wild-type and \textit{spt6-1004} cells, at 30\textdegree C and after 80 minutes at 37\textdegree C.]{Western blot for Spt6 in wild-type and \textit{spt6-1004} cells, at 30\textdegree C and after 80 minutes at 37\textdegree C. Spt6 and Dst1 from a spike-in were detected using $\alpha$-FLAG and $\alpha$-Myc antibodies, respectively. The mean $\pm$ standard deviation of three blots are shown below each lane.}
    \label{fig:six_spt6_western}
\end{wrapfigure}

Studies in the yeasts \textit{Saccharomyces cerevisiae} and \textit{Schizosaccharomyces pombe} have previously examined the requirement for Spt6 in normal transcription \citep{cheung2008, degennaro2013, kaplan2003, pathak2018, uwimana2017, vanbakel2013}.
As Spt6 is essential for viability in \textit{S. cerevisiae}, many of these studies use the same temperature-sensitive \textit{spt6} mutant used in this project, \textbf{\textit{spt6-1004}}, which encodes an in-frame deletion of a helix-hairpin-helix domain within Spt6 \citep{kaplan2003}.
When \textit{spt6-1004} cells are shifted from 30\textdegree C to 37\textdegree C for 80 minutes, bulk Spt6 protein levels are depleted to about 20\% of wild-type levels, though cells are still viable  (Figure \ref{fig:six_spt6_western}, \citep{kaplan2003}).
A notable phenotype of the \textit{spt6-1004} mutant is the appearance of \textbf{intragenic transcripts}, transcripts which appear to arise from within protein-coding sequences in both sense and antisense orientations relative to the coding gene (Figure \ref{fig:six_gene_diagram}) \citep{cheung2008, degennaro2013, kaplan2003, uwimana2017}.

\begin{wrapfigure}[7]{r}{3in}
    \centering
    \includegraphics[width=3in]{figures/six/six_gene_diagram.pdf}
    \caption[Diagram of transcript classes.]{Diagram of transcript orientation with respect to coding DNA sequences, for the categories of transcripts referred to in this document.}
    \label{fig:six_gene_diagram}
\end{wrapfigure}
Previous genome-wide measurements of transcript levels in \textit{spt6-1004} relied on tiled microarrays \citep{cheung2008} and RNA sequencing \citep{degennaro2013, uwimana2017}.
Studying intragenic transcription is difficult with these methods, since the signal for an intragenic transcript in the same orientation as the gene it overlaps is convoluted with the signal from the full-length `genic' transcript (Figures \ref{fig:six_gene_diagram}, \ref{fig:six_aat_assay_comparison}) \citep{cheung2008, lickwar2009}.
Identification of intragenic transcription has thus relied on identifying cases where the signal towards the 3$^\prime$ end of a transcript is greater than the signal towards the 5$^\prime$ end.
However, this leads to both false positives, due to the inherent variability of the signal over a transcript, as well as false negatives, due to the requirement of the intragenic transcript to be well-expressed relative to its corresponding genic transcript in order to be identified.
Additionally, these methods are assays of steady-state RNA levels, which makes them unable to distinguish whether the intragenic transcripts observed in \textit{spt6-1004} result from: A) new intragenic transcription initiation in the mutant, B) reduced decay of intragenic transcripts which are rapidly degraded in wild-type, or C) processing of full-length protein-coding RNAs.
New transcription initiation has been shown to be responsible for individual cases of intragenic initiation \citep{kaplan2003}, but this has not previously been studied on a genome-wide scale.
% \begin{SCfigure}[40][h]
\begin{figure}[h]
    \centering
    \includegraphics[width=6in]{figures/six/six_aat_assay_comparison.pdf}
    \caption[RNA-seq, TSS-seq, and TFIIB ChIP-nexus signal at the \textit{AAT2} gene, in \textit{spt6-1004} after 80 minutes at 37\textdegree C.]{Sense strand RNA-seq signal, sense strand TSS-seq signal, and TFIIB ChIP-nexus protection at the \textit{AAT2} gene, in \textit{spt6-1004} after 80 minutes at 37\textdegree C.}
    \label{fig:six_aat_assay_comparison}
% \end{SCfigure}
\end{figure}

To address these challenges to studying intragenic transcription, we applied two genomic assays to \textit{spt6-1004}: transcription start-site sequencing (\textbf{TSS-seq}), and \textbf{ChIP-nexus of TFIIB}, a component of the RNA polymerase II pre-initiation complex (PIC).
TSS-seq sequences the 5$^\prime$ end of capped and polyadenylated RNAs \citep{arribere2013, malabat2015}, allowing separation of intragenic from genic RNA signals and identification of intragenic transcript starts with single-nucleotide resolution (Figure \ref{fig:six_aat_assay_comparison}).
ChIP-nexus is a high-resolution chromatin immunoprecipitation technique, in which the immunoprecipitated DNA is exonuclease digested up to the bases crosslinked with the protein of interest before sequencing \citep{he2015}.
When applied to the PIC component TFIIB, ChIP-nexus reports where transcription initiation is occurring, thus allowing us to determine if intragenic transcripts in \textit{spt6-1004} result from new transcription initiation.

\section{TSS-seq and TFIIB ChIP-nexus results for \textit{spt6-1004}}
\label{sec:six_tss_tfiib}

To study the relationship between Spt6 and transcription, TSS-seq and TFIIB ChIP-nexus libraries were prepared from wild-type and \textit{spt6-1004} cells, after cultures were shifted from 30\textdegree C to 37\textdegree C for 80 minutes.
In wild-type cells, TSS-seq and TFIIB ChIP-nexus recapitulate their expected distributions over the genome: Most TSS signal is restricted to annotated genic TSSs, while most TFIIB signal is localized just upstream of the TSS (Figures \ref{fig:six_tss_seq_heatmaps}, \ref{fig:six_tfiib_heatmap}).
In \textit{spt6-1004}, the signal for both assays infiltrates gene bodies, reflecting widespread intragenic expression of capped and polyadenylated transcripts, and suggesting that new transcription initiation contributes to the intragenic transcription phenotype.
Notably, sense strand TSS-seq signal in \textit{spt6-1004} tends to occur towards the 3$^\prime$ end of genes, while antisense strand TSS-seq signal tends to occur towards the 5$^\prime$ end of genes.

\clearpage

\begin{figure}[H]
    \centering
    \includegraphics[width=6in]{figures/six/six_tss_seq_heatmaps.pdf}
    \caption[Heatmaps of sense and antisense TSS-seq signal from wild-type and \textit{spt6-1004} cells, over non-overlapping coding genes.]{Heatmaps of sense and antisense TSS-seq signal from wild-type and \textit{spt6-1004} cells, over 3522 non-overlapping coding genes aligned by wild-type genic TSS and sorted by annotated transcript length. Data are shown for each gene up to 300 nucleotides 3$^\prime$ of the cleavage and polyadenylation site (CPS), indicated by the white dotted line. Values are the mean of spike-in normalized coverage over two replicates, in non-overlapping 20 nucleotide bins. Values above the 92\textsuperscript{nd} percentile are set to the 92\textsuperscript{nd} percentile for visualization.}
    \label{fig:six_tss_seq_heatmaps}
\end{figure}
\begin{SCfigure}[50][h]
% \begin{figure}[h]
    \centering
    \includegraphics[width=4in]{figures/six/six_tfiib_heatmap.pdf}
    \caption[Heatmaps of TFIIB ChIP-nexus protection from wild-type and \textit{spt6-1004} cells, over non-overlapping coding genes]{Heatmaps of TFIIB binding measured by ChIP-nexus, over the same regions shown in Figure \ref{fig:six_tss_seq_heatmaps}. Values are the mean of library-size normalized coverage over two replicates, in non-overlapping 20 bp bins. Values above the 85\textsuperscript{th} percentile are set to the 85\textsuperscript{th} percentile for visualization.}
    \label{fig:six_tfiib_heatmap}
\end{SCfigure}
% \end{figure}

\clearpage

The TSS-seq data were quantified by peak calling and differential expression analysis, and classified into genomic categories based on their position relative to coding genes.
As suggested by the heatmap visualization (Figure \ref{fig:six_tss_seq_heatmaps}), we detect significant induction of over 4000 intragenic and antisense TSSs in \textit{spt6-1004} (Figure \ref{fig:six_tss_diffexp_summary}).
Compared to previous studies identifying \textit{spt6-1004} intragenic transcription by tiled microarray and RNA-seq \citep{cheung2008, uwimana2017}, we identify intragenic transcription at over 1000 additional genes (Figure \ref{fig:six_intragenic_genes_bvenn}), with the additional information of exact start sites for all identified TSSs.
\begin{figure}[H]
    \centering
    \begin{minipage}[t]{2.875in}
        \centering
        \includegraphics[width=2.875in]{figures/six/six_tss_diffexp_summary.pdf}
        \caption[Bar plot of the number of TSS-seq peaks in various genomic classes detected as differentially expressed in \textit{spt6-1004} versus wild-type.]{Bar plot of the number of TSS-seq peaks detected as differentially expressed in \textit{spt6-1004} versus wild-type, both after 80 minutes at 37\textdegree C. The height of each bar is proportional to the total number of peaks in the category, including those not found to be significantly differentially expressed.}
        \label{fig:six_tss_diffexp_summary}
    \end{minipage}\hfill
    \begin{minipage}[t]{2.875in}
        \centering
        \includegraphics[width=2.875in]{figures/six/six_intragenic_genes_bvenn.pdf}
        \caption[Set diagram of the number of genes with \textit{spt6-1004}-induced intragenic transcripts reported by studies using tiled microarrays, RNA-seq, and our TSS-seq data.]{Set diagram of the number of genes reported to have \textit{spt6-1004}-induced intragenic transcripts using tiled microarrays \citep{cheung2008}, RNA-seq \citep{uwimana2017}, and TSS-seq (this work).}
        \label{fig:six_intragenic_genes_bvenn}
    \end{minipage}
\end{figure}

The TSS-seq data also revealed an unexpected downregulation of most genic TSSs: In this experiment, we detected a significant downregulation to levels below 67\% of wild-type levels at 75\% (3579/4792) of genic TSSs (Figure \ref{fig:six_tss_diffexp_summary}).
As a result of intragenic/antisense induction and genic repression, expression levels in \textit{spt6-1004} of all classes of transcripts become similar to one another (Figure \ref{fig:six_tss_expression_levels}).
% \begin{wrapfigure}[12]{r}{3in}
\begin{SCfigure}[50][h]
    \centering
    \includegraphics[width=3.75in]{figures/six/six_tss_expression_levels.pdf}
    \caption[Violin plots of expression level distributions for genomic classes of TSS-seq peaks in wild-type and \textit{spt6-1004} cells.]{Violin plots of expression level distributions for genomic classes of TSS-seq peaks in wild-type and \textit{spt6-1004}, both after 80 minutes at 37\textdegree C. Normalized counts are the mean of spike-in size factor normalized counts from two replicates.}
    \label{fig:six_tss_expression_levels}
\end{SCfigure}

The changes in transcript levels in \textit{spt6-1004} observed by TSS-seq correspond with substantial differences in the pattern of TFIIB binding on the genome.
While TFIIB in wild-type binds in discrete peaks within promoter regions, TFIIB in \textit{spt6-1004} binds much more promiscuously, with many loci having TFIIB signal spread over broad regions of the genome (Figure \ref{fig:six_tfiib_spreading_ssa4}).
\begin{figure}[h]
    \centering
    \includegraphics[width=6in]{figures/six/six_tfiib_spreading_ssa4.pdf}
    \caption[TFIIB ChIP-nexus protection over the 20 kb flanking the gene \textit{SSA4}, in wild-type and \textit{spt6-1004} cells.]{
        \begin{description}[align=right, nosep, itemindent=0pt, leftmargin=4.2em, font=\normalfont]
            \item [top)] TFIIB ChIP-nexus protection in wild-type and \textit{spt6-1004}, over 20 kb of\\ chromosome II flanking the \textit{SSA4} gene.
            \item [bottom)] Expanded view of TFIIB protection over the \textit{SSA4} gene.
        \end{description}
    }
    \label{fig:six_tfiib_spreading_ssa4}
\end{figure}
This difference in binding pattern makes peak calling ineffective for quantifying TFIIB signal in this case: ChIP-seq peak callers generally use different algorithms for calling `narrow' peaks (e.g. for sequence-specific transcription factors) and `broad' peaks (e.g. for histone modifications), meaning that a single algorithm is unable to call a unified set of peaks that is meaningful for differential binding analyses between wild-type and \textit{spt6-1004}.
Therefore, to see if changes in transcript levels in \textit{spt6-1004} correspond to changes in transcription initiation, we compared the change in TSS-seq signal at TSS-seq peaks in \textit{spt6-1004} to the change in TFIIB ChIP-nexus signal in the window extending 200 bp upstream of the TSS-seq peak.
Changes in TSS-seq signal in \textit{spt6-1004} are associated with a change in TFIIB signal of the same sign at over 82\% of TSSs of any genomic class (Figure \ref{fig:six_tss_v_tfiib}), indicating that the increase in intragenic transcript levels and decrease in genic transcript levels observed in \textit{spt6-1004} are in large part explained by changes in transcription initiation.
    % \vspace{0.5em}
\begin{figure}
    \centering
    \includegraphics[width=6in]{figures/six/six_tss_v_tfiib.pdf}
    \caption[Scatterplots of fold-change in \textit{spt6-1004} over wild-type, comparing TSS-seq and TFIIB ChIP-nexus.]{Scatterplots of fold-change in \textit{spt6-1004} over wild-type, comparing TSS-seq and TFIIB ChIP-nexus. Each dot represents a TSS-seq peak paired with the window extending 200 bp upstream of the TSS-seq peak summit for quantification of TFIIB ChIP-nexus signal. Fold-changes are regularized fold-change estimates from DESeq2, with size factors determined from the \textit{S. pombe} spike-in (TSS-seq), or \textit{S. cerevisiae} counts (ChIP-nexus).}
    \label{fig:six_tss_v_tfiib}
\end{figure}

\clearpage

\section{MNase-seq results from \textit{spt6-1004}}

Because a primary function of Spt6 is to act as histone chaperone that reassembles nucleosomes in the wake of transcription \citep{duina2011}, it is reasonable to expect that the transcriptional changes seen in \textit{spt6-1004} would be associated with changes in chromatin structure.
The requirement for Spt6 in maintaining normal chromatin structure has been demonstrated in previous studies \citep{bortvin1996, ivanovska2011, jeronimo2015, kaplan2003, perales2013, vanbakel2013, degennaro2013}.
To re-examine this requirement in higher resolution, we assayed nucleosome protection genome-wide using micrococcal nuclease digestion of chromatin followed by sequencing (MNase-seq).
\begin{figure}[h]
    \centering
    \includegraphics[width=6in]{figures/six/six_mnase_metagene.pdf}
    \caption[Average MNase-seq dyad signal in wild-type and \textit{spt6-1004}, over non-overlapping genes aligned by wild-type +1 nucleosome dyad.]{Average MNase-seq dyad signal in wild-type and \textit{spt6-1004}, over 3522 non-overlapping coding genes aligned by wild-type +1 nucleosome dyad. The solid line and shading are the median and inter-quartile range of the mean spike-in normalized coverage over two replicates (\textit{spt6-1004}) or one experiment (wild-type), in non-overlapping 20 bp bins.}
    \label{fig:six_mnase_metagene}
\end{figure}

\begin{wrapfigure}[10]{r}{3in}
    \centering
    \includegraphics[width=3in]{figures/six/six_global_nuc_occ_fuzz.pdf}
    \caption[Contour plot of nucleosome occupancy and fuzziness in wild-type and \textit{spt6-1004}.]{Contour plot of the distribution of nucleosome occupancy and fuzziness in wild-type and \textit{spt6-1004}. Dashed lines indicate median values.}
    \label{fig:six_global_nuc_occ_fuzz}
\end{wrapfigure}
In wild-type, the MNase-seq data recapitulate the expected signature over genes, with a nucleosome-depleted region upstream of a strongly positioned `+1' nucleosome, and a regularly phased array of nucleosomes over the gene body (Figure \ref{fig:six_mnase_metagene}).
In \textit{spt6-1004}, nucleosome signal is severely reduced at canonical nucleosome positions and spreads into inter-nucleosome regions.
Changes in aggregate nucleosome signal such as those observed in Figure \ref{fig:six_mnase_metagene} are the combination of changes to nucleosome occupancy (the number of reads assigned to a nucleosome), fuzziness (the standard deviation of read positions for a nucleosome), and position (the coordinate with the maximum reads for a nucleosome) \citep{chen2013}.
Using DANPOS2 \citep{chen2013}, we called nucleosome positions and quantified these metrics for wild-type and \textit{spt6-1004}.
Wild-type nucleosomes span a relatively wide range of occupancy and fuzziness space, with highly occupied nucleosomes tending to be less fuzzy (i.e., more well-positioned) (Figure \ref{fig:six_global_nuc_occ_fuzz}).
In \textit{spt6-1004}, the population of nucleosomes is much more homogeneous: nucleosome occupancy is decreased globally, and nucleosome fuzziness is restricted to the high end of the wild-type distribution.

Previous studies observed two trends: 1) In wild-type cells, nucleosome positioning is weaker over highly transcribed genes than over moderately transcribed genes \citep{shivaswamy2008}, and 2) In \textit{spt6-1004} cells, the decrease in nucleosome occupancy is greater for highly transcribed genes \citep{ivanovska2011}.
To re-examine these trends, we looked at the MNase-seq data in the context of NET-seq data, which reports the position of actively transcribing RNAPII and reflects a gene's level of transcription (Figure \ref{fig:six_mnase_heatmaps}) \citep{churchman2012}.
The data support the first trend: in wild-type, genes with the strongest NET-seq signal have weak patterning of MNase-seq signal.
However, we find no obvious relationship between transcription level and the nucleosome occupancy changes observed in \textit{spt6-1004} (Figure \ref{fig:six_mnase_heatmaps}): Genes with the greatest transcription do tend to have lower MNase-seq signal in \textit{spt6-1004}, but this is expected since these genes also have lower MNase-seq signal in wild-type.
The discrepancy with prior work might be explained by the greater resolution and breadth of MNase-seq versus MNase and microarray of chromosome III \citep{ivanovska2011}.

% \begin{figure}[h]
\begin{sidewaysfigure}
    \centering
    \includegraphics[width=8.25in]{figures/six/six_mnase_heatmaps.pdf}
    \caption[Heatmaps of sense NET-seq signal, MNase-seq dyad signal, nucleosome occupancy changes, and nucleosome fuzziness changes over non-overlapping coding genes, arranged by sense NET-seq signal.]{
        \begin{description}[align=right, nosep, itemindent=0pt, leftmargin=4.2em, font=\normalfont]
            \item [left)] Heatmap of sense strand NET-seq signal for 3522 non-overlapping genes, aligned by genic TSS and sorted by total sense strand NET-seq signal in the window extending 500 nt downstream from the genic TSS. Values are the mean of library-size normalized coverage over two replicates, in non-overlapping 20 nt bins.
            \item [middle)] Heatmaps of MNase-seq dyad signal in wild-type and \textit{spt6-1004} for the same genes, aligned by wild-type +1 nucleosome dyad and arranged by sense NET-seq signal as in the leftmost panel. Values are the mean of spike-in normalized coverage over two replicates (\textit{spt6-1004}) or one experiment (wild-type), in non-overlapping 20 bp bins.
            \item [right)] Heatmaps of fold-change in nucleosome occupancy and fuzziness for the same genes, aligned by wild-type +1 nucleosome dyad and arranged by sense NET-seq signal as in the leftmost panel.
        \end{description}
    }
    \label{fig:six_mnase_heatmaps}
% \end{figure}
\end{sidewaysfigure}

\clearpage
\subsection{Clustering of MNase-seq profiles at \textit{spt6-1004}-induced intragenic TSSs}

The aggregate MNase-seq dyad signal around all \textit{spt6-1004} intragenic TSSs is aperiodic (Figure \ref{fig:six_intragenic_mnase_metagenes}, top left panel), which occurs as a result of destructive interference from offset nucleosome phasing patterns.
To discover these phasing patterns, we used the wild-type and \textit{spt6-1004} MNase-seq data flanking intragenic TSSs to train a self-organizing map to assign TSSs with similar MNase-seq patterns to nearby nodes in a rectangular grid (Figure \ref{fig:six_mnase_som}).
This allowed us to see that, although there is considerable diversity in the nucleosome pattern surrounding intragenic TSSs, most intragenic TSSs occur in areas between the positions of nucleosome dyads.
By hierarchically clustering the nodes of the self-organizing map, we further grouped intragenic TSSs into three major clusters differing primarily by the phasing of the nucleosome array relative to the TSS, as shown in Figure \ref{fig:six_intragenic_mnase_metagenes}.
In all three clusters, nucleosomes are disrupted to similar levels in \textit{spt6-1004}.

Because GC-poor DNA sequences are nucleosome disfavoring and are known to occur in promoter regions \citep{iyer1995,kaplan2008,tillo2009,zhang2009}, we also examined the GC content surrounding the three clusters of intragenic TSSs.
For all three clusters, the GC content of the DNA drops just upstream of the TSS to a slightly lesser degree than for genic TSSs (Figure \ref{fig:six_intragenic_mnase_metagenes}).

\begin{sidewaysfigure}
    \centering
    \includegraphics[width=8.25in]{figures/six/six_mnase_som.pdf}
    \caption[Average MNase-seq dyad signal around all \textit{spt6-1004}-induced intragenic TSSs, grouped by a self-organizing map of the MNase-seq signal.]{Average MNase-seq dyad signal around all \textit{spt6-1004}-induced intragenic TSSs, grouped by assignment to nodes of a 6x8 super-organizing map (SOM). The number of TSSs assigned to each node is shown in the upper right of each panel, and is shaded by the node's assignment to a cluster determined by agglomerative hierarchical clustering of the nodes. The solid line and shading are the median and inter-quartile range of the mean spike-in normalized coverage over two replicates (\textit{spt6-1004}) or one experiment (wild-type), in non-overlapping 5 bp bins.}
    \label{fig:six_mnase_som}
\end{sidewaysfigure}

\begin{figure}[H]
\centering
\includegraphics[width=6in]{figures/six/six_intragenic_mnase_metagenes.pdf}
\caption[Average wild-type and \textit{spt6-1004} MNase-seq dyad signal and GC content for three clusters of \textit{spt6-1004}-induced intragenic TSSs, as well as wild-type genic TSSs.]{
    \begin{description}[align=right, nosep, itemindent=0pt, leftmargin=6.2em, font=\normalfont]
        \item [left column)] Average MNase-seq dyad signal for \textit{spt6-1004} intragenic TSSs, both aggregated and grouped into three clusters by the wild-type and \textit{spt6-1004} MNase-seq dyad signal flanking the TSS, as well as all genic TSSs detected in wild-type. Values are the mean of spike-in normalized dyad coverage in non-overlapping 10 bp bins, averaged over two replicates (\textit{spt6-1004}) or one experiment (wild-type). The solid line and shading are the median and inter-quartile range.
        \item [right column)] Average GC content of the DNA sequence in a 21 bp window, as above.
    \end{description}
}
\label{fig:six_intragenic_mnase_metagenes}
\end{figure}

\clearpage

\section{Other features of \textit{spt6-1004} intragenic promoters}

MNase-seq indicates that nucleosomes are lost across the entire genome in \textit{spt6-1004}.
However, TSSs observed in \textit{spt6-1004} occur in specific locations, suggesting that loss of nucleosomes is necessary but not sufficient for intragenic transcription, and that additional features such as the drop in GC content at intragenic TSSs (Figure \ref{fig:six_intragenic_mnase_metagenes}) may be required.
The resolution with which we were able to identify intragenic TSSs allowed us to closely examine sequence features that might contribute to intragenic transcription.

\subsection{Information content and sequence preference}

\begin{wrapfigure}[13]{R}{3in}
\centering
\includegraphics[width=3in]{figures/six/six_tss_seqlogos.pdf}
\caption[Sequence logos of TSS-seq reads overlapping genic and intragenic TSS-seq peaks in \textit{spt6-1004}.]{Sequence logos depicting information content and sequence preference of TSS-seq reads overlapping genic and intragenic TSS-seq peaks in \textit{spt6-1004}.}
\label{fig:six_tss_seqlogos}
\end{wrapfigure}

To examine the DNA sequence preference of TSSs in \textit{spt6-1004}, we aligned the sequences of all TSS-seq reads overlapping TSS-seq peaks of each class, and calculated the information content and sequence distribution for each class.
Intragenic TSSs have a sequence preference almost identical to the previously observed sequence preference of genic TSSs (Figure \ref{fig:six_tss_seqlogos}) \citep{malabat2015}, suggesting that RNA polymerase initiates transcription similarly at genic and intragenic TSSs.

\subsection{Enrichment of the TATA box}

A characteristic feature of canonical genic promoters is the presence of a TATA box or TATA-like DNA element which allows for the recruitment of Pol II and general transcription factors via binding of the TFIID complex, which includes TATA-binding protein \citep{rhee2012}.
To examine whether the presence of TATA elements might contribute to \textit{spt6-1004} intragenic transcription, we looked for exact matches to the TATA consensus sequence TATAWAWR in the window extending 200 nucleotides upstream of \textit{spt6-1004} TSSs, finding matches at 13.7\% of regions upstream of intragenic TSSs and 24.7\% for antisense TSSs, versus 24.4\% for all genic TSSs and 8.9\% for random locations in the genome.
Moreover, the TATA elements found near intragenic and antisense TSSs are highly concentrated in the region 50 to 100 nucleotides upstream of the TSS, where TATA elements are most often found for genic TSSs (Figure \ref{fig:six_intragenic_tata}).
This further supports the model that \textit{spt6-1004} intragenic promoters are sequences similar to canonical genic promoters, which become accessible for transcription initiation when the normal chromatin state is disturbed.

\begin{SCfigure}[50][h]
    \centering
    \includegraphics[width=3.7in]{figures/six/six_intragenic_tata.pdf}
    \caption[Kernel density estimate of matches to a consensus TATA-box motif upstream of genic and \textit{spt6-1004}-induced intragenic TSSs.]{Scaled density of exact matches to the motif TATAWAWR upstream of TSSs. For each category, a Gaussian kernel density estimate of the positions of motif occurrences is scaled by the number of motif occurrences per region.}
    \label{fig:six_intragenic_tata}
\end{SCfigure}

\subsection{Sequence motifs discovered}

To discover additional sequence features of \textit{spt6-1004} intragenic promoters, we performed \textit{de novo} motif discovery using MEME-ChIP \citep{machanick2011} for the regions -100 to +30 nucleotides relative to TSS summits.
The most enriched motif found by MEME at both intragenic and antisense \textit{spt6-1004} TSSs is, with respect to sense genic transcription, a GA-rich motif with 3-nucleotide periodicity (Figure \ref{fig:six_meme_motifs}).
This motif occurs at only a small subset of intragenic TSSs, but is highly unlikely to occur by chance (compare the expected to observed number of occurrences in Figure \ref{fig:six_meme_motifs}).
The motif is not enriched at genic TSSs upregulated in \textit{spt6-1004}, and is not an obvious match to a DNA-binding factor in the databases searched \citep{deboer2011,macisaac2006,newburger2008,pachkov2013,teixeira2017,zhu1999,weirauch2014}.
If this motif is directly related to intragenic transcription, we speculate that it might create a DNA structure favorable for transcription initiation.

\begin{figure}[h]
    \centering
    \includegraphics[width=6in]{figures/six/six_meme_motifs.pdf}
    \caption[Sequence logos of motifs discovered by MEME upstream of \textit{spt6-1004}-induced intragenic and antisense TSSs.]{Sequence logos of motifs discovered by MEME \citep{bailey2015} in the window -100 to +30 bp relative to \textit{spt6-1004} intragenic and antisense TSSs. For each motif, the observed number of occurrences and the expected number of occurrences if the input sequences were scrambled are shown.}
    \label{fig:six_meme_motifs}
\end{figure}

\section{Discussion}

In this work, we integrated multiple quantitative genomic approaches to study the conserved transcription elongation factor Spt6.
Our TSS-seq and TFIIB ChIP-nexus results reveal the full extent of intragenic and antisense transcript expression in \textit{spt6-1004}, and show that these transcripts are largely explained by new RNA Pol II transcription initiation.
Our MNase-seq results show that this new transcription initiation happens in the context of a global depletion and disordering of nucleosomes from chromatin.
We speculate that this dramatic decrease in nucleosome protection of the genome leads to intragenic transcription by allowing initiation factors to access normally inaccessible promoter-like sequences within coding sequences.
This model is supported by the similarities we observe between genic and intragenic promoters in DNA GC content, initiation motif, and TATA element frequency.
This may also explain the unexpected decrease in transcription initiation we see at almost all genic promoters in \textit{spt6-1004}: Assuming that the pool of transcription initiation factors in the cell is limiting, then making thousands of additional binding sites available to the initiation machinery would decrease the frequency at which the initiation machinery finds its correct targets at genic promoters.

\clearpage
\section{Methods}

\subsection{Yeast strain construction and growth conditions}

All yeast strains were constructed by standard yeast transformation or crosses.
The \textit{spt6-1004} and wild-type strains were grown as previously described \citep{cheung2008}: Cells were grown in YPD at 30\textdegree C to a density of approximately $1 \times 10^7$ cells/ml (OD$_{600} = 0.6$), at which point an equal volume of YPD medium pre-warmed to 44\textdegree C was added, and the cultures were shifted to 37\textdegree C for 80 minutes.

\subsection{Western blotting}

The protocols for western blotting and quantification are described in \citet{doris2018}.

\subsection{Sequencing library preparation\\(TSS-seq, ChIP-nexus, MNase-seq, NET-seq)}

All library preparation methods are detailed in \citet{doris2018}.

\subsection{Genome builds}

The genome build used for \textit{S. cerevisiae} was R64-2-1 \citep{engel2014}, and the genome build used for \textit{S. pombe} was ASM294v2 \citep{wood2002}.

\subsection{TSS-seq data analysis}
\label{subsec:tss_seq}

An up-to-date version of the Snakemake \citep{koster2012} workflow used to process TSS-seq libraries is maintained at \href{https://github.com/winston-lab/tss-seq}{github.com/winston-lab/tss-seq}.
At the time of writing, removal of adapter sequences and random hexamer sequences from the 3$^\prime$ end of the read and 3$^\prime$ quality trimming were performed using cutadapt \citep{martin2011}.
The random hexamer molecular barcode on the 5$^\prime$ end of the read was then removed and processed using a custom Python script (adapted from \citet{mayer2015}).
Reads were aligned to the combined \textit{S. cerevisiae} and \textit{S. pombe} reference genomes using Tophat2 \citep{kim2013} without a reference transcriptome, and uniquely mapping reads were selected using SAMtools \citep{li2009}.
Reads mapping to the same location as another read with the same molecular barcode were identified as PCR duplicates and removed using a custom Python script (adapted from \citet{mayer2015}).
Coverage of the 5$^\prime$-most base, corresponding to the TSS, was extracted using bedtools genomecov \citep{quinlan2010} and normalized to the total number of uniquely mapping, non-duplicate \textit{S. pombe} alignments.
Quality statistics of raw, cleaned, non-aligning, and uniquely aligning non-duplicate reads were assessed using FastQC \citep{andrews2010}.

The pipeline additionally performs \hyperref[subsubsec:tss_peak_calling]{TSS-seq peak calling}, \hyperref[subsubsec:tss_differential_expression]{differential expression}, \hyperref[subsubsec:tss_peak_classification]{classification of peaks into genomic categories}, \hyperref[subsubsec:tss_seqlogos]{sequence logo visualization}, motif enrichment analysis, \hyperref[subsubsec:denovo_motif_discovery]{\textit{de novo} motif discovery}, gene ontology analysis \citep{young2010}, and data visualization with the option to separate data into clusters of similar signal.

\subsubsection{Reannotation of \textit{S. cerevisiae} TSSs using TSS-seq data}
\label{subsubsec:tss_reannotation}

TSS-seq coverage from two replicates of a wild-type \textit{S. cerevisiae} strain grown at 30\textdegree C in YPD was averaged and used to adjust the 5$^\prime$ ends of an annotation of major transcript isoforms based on TIF-seq data \citep{pelechano2013}.
The 5$^\prime$ end of the original annotation was changed to the position of maximum TSS-seq signal in a window $\pm$ 250 nt of the original 5$^\prime$ end if the maximum TSS-seq signal was greater than the 95\textsuperscript{th} percentile of all non-zero TSS-seq signal.

\subsubsection{TSS-seq peak calling}
\label{subsubsec:tss_peak_calling}

TSS-seq data representing transcription from a single promoter tends to occur as a cluster of signal distributed over a range of positions, rather than a single nucleotide \citep{arribere2013, malabat2015}.
It is reasonable to consider such a cluster of TSS-seq signal as a single entity, because the signals within the cluster are usually highly correlated to one another across different conditions.
Therefore, to identify TSSs from TSS-seq data and quantify them for downstream analyses such as differential expression, it is necessary to annotate these groups of TSS-seq signal by using the data to perform peak-calling.

At the time of writing, TSS-seq peak calling for a given experimental group was performed by 1-D watershed segmentation of the data for each sample in the group, followed by filtering for reproducibility within the group by the Irreproducible Discovery Rate (IDR) method \citep{li2011}.
First, a smoothed version of the TSS-seq coverage is generated for each sample using an adaptive two-stage kernel density estimation with a discretized Gaussian kernel \citep{silverman1986}.
For a given nucleotide, the adaptive kernel bandwidth, $\sigma_{\text{adaptive}}$, is given by
\begin{align*}
    \sigma_{\text{adaptive}} &= \sigma_\text{pilot} \left( \frac{\rho_{\text{pilot}}}{g} \right)^{-\alpha},
\end{align*}
where $\sigma_\text{pilot}$ is the standard, fixed bandwidth of a Gaussian kernel used to calculate the pilot signal density $\rho_\text{pilot}$ at that nucleotide, $g$ is the geometric mean of $\rho_\text{pilot}$ over the whole genome, and $\alpha$ is a parameter in $[0,1]$ that determines the degree to which the pilot density $\rho_\text{pilot}$ affects $\sigma_\text{adaptive}$.
The adaptive kernel adjusts the kernel bandwidth to be smaller in regions of high signal density and larger in regions of lower signal density, allowing the smoother to better accommodate both `sharp' TSSs where the signal is distributed over a relatively small window, as well as `broad' TSSs where the signal is more dispersed.
For all analyses in this document, adaptive smoothing was performed with $\sigma_\text{pilot} = 10$ and $\alpha = 0.2$.

Following smoothing, an initial set of peaks is formed by assigning all nonzero signal in the original, unsmoothed coverage to the nearest local maximum of the smoothed coverage, and taking the minimum and maximum genomic coordinates of the original coverage as the peak boundaries for each local maximum of the smoothed coverage.
Peaks are then trimmed to the smallest genomic interval that includes 95\% of the original coverage, and the probability of the peak begin generated by noise is estimated by a Poisson model where $\lambda$, the expected coverage, is the maximum of the expected coverage over the chromosome and the expected coverage in the 2 kb window upstream of the peak (\`a la the ChIP-seq peak caller MACS2 \citep{zhang2008}).
Finally, peaks are ranked by their significance under the Poisson model, and a final list of peaks for the group is generated using the IDR method ($\text{IDR}=0.1$) \citep{li2011}.
In brief, IDR compares ranked lists of regions in order to set a cutoff, beyond which the regions are no longer consistent between replicates.

The python script used for 1-D watershed segmentation of TSS-seq data is \href{https://github.com/winston-lab/tss-seq/blob/master/scripts/tss_peakcalling.py}{available as part of the TSS-seq pipeline}, and the IDR implementation used in the pipeline is also \href{https://github.com/nboley/idr}{available on GitHub}.

\subsubsection{TSS differential expression analysis}
\label{subsubsec:tss_differential_expression}

For TSS-seq differential expression analysis, TSS-seq peak-calling was performed \hyperref[subsubsec:tss_peak_calling]{as described above} for both \textit{S. cerevisiae} and the \textit{S. pombe} spike-in.
The read counts for each peak in each condition were used as the input to differential expression analysis by DESeq2 \citep{love2014}, with the alternative hypothesis $\allowbreak \left\lvert\log_2 \left(\text{fold-change}\right) \right\rvert > 1.5$ and a false discovery rate of 0.1.
To normalize by spike-in, the size factors of the \textit{S. pombe} spike-in counts were used as the size factors for \textit{S. cerevisiae}, although we note that due to the median of ratios normalization used in DESeq2, the major TSS-seq results of this work are still observed when \textit{S. cerevisiae} size factors are used.

\subsubsection{Classification of TSS-seq peaks into genomic categories}
\label{subsubsec:tss_peak_classification}

TSS-seq peaks were assigned to genomic categories based on their position relative to the transcript annotation \hyperref[subsubsec:tss_reannotation]{described above} and an annotation of all verified open reading frames (ORFs) and blocked reading frames in \textit{S. cerevisiae} \citep{engel2014}.
First, `genic' regions were defined as follows: If a gene was present in both the transcript and ORF annotations, the genic region was defined as the interval (annotated TSS-30 nt, start codon).
If gene was present in the transcript annotation but not the ORF annotation, the genic region was defined as the interval (annotated TSS - 30 nt, annotated TSS + 30 nt).
If a gene was present only in the ORF annotation, the genic region was defined as the interval (start codon - 30 nt, start codon).
For the purposes of peak classification, regions were considered overlapping if they had at least one base of overlap.
TSS-seq peaks were classified as genic if they overlapped a genic region on the same strand.
Peaks were classified as intragenic if they were not classified as a genic peak, and their summit position overlapped an open or closed reading frame on the same strand.
Peaks were classified as antisense if their summit position overlapped a transcript on the opposite strand.
Finally, peaks were classified as intergenic if they did not overlap a transcript, reading frame, or genic region on either strand.

\subsubsection{TSS information content and sequence composition}
\label{subsubsec:tss_seqlogos}

TSS-seq alignments were pooled for all replicates in a condition, and the DNA sequence flanking the position of every read overlapping TSS-seq peaks of a particular genomic category was extracted using SAMtools \citep{li2009} and bedtools \citep{quinlan2010}.
The information content and sequence composition of the sequences was quantified using WebLogo \citep{crooks2004}, with the zeroth-order Markov model of the \textit{S. cerevisiae} genomic sequence as the background composition.
Sequence logos were plotting using helper functions from ggseqlogo \citep{wagih2017}.

\subsubsection{Enrichment of the TATA box}

An up-to-date version of the Snakemake \citep{koster2012} workflow used to test the enrichment of motifs is maintained at \href{https://github.com/winston-lab/motif-enrichment}{github.com/winston-lab/motif-\\enrichment}.
To test for enrichment of consensus TATA boxes, FIMO \citep{grant2011} was used to search the \textit{S.cerevisiae} genome for matches to the query motif TATAWAWR (where the ambiguous bases are equiprobable) at a p-value threshold of $6 \times 10^{-4}$.
Regions extending 200 nucleotides upstream of TSS summits were merged if they were overlapping, and were considered to contain a consensus TATA box if the entire motif was overlapping the region on the same strand.
The frequency of motif occurrences in the regions of interest was compared to the frequency of occurrences in the regions upstream of 6000 randomly chosen locations in the genome, using Fisher's exact test.

\subsubsection{\textit{De novo} motif discovery}
\label{subsubsec:denovo_motif_discovery}

\textit{De novo} motif discovery for the regions around TSSs was performed by running MEME-ChIP \citep{machanick2011} on the DNA sequence -100 to +30 nucleotides from the TSS summits of the \hyperref[subsubsec:tss_peak_classification]{genomic classes} of TSSs \hyperref[subsubsec:tss_differential_expression]{significantly upregulated} in \textit{spt6-1004} versus wild-type.

\subsection{ChIP-nexus data analysis}
An up-to-date version of the Snakemake \citep{koster2012} workflow used to process ChIP-nexus libraries is maintained at \href{https://github.com/winston-lab/chip-nexus}{github.com/winston-lab/chip-nexus}.
At the time of writing, filtering for reads containing the constant region of the adapter on the 5$^\prime$ end of the read, 3$^\prime$ adapter removal, and 3$^\prime$ quality trimming were performed using cutadapt \citep{martin2011}.
The random pentamer molecular barcode on the 5$^\prime$ end of the read was then removed and processed using a custom Python script modified from \citet{mayer2015}.
Reads were aligned to the combined \textit{S. cerevisiae} and \textit{S. pombe} genomes using Bowtie 2 \citep{langmead2012}, and uniquely mapping alignments were selected using SAMtools \citep{li2009}.
Reads mapping to the same location as another read with the same molecular barcode were identified as PCR duplicates and removed using a custom Python script modified from \citet{mayer2015}.
Coverage of the 5$^\prime$-most base, corresponding to the point of crosslinking, was extracted using bedtools genomecov \citep{quinlan2010}.
The median fragment size estimated by MACS2 \citep{zhang2008} over all samples was used to generate coverage of factor protection and fragment midpoints, by extending reads to the fragment size, or by shifting reads by half the fragment size, respectively.
Coverage was normalized to the total number of reads uniquely mapping to \textit{S. cerevisiae} (the \textit{S. pombe} spike-in was not used for normalization due to the low number of reads mapping to \textit{S. pombe}).
Quality statistics of raw, cleaned, non-aligning, and uniquely aligning non-duplicate reads were assessed using FastQC \citep{andrews2010}.

The ChIP-nexus pipeline additionally performs \hyperref[subsubsec:nexus_peak_calling]{peak calling}, \hyperref[subsubsec:nexus_differential_occupancy]{differential occupancy analysis}, and data visualization with the option to separate data into clusters of similar signal.

An second Snakemake workflow for TFIIB-specific analyses is maintained at\\\href{https://github.com/winston-lab/chip-nexus-tfiib}{github.com/winston-lab/chip-nexus-tfiib}, and performs \hyperref[subsubsec:tfiib_peak_classification]{classification of TFIIB peaks into genomic categories}, motif enrichment analysis, and gene ontology analysis.

\subsubsection{ChIP-nexus peak calling}
\label{subsubsec:nexus_peak_calling}

A number of tools have been created specifically for peak-calling using data from high-resolution ChIP techniques such as ChIP-nexus and ChIP-exo \citep{wang2014, hansen2016}.
When applied to our TFIIB ChIP-nexus data, these tools tended to split what appeared to be a single TFIIB binding event into multiple peaks.
This may be because TFIIB crosslinks to DNA at multiple points \citep{rhee2012}, suggesting that while these tools may work well for factors that bind symmetrically to DNA with a single crosslinking point on either side, there is room for improvement for factors with more complex crosslinking patterns.

The ChIP-nexus pipeline currently performs peak calling for a condition using the standard ChIP-seq peak caller MACS2 \citep{zhang2008}, followed by filtering for reproduciblity by the Irreproducible Discovery Rate (IDR) method (IDR = 0.1 for all analyses in this chapter) \citep{li2011}.

\subsubsection{TFIIB ChIP-nexus differential occupancy analysis}
\label{subsubsec:nexus_differential_occupancy}
For TFIIB ChIP-nexus differential binding analysis, TFIIB peaks were called by MACS2 and IDR filtering \hyperref[subsubsec:tfiib_peak_calling]{as described above}.
A non-redundant list of peaks called in the condition and control groups being compared was generated using bedtools multiinter \citep{quinlan2010}, and the counts of fragment midpoints for each peak in each sample were used as the input to differential binding analysis by DESeq2 \citep{love2014}, with the alternative hypothesis $\allowbreak \left\lvert\log_2 \left(\text{fold-change}\right) \right\rvert > 1.5$ and a false discovery rate of 0.1.
For estimation of change in TFIIB binding upstream of TSS-seq peaks, TFIIB fragment midpoint counts in the window extending 200 bp upstream of the TSS-seq peak summit were used as the input to DESeq2.
\textit{S. cerevisiae} counts were used for size factor calculation.

\subsubsection{Classification of TFIIB ChIP-nexus peaks into genomic categories}
\label{subsubsec:tfiib_peak_classification}
As for TSS-seq peaks, TFIIB ChIP-nexus peaks were assigned to genomic categories based on their position relative to the transcript annotation \hyperref[subsubsec:tss_reannotation]{described above}, an annotation of all verified open reading frames (ORFs) and blocked reading frames \citep{engel2014}, and \hyperref[subsubsec:tss_peak_classification]{an annotation of `genic' regions derived from the transcript and ORF annotations}.
TFIIB ChIP-nexus peaks were classified as genic if they overlapped a genic region.
Peaks were classified as intragenic if they were not classified as a genic peak, and the entire peak overlapped an open or closed reading frame.
Finally, peaks were classified as intergenic if they did not overlap a transcript, reading frame, or genic region.

\subsection{Comparison of TSS-seq to TFIIB ChIP-nexus}

An up-to-date version of the Snakemake \citep{koster2012} workflow used to compare TSS-seq data to TFIIB ChIP-nexus data is maintained at\\\href{https://github.com/winston-lab/tss-seq-vs-tfiib-nexus}{github.com/winston-lab/tss-seq-vs-tfiib-nexus}.
The pipeline matches and compares peaks from the two assays, and also performs the TFIIB differential occupancy analysis over windows upstream of TSS-seq peaks shown in section \ref{sec:six_tss_tfiib} and described in section \ref{subsubsec:nexus_differential_occupancy}.

\subsection{MNase-seq data analysis}
\label{subsec:mnase_seq}

An up-to-date version of the Snakemake \citep{koster2012} workflow used to demultiplex paired-end MNase-seq libraries is maintained at \href{https://github.com/winston-lab/demultiplex-paired-end}{github.com/winston-lab/demultiplex-paired-end}.
At the time of writing, demultiplexing was performed using fastq-multx \citep{aronesty2013}, allowing one mismatch to the barcode, followed by filtering for and removal of the barcode on read 2 using cutadapt \citep{martin2011}.

An up-to-date version of the Snakemake \citep{koster2012} workflow used to process MNase-seq libraries is maintained at \href{https://github.com/winston-lab/mnase-seq}{github.com/winston-lab/mnase-seq}.
At the time of writing, 3$^\prime$ quality trimming was performed using cutadapt \citep{martin2011}.
Reads were aligned to the combined \textit{S. cerevisiae} and \textit{S. pombe} genome using Bowtie 1 \citep{langmead2009}, and correctly paired alignments were selected using SAMtools \citep{li2009}.
Coverage of nucleosome protection and nucleosome dyads were extracted using bedtools \citep{quinlan2010} and custom shell scripts to get the entire fragment or the midpoint of the fragment, respectively.
Smoothed nucleosome dyad coverage was generated by smoothing dyad coverage with a Gaussian kernel of 20 bp bandwidth.
Coverage was normalized to the total number of correctly paired \textit{S. pombe} fragments.
Quality statistics of raw, cleaned, non-aligning, and correctly paired reads were assessed using FastQC \citep{andrews2010}.

The MNase-seq pipeline additionally performs \hyperref[subsubsec:nucleosome_quantification]{quantification of nucleosome properties}, and data visualization with the option to separate data into clusters of similar signal.

\subsubsection{Quantification of nucleosome properties}
\label{subsubsec:nucleosome_quantification}
Quantification of nucleosome occupancy, fuzziness, and position shifts were calculated using DANPOS2 \citep{chen2013}, with spike-in normalization by scaling the total counts in condition group libraries by
\begin{align*}
    \frac{\text{mean observed percent spike-in in condition libraries}}{\text{mean observed percent spike-in in control libraries}}.
\end{align*}

\subsubsection{Clustering of MNase-seq signal at \textit{spt6-1004} intragenic TSSs}

The Snakemake \citep{koster2012} workflow for clustering MNase-seq data by self/super-organizing map and hierarchical clustering is maintained at\\\href{https://github.com/winston-lab/cluster-mnase-seq}{github.com/winston-lab/cluster-mnase-seq}.
To cluster \textit{spt6-1004} intragenic TSSs based on surrounding MNase-seq signal, spike-in normalized MNase-seq dyad signal in the window $\pm150$ bp of the TSS-seq peak summit of all intragenic TSS-seq peaks significantly upregulated in \textit{spt6-1004} was binned by taking the mean signal in non-overlapping 5 bp bins, and then averaged taking the mean of two replicates (\textit{spt6-1004}) or one experiment (wild-type).
The data were then standardized over each TSS, and the wild-type and \textit{spt6-1004} data were used as equally weighted input layers to a super-organizing map \citep{wehrens2007} trained on the input data to assign similar MNase-seq observations in 60-dimensional input space to similar nodes in a 2-dimensional ($6 \times 8$) rectangular grid.
The 48 `code vectors' representing the typical MNase-seq pattern for each node (visualized in Figure \ref{fig:six_mnase_som}) were then clustered by agglomerative hierarchical clustering using sum of squares distance and Ward linkage.
The resulting dendrogram was cut to produce the three clusters of MNase-seq signal shown in Figures \ref{fig:six_mnase_som} and \ref{fig:six_intragenic_mnase_metagenes}.

\subsection{NET-seq data analysis}
\label{subsec:net_seq}

An up-to-date version of the Snakemake \citep{koster2012} workflow used to process NET-seq libraries is maintained at \href{https://github.com/winston-lab/net-and-rna-seq}{github.com/winston-lab/net-and-rna-seq}.

At the time of writing, removal of adapter sequences from the 3$^\prime$ end of the read and 3$^\prime$ quality trimming were performed with cutadapt \citep{martin2011}.
Reads were aligned to the \textit{S. cerevisiae} genome using Tophat2 \citep{kim2013} without a reference transcriptome, and uniquely mapping reads were selected using SAMtools \citep{li2009}.
Coverage of the 5$^\prime$-most base of the read, corresponding to the 3$^\prime$-most base of the nascent RNA and the active site of elongating RNA polymerase, was extracted using bedtools genomecov \citep{quinlan2010} and normalized to the total number of uniquely mapped reads.
Quality statistics of raw, cleaned, non-aligning, and uniquely aligning reads were assessed using FastQC \citep{andrews2010}.

The NET-seq pipeline additionally performs \textit{ab initio} transcript annotation \citep{pertea2015}, differential expression analysis, and data visualization with the option to split data into clusters of similar signal.
For libraries with unique molecular barcodes and/or spike-ins, the pipeline also handles PCR duplicate removal and spike-in normalization, respectively.

\clearpage
\bibliographystyle{apalike}
\begingroup
    \singlespacing
    \bibliography{references/spt6}
\endgroup

\cleardoublepage

\chapter{Genomics of transcription elongation factor Spt5}

\section{Collaborators}

\begin{description}[align=right, leftmargin=!, labelwidth=5cm, noitemsep]
    \item [Ameet Shetty] generated TSS-seq, MNase-seq, NET-seq,\\RNA-seq, and ChIP-seq libraries
\end{description}

\newpage
\bibliographystyle{apalike}
\begingroup
\singlespacing
\bibliography{references/spt5}
\endgroup

\cleardoublepage

\chapter{Stress-responsive intragenic transcription}
\label{chapter:stress}

\section{Abstract}

\lipsum[1]

\section{Collaborators}

\begin{description}[align=right, labelwidth=5cm, noitemsep, leftmargin=!]
    \item [Steve Doris] generated TSS-seq and ChIP-nexus libraries
    \item [Dan Spatt] polyribosome fractionation, fitness competitions,\\and other experiments
    \item [James Warner] fitness competitions and other experiments
\end{description}

\section{Possible functions for intragenic transcription in wild-type cells}

ASE1 \citep{mcknight2014}.
KAR4 \citep{gammie1999}.
CIK1 \citep{benanti2009}.
ASP3 \citep{huang2010}.

\clearpage

\section{Discovery of stress-induced intragenic promoters by TFIIB ChIP-nexus and TSS-seq}

\begin{wrapfigure}[20]{r}{3in}
    \centering
    \includegraphics[width=3in]{figures/stress/stress_gasch_comparison.pdf}
    \caption[Scatterplots comparing change in genic TFIIB signal to change in RNA microarray signal, for oxidative and amino acid stresses.]{Scatterplots comparing change in genic TFIIB signal to change in RNA microarray signal from \citet{gasch2000}, for oxidative and amino acid stresses. The Pearson correlation coefficient is shown for each comparison.}
    \label{fig:stress_gasch_comparison}
\end{wrapfigure}
To discover cases of stress-induced intragenic transcription initiation, we performed ChIP-nexus of TFIIB for wild-type yeast under conditions of oxidative stress, amino acid stress, and nitrogen stress, along with controls of growth in rich YPD medium and defined SC medium.
The genic TFIIB response to each of the stresses either correlates well with the expected transcriptomic response to the stress (Figure \ref{fig:stress_gasch_comparison}), or is enriched for metabolic pathways consistent with the cellular response to the stress (Figure \ref{fig:stress_nitrogen_gene_ontology}), confirming that TFIIB ChIP-nexus captures changes in transcription initiation and that the cells were stressed as intended.
In total, we identify 140 intragenic TFIIB peaks significantly induced at least 1.5-fold in at least one stress condition, with some peaks being induced in more than one stress (Figures \ref{fig:stress_tfiib_coverage}, \ref{fig:stress_tfiib_ridgelines}).
We also observe a slight negative correlation between stress-induced changes in intragenic TFIIB signal and changes in the corresponding genic TFIIB signal (Figure \ref{fig:stress_genic_vs_intra}).
\begin{figure}[h]
    \centering
    \includegraphics[width=4in]{figures/stress/stress_nitrogen_gene_ontology.pdf}
    \caption[Gene ontology terms enriched in genes with nitrogen-stress-induced genic TFIIB peaks]{Gene ontology terms enriched in genes with nitrogen-stress-induced genic TFIIB peaks.}
    \label{fig:stress_nitrogen_gene_ontology}
\end{figure}

\begin{figure}[h]
    \includegraphics[width=6in]{figures/stress/stress_tfiib_coverage.pdf}
    \label{fig:stress_tfiib_coverage}
    \caption[TFIIB ChIP-nexus protection over four genes with stress-induced intragenic TFIIB peaks.]{Relative TFIIB ChIP-nexus protection over four genes with an intragenic TFIIB peak significantly induced in one or more of the stress conditions.}
\end{figure}

\begin{sidewaysfigure}
    \includegraphics[width=8.25in]{figures/stress/stress_tfiib_ridgelines.pdf}
    \caption[TFIIB ChIP-nexus protection over all genes with stress-induced intragenic TFIIB peaks.]{Relative TFIIB ChIP-nexus protection over all genes with an intragenic TFIIB peak significantly induced in one or more of the stress conditions tested, as depicted in the left panel. Genes are aligned by start codon, and are sorted within each group by the distance from the start codon to the summit of the induced intragenic TFIIB peak. Data are shown for each gene up to the stop codon of the gene. Regions where TFIIB peaks are called are shaded in the stress conditions according to the fold-change of the peak relative to the corresponding control condition.}
    \label{fig:stress_tfiib_ridgelines}
\end{sidewaysfigure}

\begin{figure}[h]
    \includegraphics[width=4in]{figures/stress/stress_genic_vs_intra.pdf}
    \label{fig:stress_genic_vs_intra}
    \caption[Scatterplot of change in intragenic versus genic TFIIB ChIP-nexus signal, for all pairs of intragenic and genic TFIIB peaks in the three stress conditions.]{Scatterplot comparing change in intragenic TFIIB ChIP-nexus signal to the change in genic TFIIB signal at the same gene, for all pairs of intragenic and genic TFIIB peaks in the three stress conditions. Error bars indicate $\pm$ 1 standard error, and the Pearson correlation coefficient is shown.}
\end{figure}

Because the greatest changes to intragenic transcription initiation were detected in oxidative stress, we focused on this condition and performed TSS-seq to determine which intragenic initiation events produce stable RNAs and in which strand orientation these events occur.
Considering only TSS peaks with a TFIIB peak overlapping the window extending 200 base pairs upstream of the TSS summit, we find cases of both sense intragenic and antisense TSSs that are differentially expressed in oxidative stress (Figure \ref{fig:stress_promoter_tss_diffexp_summary}).
In general, intragenic TSSs are expressed at lower levels than genic TSSs: Among oxidative-stress-induced TSSs, the most abundant intragenic TSS in oxidative stress is present at levels comparable to the 54\textsuperscript{th} percentile of genic TSS abundances (Figure \ref{fig:stress_promoter_tss_expression}).

\begin{figure}[h]
    \includegraphics[width=3in]{figures/stress/stress_promoter_tss_diffexp_summary.pdf}
    \caption[Bar plot of the number of promoters in various genomic classes differentially expressed in oxidative stress.]{Bar plot of the number of promoters in various genomic classes differentially expressed in oxidative stress.}
    \label{fig:stress_promoter_tss_diffexp_summary}
\end{figure}

\begin{figure}[h]
    \includegraphics[width=6in]{figures/stress/stress_promoter_tss_expression.pdf}
    \caption[TSS-seq expression levels in oxidative stress of oxidative-stress-induced genic and intragenic promoters.]{Cumulative distributions of TSS-seq expression levels in oxidative stress, for all genic and intragenic promoters significantly induced in oxidative stress. Error bars indicate $\pm$ one standard deviation.}
    \label{fig:stress_promoter_tss_expression}
\end{figure}

\clearpage

% \section{Chromatin landscape of oxidative-stress-induced promoters.}

% \lipsum[1]

% \begin{figure}
% % \includegraphics[width=6in]{figures/stress/stress_promoter_tss_expression.pdf}
% \caption[A figure showing TSS-seq, TFIIB ChIP-nexus, and MNase-ChIP-seq for the oxidative-stress-induced promoters.]{Caption dsafklj .}
% % \label{fig:stress_promoter_tss_diffexp_summary}
% \end{figure}

\section{Polysome enrichment of oxidative-stress-induced intragenic transcripts}

Translation of a transcript requires that the transcript possesses both a 5$^\prime$-cap and a poly-A tail.
Since the TSS-seq protocol enriches for both of these features, this implies that the oxidative-stress-dependent intragenic transcripts we detect by TSS-seq could potentially be translated.
For sense strand intragenic transcripts, this could generate N-terminally-truncated protein isoforms.
To see whether oxidative-stress-dependent intragenic transcripts are actually translated, we performed sucrose gradient fractionation, isolated the RNA associated with the polysome fraction, and sequenced TSS-seq libraries of the polysome-associated RNA.
Among oxidative-stress-induced TSSs with corresponding TFIIB peaks, intragenic TSSs in oxidative stress are less enriched in the polysome fraction when compared to genic TSSs (Figure \ref{fig:stress_promoter_tss_polyenrichment}).
However, during oxidative stress, half of the oxidative-stress-induced intragenic TSSs are present in polysomes at levels greater than the 25\textsuperscript{th} percentile of oxidative-stress-induced genic TSSs, indicating that many of the intragenic transcripts are translated at some level.

\begin{figure}[h]
    \includegraphics[width=6in]{figures/stress/stress_promoter_tss_polyenrichment.pdf}
    \caption[Relative polysome enrichment in oxidative stress, for oxidative-stress-induced genic and intragenic promoters.]{Relative polysome enrichment in oxidative stress, for oxidative-stress-induced genic and intragenic promoters. Error bars indicate $\pm$ one standard error.}
    \label{fig:stress_promoter_tss_polyenrichment}
\end{figure}

\section{Functions of intragenic DSK2 expression in oxidative stress}

\lipsum[1]

\begin{figure}[h]
    \includegraphics[width=6in]{figures/stress/stress_dsk2_summary.pdf}
    \caption[A figure showing TSS-seq, TFIIB ChIP-nexus, and MNase-ChIP-seq at DSK2.]{Caption dsafklj .}
    \label{fig:stress_dsk2_summary}
\end{figure}

\begin{figure}
% \includegraphics[width=6in]{figures/stress/stress_promoter_tss_expression.pdf}
\caption[A figure showing DSK2 fitness competition results.]{Caption dsafklj .}
% \label{fig:stress_promoter_tss_diffexp_summary}
\end{figure}


\section{TSS-seq analysis of oxidative stress in \textit{Saccharomyces sensu stricto} species}

\lipsum[1]

\begin{figure}
    % \includegraphics[width=6in]{figures/stress/stress_promoter_tss_expression.pdf}
    \caption[A figure showing TSS-seq coverage over oxidative-stress-induced TSSs in the three species.]{Caption dsafklj .}
    % \label{fig:stress_promoter_tss_diffexp_summary}
\end{figure}

\begin{figure}
    % \includegraphics[width=6in]{figures/stress/stress_promoter_tss_expression.pdf}
    \caption[A figure showing TSS-seq coverage over DSK2 in the three species, possibly with the corresponding northern blot.]{Caption dsafklj .}
    % \label{fig:stress_promoter_tss_diffexp_summary}
\end{figure}

\section{Discussion}

\lipsum[1]

\section{Methods}

\subsection{Yeast growth conditions}

\subsection{Genome builds}

\subsection{TFIIB ChIP-nexus data analysis}

\subsection{TSS-seq data analysis}

\subsection{MNase-ChIP-seq data analysis}

\subsection{Sucrose gradient fractionation}

\subsection{Polysome-associated TSS-seq analysis}

\subsection{Multiple genome alignment}

\subsection{Diamide competitive fitness assays}

\newpage
\bibliographystyle{apalike}
\begingroup
\singlespacing
\bibliography{references/stress}
\endgroup

\cleardoublepage

\newpage
\addcontentsline{toc}{chapter}{Bibliography}
\begingroup
    \singlespacing
    \bibliography{}
\endgroup
\cleardoublepage


\chapter*{Vita}

\textbf{James Chuang}

\begin{description}[align=right, labelwidth=4cm, noitemsep, leftmargin=!]
    \item [year of birth:] 1991
    \item [contact address:] 77 Avenue Louis Pasteur\\
                             Room 239\\
                             Boston, MA 02115
\end{description}
\noindent\hrulefill

\section*{Education}
\begin{description}[align=right, labelwidth=1.5cm, noitemsep, leftmargin=!]
    \item [2018] MSc, Biomedical Engineering, Boston University
    \item [2013] BSc, Biomedical Engineering, Johns Hopkins University
\end{description}


\section*{Publications}
\begingroup
\singlespacing
\begin{description}[align=right, labelwidth=1.5cm, itemsep=1em, leftmargin=!]
    \item [2018] Doris SM*, \textbf{Chuang J}*, Viktorovskaya O, Murawska M, Spatt D, Churchman LS, Winston F (2018). Spt6 is required for the fidelity of promoter selection. \textbf{Molecular Cell}, doi:\href{https://doi.org/10.1016/j.molcel.2018.09.005}{10.1016/j.molcel.2018.09.005}
    \item [2018] \textbf{Chuang J}, Boeke JD, Mitchell LA (2018). Coupling Yeast Golden Gate and VEGAS for Efficient Assembly of the Violacein Pathway in \textit{Saccharomyces cerevisiae}. \textbf{Synthetic Metabolic Pathways}, doi:\href{https://doi.org/10.1007/978-1-4939-7295-1_14}{10.1007/978-1-4939-7295-1\_14}
    \item [2017] Aquino P, Honda B, Suma Jaini, Lyubetskaya A, Hosur K, Chiu JG, Ekladious I, Hu D, Jin L, Sayeg MK, Stettner AI, Wang J, Wong BG, Wong WS, Alexander SL, Ba C, Bensussen SI, Chou K, \textbf{Chuang J}, Gastler DE, Grasso DJ, Greifenberger JS, Guo C, Hawes AK, Israni DV, Jain SR, Kim J, Lei J, Li H, Li D, Li Q, Mancuso CP, Mao N, Masud SF, Meisel CL, Mi J, Nykyforchyn CS, Park M, Peterson HM, Ramirez AK, Reynolds DS, Rim NG, Saffie JC, Su H, Su WR, Su Y, Sun M, Thommes MM, Tu T, Varongchayakul N, Wagner TE, Weinberg BH, Yang R, Yaroslavsky A, Yoon C, Zhao Y, Zollinger AJ, Stringer AM, Foster JW, Wade J, Raman S, Broude N, Wong WW, Galagan JE (2017). Coordinated regulation of acid resistance in \textit{Escherichia coli}. \textbf{BMC Systems Biology}, doi:\href{https://doi.org/10.1186/s12918-016-0376-y}{10.1186/s12918-016-0376-y}
    \item [2015] Mitchell, LA*, \textbf{Chuang J}*, Agmon N, Khunsriraksakul C, Phillips NA, Cai Y, Truong DM, Veerakumar A, Wang Y, Mayorga M, Blomquist P, Sadda P, Trueheart J, Boeke JD (2015). Versatile genetic assembly system (VEGAS) to assemble pathways for expression in \textit{S. cerevisiae}. \textbf{Nucleic Acids Research}, doi:\href{https://doi.org/10.1093/nar/gkv466}{10.1093/nar/gkv466}
    \item [2015] Agmon N, Mitchell LA, Cai Y, Ikushima S, \textbf{Chuang J}, Zheng A, Choi W, Martin JA, Caravelli K, Stracquadanio G, Boeke JD (2015). Yeast Golden Gate (yGG) for the Efficient Assembly of \textit{S. cerevisiae} Transcription Units. \textbf{ACS Synthetic Biology}, doi:\href{https://doi.org/10.1021/sb500372z}{10.1021/sb500372z}
    \item [2013] Mitchell LA, Cai Y, Taylor M, Noronha AM, \textbf{Chuang J}, Dai L, Boeke JD (2013). Multichange isothermal mutagenesis: a new strategy for multiple site-directed mutations in plasmid DNA. \textbf{ACS Synthetic Biology}, \\doi:\href{https://doi.org/10.1021/sb300131w}{10.1021/sb300131w}
\end{description}
\endgroup


\end{document}
