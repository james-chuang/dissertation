\documentclass[12pt,letterpaper,oneside]{book}
\pagestyle{plain}

\usepackage{sectsty}

\chapternumberfont{\normalsize}
\chaptertitlefont{\normalsize}
\sectionfont{\normalsize}
\subsectionfont{\normalsize}

\usepackage{bu_ece_thesis_minimal}

\usepackage{setspace}
\doublespacing

\usepackage{amsmath}
\usepackage{amssymb}
\usepackage{mathspec}
\setmainfont{FreeSans}[
    Path=fonts/,
    BoldFont=FreeSansBold,
    ItalicFont=FreeSansOblique,
    BoldItalicFont=FreeSansBoldOblique
]
\setmathsfont(Digits)[Path=fonts/]{FreeSans}
\setmathrm[Path=fonts/]{FreeSans}

\usepackage[paper=letterpaper,
            layout=letterpaper,
            lmargin=1.5in,
            rmargin=1.0in,
            tmargin=1.5in,
            bmargin=1.25in,
            footskip=0.50in]{geometry}

\usepackage{enumitem}

\usepackage{graphicx}
\usepackage{float}
\usepackage{wrapfig}
\usepackage{caption}
\captionsetup{font={normalsize, stretch=1.3}, singlelinecheck=off}
\usepackage{sidecap}
\usepackage{rotating}

\usepackage{xcolor}
\definecolor{blue}{HTML}{114477}
\definecolor{grey}{HTML}{aaaaaa}
\definecolor{lightpurple}{HTML}{beaed4}
\usepackage[colorlinks=true,
            linkcolor=black,
            urlcolor=blue,
            citecolor=grey,
            backref,
            linktoc=all,
            breaklinks=false,
            pdftitle=Title\ of\ thesis,
            pdfcreator=James\ Chuang]{hyperref}

\usepackage[round, sort]{natbib}
\setcitestyle{aysep={,}, citesep={;}}
\usepackage[]{chapterbib}
\sectionbib{\section}{section}

\usepackage{tikz}
\usetikzlibrary{shapes}

\usepackage{lipsum}

\setcounter{secnumdepth}{3}
\setcounter{tocdepth}{3}

\begin{document}
\bibliographystyle{apalike}

\frontmatter


% Type of document prepared for this degree:
%   1 = Master of Science thesis,
%   2 = Doctor of Philosophy dissertation.
%   3 = Master of Science thesis and Doctor of Philisophy dissertation.
\degree=2

\prevdegrees{B.S., Johns Hopkins University, 2013\\
	M.S., Boston University, 2018}

\department{Department of Biomedical Engineering}

% Degree year is the year the diploma is expected, and defense year is
% the year the dissertation is written up and defended. Often, these
% will be the same, except for January graduation, when your defense
% will be in the fall of year X, and your graduation will be in
% January of year X+1
\defenseyear{2019}
\degreeyear{2019}

% For each reader, specify appropriate label {First, Second, Third},
% then name, and title. IMPORTANT: The title should be:
%   "Professor of Electrical and Computer Engineering",
% or similar, but it MUST NOT be:
%   Professor, Department of Electrical and Computer Engineering"
% or you will be asked to reprint and get new signatures.
% Warning: If you have more than five readers you are out of luck,
% because it will overflow to a new page. You may try to put part of
% the title in with the name.
\reader{First}{Fred Winston, Ph.D.}{John Emory Andrus Professor of Genetics\\Harvard Medical School}
\reader{Second}{Ahmad S. Khalil, Ph.D.}{Associate Professor of Biomedical Engineering}
\reader{Third}{L. Stirling Churchman, Ph.D.}{Associate Professor of Genetics\\Harvard Medical School}
\reader{Fourth}{John T. Ngo, Ph.D.}{Assistant Professor of Biomedical Engineering}
\reader{Fifth}{Wilson W. Wong, Ph.D.}{Associate Professor of Biomedical Engineering}

% The Major Professor is the same as the first reader, but must be
% specified again for the abstract page. Up to 4 Major Professors
% (advisors) can be defined.
\numadvisors=2
\majorprof{Fred Winston, PhD}{{John Emory Andrus Professor of Genetics\\Harvard Medical School}}
\majorprofb{Ahmad S. Khalil, PhD}{{Associate Professor of Biomedical Engineering}}
%\majorprofc{First M. Last, PhD}{{Professor of Astronomy}}
%\majorprofd{First M. Last, PhD}{{Professor of Biomedical Engineering}}

\maketitle
\cleardoublepage

% The copyright page is blank except for the notice at the bottom. You
% must provide your name in capitals.
\copyrightpage
\cleardoublepage

% Now include the approval page based on the readers information
\approvalpage
\cleardoublepage

% % The acknowledgment page should go here. Use something like
% % \newpage\section*{Acknowledgments} followed by your text.
% \newpage
% \section*{\centerline{Acknowledgments}}
% \vskip 1in
% \noindent
% James Chuang
% \cleardoublepage

% The abstractpage environment sets up everything on the page except
% the text itself.  The title and other header material are put at the
% top of the page, and the supervisors are listed at the bottom.  A
% new page is begun both before and after.  Of course, an abstract may
% be more than one page itself.  If you need more control over the
% format of the page, you can use the abstract environment, which puts
% the word "Abstract" at the beginning and single spaces its text.

\begin{abstractpage}

Transcription of protein-coding genes in eukaryotic cells is carried out by the protein complex RNA polymerase II.
During the elongation phase of transcription, RNA polymerase II associates with transcription elongation factors which modulate the activity of the transcription complex and are needed to carry out co-transcriptional processes.
Chapters \ref{chapter:six} and \ref{chapter:five} of this dissertation describe studies of Spt6 and Spt5, two conserved transcription elongation factors.

Spt6 is a transcription elongation factor thought to replace nucleosomes in the wake of transcription.
\textit{Saccharomyces cerevisiae} \textit{spt6} mutants express elevated levels of intragenic transcripts, transcripts appearing to initiate from within gene bodies.
We applied high resolution genomic assays of transcription initiation to an \textit{spt6-1004} mutant, allowing us to catalog the full extent of intragenic transcription in \textit{spt6-1004} and show for the first time on a genome-wide scale that the intragenic transcripts observed in \textit{spt6-1004} are largely explained by new transcription initiation.
We also assayed chromatin structure genome-wide in \textit{spt6-1004}, finding a global depletion and disordering of nucleosomes.
In addition to increased intragenic transcription in \textit{spt6-1004}, our results also reveal an unexpected decrease in expression from most canonical genic promoters.
Comparing intragenic and genic promoters, we find that intragenic promoters share some features with genic promoters.
Altogether, we propose that the transcriptional changes in \textit{spt6-1004} are explained by a competition for transcription initiation factors between genic and intragenic promoters, which is made possible by a global decrease in nucleosome protection of the genome.

Spt5 is another transcription elongation factor, important for the processivity of the transcription complex and many transcription-related processes.
To study the requirement for Spt5 \textit{in vivo}, we applied multiple genomic assays to \textit{Schizosaccharomyces pombe} cells depleted of Spt5.
Our results reveal an accumulation of RNA polymerase II over the 5$^\prime$ ends of genes upon Spt5 depletion, and a progressive decrease in transcript abundance towards the 3$^\prime$ ends of genes.
This is consistent with a model in which Spt5 depletion causes transcription elongation defects and increases early termination.
We also unexpectedly discover that Spt5 depletion causes hundreds of antisense transcripts to be expressed across the genome, primarily initiating from within the first 500 base pairs of genes.

The expression of intragenic transcripts when transcription elongation factors are disrupted suggests that cells have evolved to prevent spurious intragenic transcription.
However, some cases of intragenic transcription are consistently detected in wild-type cells, and some of these cases are known to be important for different biological functions.
Chapter \ref{chapter:stress} of this dissertation describes our efforts to better understand the functions of intragenic transcription in wild-type cells by studying uncharacterized instances of intragenic transcription.
To discover uncharacterized instances of intragenic transcription, we applied high resolution genomic assays of transcription initiation to wild-type \textit{Saccharomyces cerevisiae} under three stress conditions.
For the condition of oxidative stress, we show that intragenic transcripts are generally expressed at lower levels than genic transcripts, and that many intragenic transcripts are likely to be translated at some level.
By comparing intragenic transcription in three yeast species, we find that most examples of oxidative-stress regulated intragenic transcription identified in \textit{S. cerevisiae} are not conserved.
Finally, we show that the expression of an oxidative-stress-induced intragenic transcript at the gene \textit{DSK2} is needed for \textit{S. cerevisiae} to survive in conditions of oxidative stress.

\end{abstractpage}
\cleardoublepage

\tableofcontents
\cleardoublepage

% \listoftables
% \addcontentsline{toc}{chapter}{List of Tables}
% \cleardoublepage

\listoffigures
\addcontentsline{toc}{chapter}{List of Figures}
\cleardoublepage


\cleardoublepage

\mainmatter
\chapter{Introduction}

\section{A brief introduction to transcription}

In eukaryotic cells, transcription of protein-coding genes is carried out by the protein complex RNA polymerase II

\lipsum[1]

\section{Transcription elongation factors Spt6 and Spt5}

\lipsum[1]

\section{Reproducible data analysis for genomics}

\lipsum[1]


\cleardoublepage

\chapter{Genomics of transcription elongation factor Spt6}

\section{Collaborators}

\begin{description}[align=right, labelwidth=5cm, noitemsep]
    \item [Steve Doris] optimized TSS-seq and ChIP-nexus protocols
    \item [] generated TSS-seq and ChIP-nexus libraries
    \item [Olga Viktorovskaya] generated MNase-seq libraries
    \item [Magdalena Murawska] generated NET-seq libraries
    \item [Dan Spatt] various experiments for publication
\end{description}

\section{Introduction to Spt6 and intragenic transcription}

% The work described in this chapter relates to understanding how a eukaryotic cell specifies which sites in its genome are permitted to become sites of transcription initiation.
% To get a rough idea of the specificity of transcription initiation, it is useful to start with a simple back-of-the-envelope calculation of the proportion of the human genome at which transcription initiation occurs.
% The human genome is approximately three billion base pairs in length, and each base pair can potentially be transcribed from each of its two strands.
% Each gene in the genome can have multiple transcription start sites (TSSs), which I assume to be five in number for the average gene.
% At last count, the human genome contains about twenty thousand protein-coding genes.
% To be conservative in our estimate with regards to specificity, we assume that all twenty thousand genes are expressed, which leads to the following proportion:
% \begin{align*}
%     \frac{\left(2 \times 10^4 \; \text{genes}\right) \left(5 \; \frac{\text{TSS}}{\text{gene}} \right)}
%          {\left(3 \times 10^9 \; \text{base pairs} \right) \left(2 \; \frac{\text{TSS}}{\text{base pair}} \right)}.
% \end{align*}
% However, this expression underestimates the extent of transcription initiation by only considering protein-coding genes, neglecting the many classes of noncoding genes present in the genome.
% If we assume that there are five noncoding genes for each coding gene, the updated expression,
% \begin{align*}
%     \frac{\left(1.2 \times 10^5 \; \text{genes}\right) \left(5 \; \frac{\text{TSS}}{\text{gene}} \right)}
%          {\left(3 \times 10^9 \; \text{base pairs} \right) \left(2 \; \frac{\text{TSS}}{\text{base pair}} \right)}
%     &= 0.0001,
% \end{align*}
% says that when presented with a thousand positions to choose from, RNA polymerase chooses just one to start transcribing from!

% Where transcription initiates is determined in large part by DNA sequence: the presence of certain sequence motifs increases the probability that RNA polymerase binds to DNA along with numerous co-factors required for initiation.
% However, DNA sequence alone does not entirely account for the specificity of transcription initiation.
% Genetic studies in yeast first showed that some transcription \textit{elongation} factors, including histone chaperones and histone modification enzymes, play a role in restricting where transcription is allowed to initiate \citep{kaplan2003, cheung2008, hennig2013}.
% In this project, we study the role of a particular transcription elongation factor called \textbf{Spt6} in this process.
% Many years of research on Spt6 is summarised as follows \citep{doris2018}:

% \begin{itemize}[nosep, topsep=.5em]
% \item Spt6 interacts directly with:
% 	\begin{itemize}[nosep]
% 	\item RNA polymerase II (RNAPII) \citep{close2011, diebold2011, liu2011, sdano2017, sun2010, yoh2007}
% 	\item histones \citep{bortvin1996, mccullough2015}
% 	\item the essential factor Spn1 (IWS1) \citep{diebold2010b, li2018, mcdonald2010}
% 	\end{itemize}
% \item Spt6 is believed to function primarily as an elongation factor based on:
% 	\begin{itemize}[nosep]
% 	\item association with elongating RNAPII \citep{andrulis2000, ivanovska2011, kaplan2000, mayer2010}
% 	\item ability to enhance elongation in vitro \citep{endoh2004} and in vivo \citep{ardehali2009}
% 	\end{itemize}
% \item Spt6 has been shown to regulate initiation in some cases \citep{adkins2006, ivanovska2011}
% \item Spt6 regulates:
% 	\begin{itemize}[nosep]
% 	\item chromatin structure \citep{bortvin1996, degennaro2013, ivanovska2011, jeronimo2015, kaplan2003, perales2013, vanbakel2013}
% 	\item histone modifications, including:
% 		\begin{itemize}[nosep]
% 		\item H3K36 methylation \citep{carrozza2005, chu2006, yoh2008, youdell2008}
% 		\item in some organisms, H3K4 and H3K27 methylation \citep{begum2012, chen2012, degennaro2013, wang2017, wang2013}
% 		\end{itemize}
% 	\end{itemize}
% \item Spt6 is likely a histone chaperone required to reassemble nucleosomes in the wake of transcription \citep{duina2011}.
% \end{itemize}

\begin{wrapfigure}[18]{r}{3in}
    \centering
    \includegraphics[width=3in]{figures/six/six_spt6_western.pdf}
    \caption[Western blot showing Spt6 protein levels in wild-type and \textit{spt6-1004} cells, at 30\textdegree C and after 80 minutes at 37 \textdegree C.]{Western blot showing Spt6 protein levels in wild-type and \textit{spt6-1004} cells, at 30\textdegree C and after 80 minutes at 37 \textdegree C. Immunoblotting was performed using $\alpha$-FLAG antibody to detect Spt6 and $\alpha$-Myc antibody to detect Dst1 from a spike-in strain. The quantification shown is the mean $\pm$ standard deviation of three blots.}
    \label{fig:six_spt6_western}

    \vspace{0.5em}

    \centering
    \includegraphics[width=3in]{figures/six/six_gene_diagram.pdf}
    \caption[Diagram of transcript classes.]{Diagram of transcript orientation with respect to coding DNA sequences, for the categories of transcripts referred to in this document.}
    \label{fig:six_gene_diagram}
\end{wrapfigure}

Studies in the yeasts \textit{Saccharomyces cerevisiae} and \textit{Schizosaccharomyces pombe} have previously examined the requirement for Spt6 in normal transcription \citep{cheung2008, degennaro2013, kaplan2003, pathak2018, uwimana2017, vanbakel2013}.
As Spt6 is essential for viability in \textit{S. cerevisiae}, many of these studies use the same temperature-sensitive \textit{spt6} mutant used in this project, \textbf{\textit{spt6-1004}}, which encodes an in-frame deletion of a helix-hairpin-helix domain within Spt6 \citep{kaplan2003}.
When \textit{spt6-1004} cells are shifted from 30\textdegree C to 37\textdegree C for 80 minutes, bulk Spt6 protein levels are depleted to about 20\% of wild-type levels (Figure \ref{fig:six_spt6_western}).
The most notable phenotype of the \textit{spt6-1004} mutant is the appearance of \textbf{intragenic transcripts}, transcripts which appear to arise from within protein-coding sequences, in both sense and antisense orientations relative to the coding gene (Figure \ref{fig:six_gene_diagram}) \citep{cheung2008, degennaro2013, kaplan2003, uwimana2017}.

\begin{SCfigure}[40][h]
    \centering
    \includegraphics[width=4.4in]{figures/six/six_aat_assay_comparison.pdf}
    \caption[RNA-seq, TSS-seq, and TFIIB ChIP-nexus signal at the \textit{AAT2} gene, in \textit{spt6-1004} after 80 minutes at 37\textdegree C.]{Sense strand RNA-seq signal, sense strand TSS-seq signal, and TFIIB ChIP-nexus protection at the \textit{AAT2} gene, in \textit{spt6-1004} after 80 minutes at 37\textdegree C.}
    \label{fig:six_aat_assay_comparison}
\end{SCfigure}

Previous genome-wide measurements of transcript levels in \textit{spt6-1004} relied on tiled microarrays \citep{cheung2008} and RNA sequencing \citep{uwimana2017}.
Studying intragenic transcription is difficult with these methods, since the signal for an intragenic transcript in the same orientation as the gene it overlaps is convoluted with the signal from the full-length `genic' transcript (Figure \ref{fig:six_aat_assay_comparison}) \citep{cheung2008, lickwar2009}. Therefore, these methods can only discover intragenic transcripts which are highly expressed relative to the corresponding genic transcript, and are likely to find many false positives.
Additionally, these methods are assays of steady-state RNA levels, which makes them unable to distinguish whether the intragenic transcripts observed in \textit{spt6-1004} result from: A) new intragenic transcription initiation in the mutant, B) reduced decay of intragenic transcripts which are rapidly degraded in wild-type, or C) processing of full-length protein-coding RNAs.

To address these challenges to studying intragenic transcription, we applied two genomic assays to \textit{spt6-1004}: transcription start-site sequencing (\textbf{TSS-seq}), and \textbf{ChIP-nexus of TFIIB}, a component of the RNA polymerase II pre-initiation complex (PIC).
TSS-seq sequences the 5$^\prime$ end of capped and polyadenylated RNAs \citep{arribere2013, malabat2015}, allowing separation of intragenic from genic RNA signals and identification of intragenic transcript starts with single-nucleotide resolution (Figure \ref{fig:six_aat_assay_comparison}).
ChIP-nexus is a high-resolution chromatin immunoprecipitation technique, in which the immunoprecipitated DNA is exonuclease digested up to the bases crosslinked with the protein of interest before sequencing \citep{he2015}.
When applied to the PIC component TFIIB, ChIP-nexus reports where transcription initiation is occurring, thus allowing us to determine if intragenic transcripts in \textit{spt6-1004} result from new transcription initiation.

\section{Data analysis pipelines for TSS-seq and ChIP-nexus}

% In order to use TSS-seq and ChIP-nexus to answer questions about Spt6 and intragenic transcription, I developed analysis pipelines for TSS-seq and ChIP-nexus data.
% The pipelines are written using the Python-based Snakemake workflow specification language \citep{koster2012}, and perform steps including read cleaning \citep{martin2011}, various quality controls \citep{andrews2012}, read alignment \citep{kim2013, langmead2012}, data normalization, coverage track generation \citep{quinlan2010}, peak calling \citep{zhang2008}, differential expression/binding analyses \citep{love2014}, data visualization with clustering, motif enrichment analyses \citep{bailey2015}, and gene ontology analyses \citep{young2010}.
% The Snakemake framework allows these data analyses to be reproducible and scalable from workstations up to computing clusters.
% Updated versions of these pipelines with more details on their capabilities are available at \href{https://github.com/winston-lab}{github.com/winston-lab}.
% In the following subsections I will describe the thought behind only a few of the more novel pipeline steps before moving on to results relating to Spt6 and intragenic transcription.

\subsection{TSS-seq peak calling}

% TSS-seq data from a single region of transcription initiation tends to occur as a cluster of signal distributed over a range of positions, rather than a single nucleotide (Figure \ref{fig:tss_coverage}) \citep{arribere2013, malabat2015}.
% It is reasonable to consider such a cluster of TSS-seq signal as a single entity, because the signals within the cluster are usually highly correlated to one another across different conditions.
% Therefore, to identify TSSs from TSS-seq data and quantify them for downstream analyses such as differential expression, it is necessary to annotate these groups of signal by using the data to perform peak-calling.

% In its current state, the TSS-seq pipeline calls peaks using 1-D \href{https://en.wikipedia.org/wiki/Watershed_(image_processing)}{watershed segmentation}, followed by filtering for reproducibility by the Irreproducible Discovery Rate (IDR) method \citep{li2011}.
% First, a smoothed version of the TSS-seq coverage is generated for each sample using a discretized Gaussian kernel.
% Next, an initial set of peaks is generated by: 1) assigning all nonzero signal in the original, unsmoothed coverage to the nearest local maximum of the smoothed coverage in the direction of positive derivative, and 2) taking the minimum and maximum genomic coordinates of the original coverage assigned to each local maximum as the peak boundaries.
% The peaks are then trimmed to the smallest genomic window that includes 95\% of the original coverage, and the probability of the peak being generated by noise is estimated by a Poisson model where $\lambda$, the expected coverage, is the maximum of the expected coverage over the chromosome and the expected coverage in a window upstream of the peak (as for the ChIP-seq peak caller MACS2 \citep{zhang2008}).
% The influence of local read density on $\lambda$ is intended to reduce false positive peaks within gene bodies, especially for highly expressed genes: Since there are more fragments of RNA present for highly expressed genes, more fragments within the gene body will make it into the final library, even if they are not true 5' ends.
% To generate the final set of peaks, the peaks are ranked by significance under the Poisson model, and filtered by IDR.
% In brief, IDR attempts to separate true peaks from experimental noise based on the intuition that, when peaks in each replicate are independently ranked by a metric such as significance, true peaks will have more similar ranks between replicates than peaks representing noise \citep{li2011}.

% The IDR algorithm currently only works for two replicates.
% Future improvements could include expanding the IDR implementation to handle more replicates and improve the accuracy of peak calling with more data.

\subsection{A note on ChIP-nexus peak calling}

% A number of tools have been created specifically for peak-calling using data from high-resolution ChIP techniques such as ChIP-nexus and ChIP-exo \citep{wang2014, hansen2016}.
% When applied to our TFIIB ChIP-nexus data, these tools tended to split what appeared to be a single TFIIB binding event into multiple peaks.
% This may be because TFIIB has been observed to crosslink to DNA at multiple points (Figure \ref{fig:tfiib_tata}) \citep{rhee2012}, which suggests that while these tools may work well for factors that bind symmetrically with a single crosslinking point on either side, there is still room for improvement when it comes to factors with more complex binding patterns.
% For the purposes of this project, the standard ChIP-seq peak caller MACS2 was used \citep{zhang2008}.

% ChIP-seq peaks lack strand information, as DNA binding factors usually do not bind DNA in a strand-specific manner.
% Because of this, we could not separate intragenic TFIIB peaks into peaks associated with sense or antisense transcription.
% The distinctive shape of the aggregate TFIIB ChIP-nexus signal (Figure \ref{fig:tfiib_tata}) suggests that information about the strand of transcription may be present in the ChIP-nexus binding profile.
% Future work could include learning the direction of transcription from labeled ChIP-nexus training data.

\section{TSS-seq and TFIIB ChIP-nexus results for \textit{spt6-1004}}

To study the relationship between Spt6 and transcription, TSS-seq and TFIIB ChIP-nexus libraries were prepared from wild-type and \textit{spt6-1004} cells, both shifted from 30\textdegree C to 37\textdegree C for 80 minutes.
In wild-type cells, TSS-seq and TFIIB ChIP-nexus recapitulate their expected distributions over the genome: Most TSS signal is restricted to annotated genic TSSs, while most TFIIB signal is localized just upstream of the TSS (Figures \ref{fig:six_tss_seq_heatmaps}, \ref{fig:six_tfiib_heatmap}).
In \textit{spt6-1004}, the signal for both assays infiltrates gene bodies, already suggesting that new transcription initiation does contribute to the intragenic transcription phenotype.
Notably, sense strand TSS-seq signal in \textit{spt6-1004} tends to occur towards the 3$^\prime$ end of genes, while antisense strand TSS-seq signal tends to occur towards the 5$^\prime$ end of genes.

\begin{figure}[H]
\centering
\includegraphics[width=6in]{figures/six/six_tss_seq_heatmaps.pdf}
\caption[Heatmaps of sense and antisense TSS-seq signal from wild-type and \textit{spt6-1004} cells, over non-overlapping coding genes.]{Heatmaps of sense and antisense TSS-seq signal from wild-type and \textit{spt6-1004} cells, over 3522 non-overlapping genes aligned by wild-type genic TSS and sorted by annotated transcript length. Data are shown for each gene up to 300 nucleotides 3$^\prime$ of the cleavage and polyadenylation site (CPS), indicated by the white dotted line. Values are the mean of spike-in normalized coverage in non-overlapping 20 nucleotide bins, averaged over two replicates. Values above the 92nd percentile are set to the 92nd percentile for visualization.}
\label{fig:six_tss_seq_heatmaps}
\end{figure}

\begin{wrapfigure}[19]{R}{3in}
    \centering
    \includegraphics[width=3in]{figures/six/six_tfiib_heatmap.pdf}
    \caption[Heatmaps of TFIIB ChIP-nexus protection from wild-type and \textit{spt6-1004} cells, over non-overlapping coding genes]{Heatmaps of TFIIB binding measured by ChIP-nexus, over the same regions shown in Figure \ref{fig:six_tss_seq_heatmaps}. Values are the mean of library-size normalized coverage in non-overlapping 20 bp bins, averaged over two replicates. Values above the 85th percentile are set to the 85th percentile for visualization.}
    \label{fig:six_tfiib_heatmap}
\end{wrapfigure}

The TSS-seq data were quantified by peak calling and differential expression analysis, and classified into genomic categories based on their position relative to coding genes.
As suggested by the heatmap visualization (Figure \ref{fig:six_tss_seq_heatmaps}), we detect significant induction of over 4000 intragenic and antisense TSSs in \textit{spt6-1004} (Figure \ref{fig:six_tss_diffexp_summary}).
Compared to previous studies identifying \textit{spt6-1004} intragenic transcription by tiled microarray and RNA-seq, we identify intragenic transcription at over 1000 additional genes (Figure \ref{fig:six_intragenic_genes_bvenn}) and have the exact start sites of all identified TSSs.
The TSS-seq data also revealed an unexpected downregulation of most genic TSSs: In this experiment, we detected a significant downregulation to levels below 67\% of wild-type levels at 75\% (3579/4792) of genic TSSs (Figure \ref{fig:six_tss_diffexp_summary}).
As a result of intragenic/antisense induction and genic repression, expression levels in \textit{spt6-1004} of all classes of transcripts become similar to one another (Figure \ref{fig:six_tss_expression_levels}).

\begin{figure}[H]
    \centering
    \begin{minipage}[t]{2.875in}
        \centering
        \includegraphics[width=2.875in]{figures/six/six_tss_diffexp_summary.pdf}
        \caption[Bar plot of the number of TSS-seq peaks of various genomic classes differentially expressed in \textit{spt6-1004} versus wild-type.]{Bar plots of the number of TSS-seq peaks differentially expressed in \textit{spt6-1004} after 80 minutes at 37\textdegree C versus wild-type after 80 minutes at 37\textdegree C. The height of each bar is proportional to the total number of peaks in the category, including those not found to be significantly differentially expressed.}
        \label{fig:six_tss_diffexp_summary}
    \end{minipage}\hfill
    \begin{minipage}[t]{2.875in}
        \centering
        \includegraphics[width=2.875in]{figures/six/six_intragenic_genes_bvenn.pdf}
        \caption[Set diagram of the number of genes with \textit{spt6-1004}-induced intragenic transcripts reported in \citet{cheung2008}, \citet{uwimana2017}, and our TSS-seq data.]{Set diagram of the number of genes reported to have \textit{spt6-1004}-induced intragenic transcripts using tiled arrays \citet{cheung2008}, RNA-seq \citet{uwimana2017}, and TSS-seq (this work).}
        \label{fig:six_intragenic_genes_bvenn}
    \end{minipage}
\end{figure}

\begin{wrapfigure}[10]{r}{3in}
    \centering
    \includegraphics[width=3in]{figures/six/six_tss_expression_levels.pdf}
    \caption[Violin plots of expression level distributions for genomic classes of TSS-seq peaks in wild-type and \textit{spt6-1004} cells.]{Violin plots of expression level distributions for genomic classes of TSS-seq peaks in wild-type and \textit{spt6-1004}, both after 80 minutes at 37\textdegree C. Normalized counts are the mean of spike-in size factor normalized counts from two replicates.}
    \label{fig:six_tss_expression_levels}
\end{wrapfigure}

The changes in transcript levels in \textit{spt6-1004} observed by TSS-seq correspond with substantial differences in the pattern of TFIIB binding on the genome.
In contrast to the discrete peaks in promoter regions seen in wild-type, TFIIB in \textit{spt6-1004} binds much more promiscuously, with many loci having TFIIB signal spread over broad regions of the genome (Figure \ref{fig:six_tfiib_spreading_ssa4}).
This difference in binding pattern makes peak calling ineffective for quantifying TFIIB signal in this case: ChIP-seq peak callers generally use different algorithms for calling `narrow' peaks (e.g. for sequence-specific transcription factors) and `broad' peaks (e.g. for histone modifications), meaning that a single algorithm is unable to call peaks that are meaningful for differential binding analyses between wild-type and \textit{spt6-1004}.
Therefore, to see if changes in transcript levels in \textit{spt6-1004} correspond to changes in transcription initiation, we compared the change in TSS-seq signal at TSS-seq peaks in \textit{spt6-1004} to the change in TFIIB ChIP-nexus signal in the window extending 200 bp upstream of the TSS-seq peak.
Changes in TSS-seq signal in \textit{spt6-1004} are associated with a change in TFIIB signal of the same sign at over 82\% of TSSs of any genomic class, indicating that the increase in intragenic transcript levels and decrease in genic transcript levels observed in \textit{spt6-1004} are in large part explained by changes in transcription initiation.

\begin{SCfigure}[50][h]
\centering
\includegraphics[width=3.5in]{figures/six/six_tfiib_spreading_ssa4.pdf}
\caption[TFIIB ChIP-nexus protection over the 20 kb flanking the gene \textit{SSA4}, in wild-type and \textit{spt6-1004} cells.]{
    \begin{description}[align=right, nosep, itemindent=0pt, leftmargin=4.2em, font=\normalfont]
        \item [top)] TFIIB ChIP-nexus protection in wild-type and \textit{spt6-1004}, over 20 kb of chromosome II flanking the \textit{SSA4} gene.
        \item [bottom)] Expanded view of TFIIB protection over the \textit{SSA4} gene.
    \end{description}
}
\label{fig:six_tfiib_spreading_ssa4}
\end{SCfigure}

\begin{figure}[h]
\centering
\includegraphics[width=6in]{figures/six/six_tss_v_tfiib.pdf}
\caption[Scatterplots of fold-change in \textit{spt6-1004} over wild-type, comparing TSS-seq and TFIIB ChIP-nexus.]{Scatterplots of fold-change in \textit{spt6-1004} over wild-type, comparing TSS-seq and TFIIB ChIP-nexus. Each dot represents a TSS-seq peak paired with the window extending 200 bp upstream of the TSS-seq peak summit for quantification of TFIIB ChIP-nexus signal. Fold-changes are regularized fold-change estimates from DESeq2, with size factors determined from the \textit{S. pombe} spike-in (TSS-seq), or \textit{S. cerevisiae} counts (ChIP-nexus).}
\end{figure}

\section{MNase-seq results from \textit{spt6-1004}}

Because a primary function of Spt6 is to act as histone chaperone that reassembles nucleosomes in the wake of transcription \citep{duina2011}, it is reasonable to expect that the transcriptional changes seen in \textit{spt6-1004} would be associated with changes in chromatin structure.
The requirement for Spt6 in maintaining normal chromatin structure has been demonstrated in previous studies \citep{bortvin1996, ivanovska2011, jeronimo2015, kaplan2003, perales2013, vanbakel2013}.
To re-examine this requirement in higher resolution, we assayed nucleosome protection genome-wide using micrococcal nuclease digestion of chromatin followed by sequencing (MNase-seq).

\begin{figure}[H]
    \centering
    \begin{minipage}[t]{2.875in}
        \centering
        \includegraphics[width=2.875in]{figures/six/six_mnase_metagene.pdf}
        \caption[Average MNase-seq dyad signal in wild-type and \textit{spt6-1004}, over non-overlapping genes aligned by wild-type +1 nucleosome dyad.]{Average MNase-seq dyad signal in wild-type and \textit{spt6-1004}, over 3522 non-overlapping genes aligned by wild-type +1 nucleosome dyad. Values are the mean of spike-in normalized coverage in non-overlapping 20 bp bins, averaged over two replicates (\textit{spt6-1004}) or one experiment (wild-type). The solid line and shading are the median and the inter-quartile range.}
        \label{fig:six_mnase_metagene}
    \end{minipage}\hfill
    \begin{minipage}[t]{2.875in}
        \centering
        \includegraphics[width=2.875in]{figures/six/six_global_nuc_occ_fuzz.pdf}
        \caption[Contour plot of nucleosome occupancy and fuzziness in wild-type and \textit{spt6-1004}.]{Contour plot of the global distributions of nucleosome occupancy and fuzziness in wild-type and \textit{spt6-1004}. Dashed lines indicate median values.}
        \label{fig:six_global_nuc_occ_fuzz}
    \end{minipage}
\end{figure}

In wild-type, the MNase-seq data recapitulate the expected signature over genes, with a nucleosome-depleted region upstream of a strongly positioned `+1' nucleosome, and a regularly phased array of nucleosomes over the gene body (Figure \ref{fig:six_mnase_metagene}).
In \textit{spt6-1004}, nucleosome signal is severely reduced at canonical nucleosome positions and spreads into inter-nucleosome regions.
Changes in aggregate nucleosome signal such as those observed in Figure \ref{fig:six_mnase_metagene} are the combination of changes to nucleosome occupancy (the number of reads assigned to a nucleosome), fuzziness (the standard deviation of read positions for a nucleosome), and position (the coordinate with the maximum reads for a nucleosome) \citep{chen2013}.
Using DANPOS2 \citep{chen2013}, we called nucleosome positions and quantified these metrics for wild-type and \textit{spt6-1004}.
Wild-type nucleosomes span a relatively wide range of occupancy and fuzziness space, with highly occupied nucleosomes tending to be less fuzzy (i.e., more well-positioned) (Figure \ref{fig:six_global_nuc_occ_fuzz}).
In \textit{spt6-1004}, the population of nucleosomes is much more homogeneous: nucleosome occupancy is decreased globally, and nucleosome fuzziness is restricted to the high end of the wild-type distribution.

Previous studies observed two trends: 1) In wild-type cells, nucleosome positioning is weaker over highly transcribed genes than over moderately transcribed genes \citep{shivaswamy2008}, and 2) In \textit{spt6-1004} cells, the decrease in nucleosome occupancy is greater for highly transcribed genes \citep{ivanovska2011}.
To re-examine these trends, we looked at the MNase-seq data in the context of NET-seq data, which reports the position of actively transcribing RNAPII and reflects a gene's level of transcription (Figure \ref{fig:six_mnase_heatmaps}) \citep{churchman2012}.
The data support the first trend: in wild-type, genes with the strongest NET-seq signal have decreased MNase-seq signal.
However, there is no obvious relationship between transcription level and the nucleosome changes observed in \textit{spt6-1004} (Figure \ref{fig:six_mnase_heatmaps}).
The apparent discrepancy might be explained by the improved resolution and breadth of MNase-seq versus the MNase and microarray of chromosome III used in the previous study \citep{ivanovska2011}.

\begin{figure}[H]
    \centering
    \includegraphics[width=6in]{figures/six/six_mnase_heatmaps.pdf}
    \caption[Heatmaps of sense NET-seq signal, MNase-seq dyad signal, nucleosome occupancy changes, and nucleosome fuzziness changes over non-overlapping coding genes, aligned by genic TSS and arranged by sense NET-seq signal.]{
        \begin{description}[align=right, nosep, itemindent=0pt, leftmargin=4.2em, font=\normalfont]
            \item [left)] Heatmap of sense strand NET-seq signal for 3522 non-overlapping genes, aligned by genic TSS and sorted by total sense strand NET-seq signal in the window extending 500 nucleotides downstream from the genic TSS. Values are the mean of library-size normalized coverage in non-overlapping 20 nt bins, averaged over two replicates.
            \item [middle)] Heatmaps of MNase-seq dyad signal in wild-type and \textit{spt6-1004} for the same genes, aligned by wild-type +1 nucleosome dyad and arranged by sense NET-seq signal as in the leftmost panel. Values are the mean of spike-in normalized coverage in non-overlapping 20 bp bins, averaged over two replicates (\textit{spt6-1004}) or one experiment (wild-type).
            \item [right)] Heatmaps of fold-change in nucleosome occupancy and fuzziness for the same genes, aligned by wild-type +1 nucleosome dyad and arranged by sense NET-seq signal as in the leftmost panel.
        \end{description}
    }
    \label{fig:six_mnase_heatmaps}
\end{figure}

\subsection{Clustering of MNase-seq profiles at \textit{spt6-1004}-induced intragenic TSSs}

\begin{sidewaysfigure}
    \centering
    \includegraphics[width=8.25in]{figures/six/six_mnase_som.pdf}
    \caption[Average MNase-seq dyad signal around all \textit{spt6-1004}-induced intragenic TSSs, grouped by a self-organizing map of the MNase-seq signal.]{Average MNase-seq dyad signal around all \textit{spt6-1004}-induced intragenic TSSs, grouped by assignment to nodes of a 6x8 super-organizing map (SOM). The number of TSSs assigned to each node is shown in the upper right of each panel, and is shaded by the node's assignment to a cluster determined by agglomerative hierarchical clustering of the nodes. The solid line and shading are the median and inter-quartile range of the mean spike-in normalized coverage over two replicates (\textit{spt6-1004}) or one experiment (wild-type), in non-overlapping 5 bp bins.}
    \label{fig:six_mnase_som}
\end{sidewaysfigure}

\begin{figure}[h]
\centering
\includegraphics[width=6in]{figures/six/six_intragenic_mnase_metagenes.pdf}
\caption[Average wild-type and \textit{spt6-1004} MNase-seq dyad signal and GC content for three clusters of \textit{spt6-1004}-induced intragenic TSSs, as well as wild-type genic TSSs.]{
    \begin{description}[align=right, nosep, itemindent=0pt, leftmargin=6.2em, font=\normalfont]
        \item [top row)] Average MNase-seq dyad signal for \textit{spt6-1004} intragenic TSSs, both aggregated and grouped into three clusters by the wild-type and \textit{spt6-1004} MNase-seq dyad signal flanking the TSS, as well as all genic TSSs detected in wild-type. Values are the mean of spike-in normalized dyad coverage in non-overlapping 10 bp bins, averaged over two replicates (\textit{spt6-1004}) or one experiment (wild-type). The solid line and shading are the median and inter-quartile range.
        \item [bottom row)] Average GC content of the DNA sequence, as above.
    \end{description}
}
\label{fig:six_intragenic_mnase_metagenes}
\end{figure}

\lipsum[1]

\section{Other features of \textit{spt6-1004} intragenic promoters}

\subsection{Information content of intragenic TSSs}

\begin{wrapfigure}[12]{R}{3in}
\centering
\includegraphics[width=3in]{figures/six/six_tss_seqlogos.pdf}
\caption[Sequence logos of TSS-seq reads overlapping genic and intragenic TSS-seq peaks in \textit{spt6-1004}.]{Sequence logos of the information content of TSS-seq reads overlapping genic and intragenic TSS-seq peaks in \textit{spt6-1004}.}
\label{fig:six_tss_seqlogos}
\end{wrapfigure}

\lipsum[1]

\subsection{Sequence motifs enriched at intragenic TSSs}

\begin{wrapfigure}[15]{r}{3in}
\centering
\includegraphics[width=3in]{figures/six/six_intragenic_tata.pdf}
\caption[Kernel density estimate of matches to a consensus TATA-box motif upstream of genic and \textit{spt6-1004}-induced intragenic TSSs.]{Scaled density of occurrences of exact matches to the motif TATAWAWR upstream of TSSs. For each category, a Gaussian kernel density estimate of the positions of motif occurrences is multiplied by the number of motif occurrences in the genomic category and divided by the number of regions in the category.}
\label{fig:six_intragenic_tata}
\end{wrapfigure}

\lipsum[1]

\begin{figure}[]
% \centering
% \includegraphics[width=6in]{figures/six/six_intragenic_mnase_metagenes.pdf}
    \caption[Sequence logos of motifs enriched upstream of \textit{spt6-1004}-induced intragenic and antisense TSSs.]{}
% \label{fig:six_intragenic_mnase_metagenes}
\end{figure}

\section{Summary}

\newpage
\bibliographystyle{apalike}
\begingroup
\singlespacing
\bibliography{references/spt6}
\endgroup

\cleardoublepage

\chapter{Genomics of transcription elongation factor Spt5}
\label{chapter:five}

\section{Collaborators}

\begin{description}[align=right, leftmargin=!, labelwidth=5cm, noitemsep]
    \item [Ameet Shetty] generated TSS-seq, MNase-seq, NET-seq,\\RNA-seq, and ChIP-seq libraries
\end{description}

\section{Introduction to Spt5 and prior work}

Relevant information about Spt5 is summarized as follows \citep{shetty2017}:

\begin{itemize}[nosep, topsep=.5em]
    \item Spt5 is the only transcription factor known to be conserved in all three domains of life \citep{hartzog2013, werner2012}.
    \item Spt5 co-localizes with elongating RNA Pol II \citep{mayer2010, rahl2010}.
    \item Spt5 binds over the Pol II clamp domain, likely stabilizing the elongation complex \citep{hirtreiter2010, klein2011, martinez-rucobo2011}.
    \item Spt5 physically recruits factors to the elongating transcription complex, in a manner dependent on the modification status of its C-terminal region (CTR) \citep{hartzog2013}:
    \begin{itemize}[nosep]
        \item in its unphosphorylated state, the CTR aids in recruiting the mRNA capping enzyme \citep{doamekpor2014, doamekpor2015, schneider2010, wen1999}
        \item in its phosphorylated state, the CTR recruits the Paf1 complex, which is important for Pol II elongation \citep{liu2009, mbogning2013, wier2013, zhou2009}
        \item Spt5 helps to recruit mRNA 3' end processing factors \citep{mayer2012, stadelmayer2014, yamamoto2014}.
        \item Spt5 helps to recruit the Rpd3S histone deacetylase complex \citep{drouin2010}.
    \end{itemize}
\end{itemize}

\begin{wrapfigure}[11]{r}{3in}
    \begin{tikzpicture}[x=3in, y=1in]
        \draw [line width=2pt] (0,0.1) -- (0.95,0.1);
        \draw [->, line width=3pt] (0.1,0.1) -- (0.1,0.25) -- (0.25,0.25);
        \draw [line width=2pt] (0.175,0.6) -- (0.175, 0.32);
        \draw [line width=2pt] (0.15, 0.32) -- (0.20, 0.32);
        \node at (0.175, 0.7) {\normalsize thiamine};
        \draw [fill=lightpurple, line width=1pt] (0.28, 0) rectangle (0.8, 0.2);
        \node at (0.54, 0.1) {\normalsize spt5};
        \draw [->, line width=2pt] (0.33, 0.3) to [out=90, in=180] (0.5,0.7);
        \draw [fill=lightpurple, line width=1pt] (0.62, 0.7) circle [x radius=0.1, y radius=0.15];
        \node at (0.62, 0.7) {\normalsize Spt5};
        \draw [->, line width=2pt] (0.73, 0.7) to (0.85, 0.7);
        \node at (0.77, 1) {\normalsize auxin};
        \draw [, line width=2pt] (0.77, 0.9) to [out=270, in=180] (0.80,0.7);
        \node at (0.9, 0.7) {\scalebox{2}{$\varnothing$}};
    \end{tikzpicture}
    \caption[Diagram of the dual-shutoff system used to deplete Spt5 from \textit{S. pombe}]{Diagram of the dual-shutoff system used to deplete Spt5 from \textit{S. pombe}. Spt5 is expressed from a thiamine-repressible promoter, and tagged with an auxin-inducible degron tag for specific degradation upon addition of auxin.}
    \label{fig:five_depletion_diagram}
\end{wrapfigure}

\lipsum

\begin{wrapfigure}[14]{r}{3in}
    \includegraphics[width=3in]{figures/five/five_summary_metagenes.pdf}
    \caption[Average Spt5 ChIP-seq, RNAPII ChIP-seq, and sense NET-seq signal over non-overlapping coding genes, from Spt5 depleted and non-depleted cells.]{Average Spt5 ChIP-seq, RNAPII ChIP-seq, and sense NET-seq signal in Spt5 non-depleted and depleted cells, over 1989 non-overlapping coding genes scaled from TSS to CPS. The solid line and shading are the median and inter-quartile range of the mean spike-in normalized coverage over two replicates, in non-overlapping 20 bp bins.}
    \label{fig:five_suummary_metagenes}
\end{wrapfigure}

\lipsum[2]

\begin{wrapfigure}[6]{r}{4.5in}
    \includegraphics[width=4.5in]{figures/five/five_rnapii_phosphomark_enrichment.pdf}
    \caption[Enrichment of RNAPII phospho-serine 5 and phospho-serine 2 over non-overlapping coding genes, in Spt5 depleted and non-depleted cells.]{Caption wsdasdr zzzz.}
    \label{fig:five_rnapii_phosphomark_enrichment}
\end{wrapfigure}

\begin{figure}
    \includegraphics[width=3in]{figures/five/five_rnaseq_heatmaps.pdf}
    \caption[Heatmaps of antisense RNA-seq signal from Spt5 depleted and non-depleted cells, over non-overlapping coding genes.]{Caption wsdasdr zzzz.}
    \label{fig:five_rnaseq_heatmaps}
\end{figure}

\lipsum[2]

\section{TSS-seq results from Spt5 depletion}

\begin{wrapfigure}[6]{r}{3in}
    \includegraphics[width=3in]{figures/five/five_tss_diffexp_summary.pdf}
    \caption[Bar plot of the number of TSS-seq peaks of various genomic classes differentially expressed in Spt5 depleted versus non-depleted cells.]{Caption wsdasdr zzzz.}
    \label{fig:five_tss_diffexp_summary}
\end{wrapfigure}

\lipsum[1]

\begin{figure}
% \includegraphics[width=6in]{figures/stress/stress_promoter_tss_polyenrichment.pdf}
\caption[Heatmaps of antisense TSS-seq, RNA-seq, and NET-seq signal from Spt5 depleted and non-depleted cells, over genes with Spt5-depletion-induced antisense TSSs.]{Caption wsdasdr zzzz.}
% \label{fig:stress_promoter_tss_polyenrichment}
\end{figure}

\section{MNase-seq results from Spt5 depletion}

\begin{wrapfigure}[10]{r}{3in}
    \includegraphics[width=3in]{figures/five/five_mnase_metagene.pdf}
    \caption[Average MNase-seq dyad signal from Spt5 depleted and non-depleted cells, over non-overlapping coding genes.]{Caption wsdasdr zzzz.}
    \label{fig:five_mnase_metagene}
\end{wrapfigure}

\lipsum[1]

\subsection{MNase-seq profile at Spt5-depletion-induced antisense TSSs}

\begin{figure}
% \includegraphics[width=6in]{figures/stress/stress_promoter_tss_polyenrichment.pdf}
\caption[A figure showing MNase-seq signal around Spt5-depletion-induced antisense TSSs.]{Caption wsdasdr zzzz.}
% \label{fig:stress_promoter_tss_polyenrichment}
\end{figure}

\section{Sequence motifs enriched at antisense TSSs}

\begin{figure}
% \includegraphics[width=6in]{figures/stress/stress_promoter_tss_polyenrichment.pdf}
\caption[A figure showing motifs enriched upstream of Spt5-depletion-induced antisense TSSs.]{Caption wsdasdr zzzz.}
% \label{fig:stress_promoter_tss_polyenrichment}
\end{figure}

\section{Discussion}

\section{Methods}

\subsection{A note on spike-in normalization for ChIP-seq experiments with input samples}

In the course of determining how to do spike-in normalization for ChIP-seq libraries, I discovered the following error in a published spike-in normalization method.
Throughout the following explanation, I use `experimental' and `spike-in' to refer to the two genomes present in the experiment, e.g., experimental signal and spike-in signal.

The goal when including spike-ins in a ChIP-seq experiment is to be able to normalize the experimental signal, such that the normalized signal is proportional to the absolute abundance of the factor being immunoprecipitated.
A straightforward method to accomplish this normalization is to linearly scale the experimental signal of a library by a normalization factor, which we will call $\alpha$.
To calculate $\alpha$ for each library, we can use the fact that a normalized `spike-in signal' should be the same for all libraries, since the biological state of the spike-in cells is the same in all libraries.
The key to correctly determining $\alpha$ is defining exactly what this spike-in signal is.

The measurement we begin with for determination of the spike-in signal of a library is the number of reads in the library which map uniquely to the spike-in genome ($R_{\text{spike}}$).
This value will vary based on two factors: the sequencing depth of the library, and the proportion of cells which were spike-in cells ($\phi$):
\begin{align*}
    R_{\text{spike}} &\equiv \text{the number of reads in the library mapping uniquely to the spike-in genome}; \\
    \phi &\equiv \text{the proportion of spike-in cells in the sample}.
\end{align*}
However, the derivation of $\alpha$ is more easily understood in terms of absolute cell numbers rather than $\phi$:
\begin{align*}
    C_{\text{exp}} &\equiv \text{the number of experimental cells used to prepare a library}; \\
    C_{\text{spike}} &\equiv \text{the number of spike-in cells used to prepare a library}. \\
\end{align*}
We can express the \textbf{number of spike-in reads per spike-in cell} by simply taking the fraction $\frac{R_{\text{spike}}}{C_{\text{spike}}}$.
We know that the biological state of a spike-in cell is the same regardless of which sample it belongs to, so we \textit{could} set $\frac{R_{\text{spike}}}{C_{\text{spike}}}$ equal to all samples in order to calculate $\alpha$.
However, this would not account for differences in $\phi$ between samples: Two libraries representing the same condition and sequenced to the same depth should have equivalent values of $\frac{R_{\text{spike}}}{C_{\text{spike}}}$, which does not hold true if they differed in the proportion of spike-in added.

The metric for `spike-in signal' that leads to the correct expression for $\alpha$ is the \textbf{number of spike-in reads per spike-in cell \textit{per experimental cell}}:
\begin{align*}
    & \frac{ \frac{R_\text{spike}}{C_\text{spike}}}{C_\text{exp}} \\
    = & \frac{R_\text{spike} C_\text{exp}}{C_\text{spike}}.
\end{align*}
From here, it's simple to calculate $\alpha$ by setting this value to be equal for all samples.
Since the actual value of the spike-in signal doesn't matter as long as it is equal for all libraries, we can arbitrarily set it to $1$ for convenience:
\begin{align*}
    \alpha \frac{R_\text{spike} C_\text{exp}}{C_\text{spike}} &= 1 \\
    \alpha &= \frac{C_\text{spike}}{R_\text{spike} C_\text{exp}}.
\end{align*}
Notice that only the ratio of spike-in to experimental cells is needed to calculate $\alpha$, and not the absolute number of spike-in and experimental cells.
We can rewrite this expression in terms of $\phi$, the proportion of the sample that was spike-in cells:
\begin{align*}
    \phi &= \frac{C_\text{spike}}{C_\text{spike} + C_\text{exp}} \\
    C_\text{spike} & = \phi \left(C_\text{spike} + C_\text{exp} \right) \\
    C_\text{spike} \left(1-\phi \right) & = \phi C_\text{exp} \\
    \frac{C_\text{spike}}{C_\text{exp}} & = \frac{\phi}{1-\phi} & \alpha &= \frac{C_\text{spike}}{R_\text{spike} C_\text{exp}} \\
                                        && \alpha &= \frac{\phi}{R_\text{spike} \left(1-\phi \right)}.
\end{align*}
This form for $\alpha$ differs from the one presented in \citet{orlando} with no derivation:
\begin{align*}
    \alpha &= \frac{\phi}{R_\text{spike} \left(1-\phi \right)} & \alpha_\text{orlando} &= \frac{\phi}{R_\text{spike}}.
\end{align*}
Working through a few examples with both versions of $\alpha$ will reveal that $\alpha_\text{orlando}$ leads to incorrect normalization when $\phi$ is not equivalent for all samples.

In the first example, we will vary sequencing depth between two libraries, keeping everything else constant.
Consider a single ChIP library prep in which 20\% of the cells were spike-in cells (i.e., $\phi=0.2$).
The library is then unevenly split into two aliquots and sequenced.
One library has four times the reads of the other library.
\begin{align*}
    R_{\text{spike}_1} &= 1 & R_{\text{spike}_2} &= 4 \\
    R_{\text{exp}_1} &= 4 & R_{\text{exp}_2} &= 16 \\
\end{align*}
\begin{align*}
    \alpha_1 &= \frac{\phi}{R_{\text{spike}_1} \left(1-\phi \right)} &
    \alpha_2 &= \frac{\phi}{R_{\text{spike}_2} \left(1-\phi \right)} &
    \alpha_{\text{orlando}_1} &= \frac{\phi}{R_{\text{spike}_1}} &
    \alpha_{\text{orlando}_2} &= \frac{\phi}{R_{\text{spike}_2}} \\
    \alpha_1 &= \frac{0.2}{1 \left(0.8 \right)} &
    \alpha_2 &= \frac{0.2}{4 \left(0.8 \right)} &
    \alpha_{\text{orlando}_1} &= \frac{0.2}{1} &
    \alpha_{\text{orlando}_2} &= \frac{0.2}{4} \\
    \alpha_1 &= \frac{4}{16} &
    \alpha_2 &= \frac{1}{16} &
    \alpha_{\text{orlando}_1} &= \frac{4}{20} &
    \alpha_{\text{orlando}_2} &= \frac{1}{20}.
\end{align*}
The total levels of spike-in normalized experimental signal can be found for each library by multiplying $\alpha$ by $R_\text{exp}$, for our version of $\alpha$,
\begin{align*}
    \text{signal}_1 &= \alpha_1 R_{\text{exp}_1}  &
    \text{signal}_2 &= \alpha_2 R_{\text{exp}_2}  \\
    \text{signal}_1 &=  \frac{4}{16} \left(4 \right)  &
    \text{signal}_2 &=  \frac{1}{16} \left(16 \right)  \\
    \text{signal}_1 &=  1 &
    \text{signal}_2 &=  1 \\
\end{align*}
and for $\alpha_\text{orlando}$:
\begin{align*}
    \text{signal}_{\text{orlando}_1} &= \alpha_{\text{orlando}_1} R_{\text{exp}_1} &
    \text{signal}_{\text{orlando}_2} &= \alpha_{\text{orlando}_2} R_{\text{exp}_2} \\
    \text{signal}_{\text{orlando}_1} &= \frac{4}{20} \left(4\right) &
    \text{signal}_{\text{orlando}_2} &= \frac{1}{20} \left(16\right) \\
    \text{signal}_{\text{orlando}_1} &= 0.8 &
    \text{signal}_{\text{orlando}_2} &= 0.8 \\
\end{align*}
Only the relative abundances within normalization methods matter, so in this case both calculations correctly normalized for library size and say that the normalized signal in the two libraries are the same.

Now let's consider two libraries from two different conditions with $\phi=0.1$.
In condition 2, there is a known global decrease in experimental signal expected.
This time, we will skip the algebra:
\begin{align*}
    R_{\text{spike}_1} &= 1 & R_{\text{spike}_2} &= 4 \\
    R_{\text{exp}_1} &= 9 & R_{\text{exp}_2} &= 6 \\
\end{align*}
\begin{align*}
    % \alpha_1 &= \frac{\phi}{R_{\text{spike}_1} \left(1-\phi \right)} &
    % \alpha_2 &= \frac{\phi}{R_{\text{spike}_2} \left(1-\phi \right)} &
    % \alpha_{\text{orlando}_1} &= \frac{\phi}{R_{\text{spike}_1}} &
    % \alpha_{\text{orlando}_2} &= \frac{\phi}{R_{\text{spike}_2}} \\
    % \alpha_1 &= \frac{0.1}{1 \left(0.9 \right)} &
    % \alpha_2 &= \frac{0.1}{4 \left(0.9 \right)} &
    % \alpha_{\text{orlando}_1} &= \frac{0.1}{1} &
    % \alpha_{\text{orlando}_2} &= \frac{0.1}{4} \\
    \alpha_1 &= \frac{4}{36} &
    \alpha_2 &= \frac{1}{36} &
    \alpha_{\text{orlando}_1} &= \frac{4}{40} &
    \alpha_{\text{orlando}_2} &= \frac{1}{40}
\end{align*}
\begin{align*}
    % \text{signal}_1 &= \alpha_1 R_{\text{exp}_1}  &
    % \text{signal}_2 &= \alpha_2 R_{\text{exp}_2}  &
    % \text{signal}_{\text{orlando}_1} &= \alpha_{\text{orlando}_1} R_{\text{exp}_1} &
    % \text{signal}_{\text{orlando}_2} &= \alpha_{\text{orlando}_2} R_{\text{exp}_2} & \\
    % \text{signal}_1 &=  \frac{4}{36} \left(9 \right)  &
    % \text{signal}_2 &=  \frac{1}{36} \left(6 \right)  &
    % \text{signal}_{\text{orlando}_1} &= \frac{4}{40} \left(9\right) &
    % \text{signal}_{\text{orlando}_1} &= \frac{1}{40} \left(6\right) & \\
    \text{signal}_1 &=  1 &
    \text{signal}_2 &=  1/6 &
    \text{signal}_{\text{orlando}_1} &= 0.9 &
    \text{signal}_{\text{orlando}_2} &= 0.15 &
\end{align*}

Both methods correctly detect that experimental signal levels in library 2 are 1/6th that of library 1.

Finally, let's consider two libraries from the same condition which were spiked in with different amounts of spike-in cells. Both libraries are sequenced to the same depth. Since the libraries are from the same condition, we expect their total experimental signal to be the same after normalization, even though they had different amounts of spike-in added.
\begin{align*}
    \phi_1 &= 0.2 & \phi_2 &=0.4 \\
    R_{\text{spike}_1} &= 2 & R_{\text{spike}_2} &= 4 \\
    R_{\text{exp}_1} &= 8 & R_{\text{exp}_2} &= 6 \\
\end{align*}
\begin{align*}
    \alpha_1 &= \frac{\phi_1}{R_{\text{spike}_1} \left(1-\phi_1 \right)} &
    \alpha_2 &= \frac{\phi_2}{R_{\text{spike}_2} \left(1-\phi_2 \right)} &
    \alpha_{\text{orlando}_1} &= \frac{\phi_1}{R_{\text{spike}_1}} &
    \alpha_{\text{orlando}_2} &= \frac{\phi_2}{R_{\text{spike}_2}} \\
    \alpha_1 &= \frac{0.2}{2 \left(0.8 \right)} &
    \alpha_2 &= \frac{0.4}{4 \left(0.6 \right)} &
    \alpha_{\text{orlando}_1} &= \frac{0.2}{2} &
    \alpha_{\text{orlando}_2} &= \frac{0.4}{4} \\
    \alpha_1 &= \frac{3}{24} &
    \alpha_2 &= \frac{4}{24} &
    \alpha_{\text{orlando}_1} &= \frac{1}{10} &
    \alpha_{\text{orlando}_2} &= \frac{1}{10}
\end{align*}
\begin{align*}
    \text{signal}_1 &= \alpha_1 R_{\text{exp}_1}  &
    \text{signal}_2 &= \alpha_2 R_{\text{exp}_2}  \\
    \text{signal}_1 &=  \frac{3}{24} \left(8 \right)  &
    \text{signal}_2 &=  \frac{4}{24} \left(6 \right)  \\
    \text{signal}_1 &=  1 &
    \text{signal}_2 &=  1 \\
\end{align*}
\begin{align*}
    \text{signal}_{\text{orlando}_1} &= \alpha_{\text{orlando}_1} R_{\text{exp}_1} &
    \text{signal}_{\text{orlando}_2} &= \alpha_{\text{orlando}_2} R_{\text{exp}_2} & \\
    \text{signal}_{\text{orlando}_1} &= \frac{1}{10} \left(8\right) &
    \text{signal}_{\text{orlando}_2} &= \frac{1}{10} \left(6\right) & \\
    \text{signal}_{\text{orlando}_1} &= 0.8 &
    \text{signal}_{\text{orlando}_2} &= 0.6 \\
\end{align*}
Here, our method correctly normalizes the two samples to the same total experimental signal while using the Orlando $\alpha$ results in an apparent decrease in signal in library 2.
This is because the Orlando $\alpha$ fails to account for the fact that when you add more spike-in to a sample, you necessarily decrease the proportion of the sample that is experimental.
In most experiments with spike-ins, this isn't really a problem because we assume that $\phi$ is the same for all samples.
However, with ChIP-seq experiments that include input samples, if we assume that the experimental and spike-in input sample read counts are proportional to the amounts of experimental and spike-in cells mixed, we can plug these values in for values of $\phi$ to get a more reliable estimation of experimental signal levels.
In this case, it becomes important to use the correct equation for $\alpha$.

So, putting everything together, here's how I use the spike-in to normalize an IP ChIP-seq library paired with an input ChIP-seq library.

As stated above, we assume that the experimental and spike-in read counts in the input sample are proportional to the numbers of experimental and spike-in cells used to prepare the library:
\begin{align*}
    R_{\text{input}_\text{exp}} \propto C_\text{exp}, \\
    R_{\text{input}_\text{spike}} \propto C_\text{spike}
\end{align*}
Therefore, we can plug these values in for $C$ for both the input and IP libraries (using the form of $\alpha$ without $\phi$):
\begin{align*}
    \alpha_\text{input} &= \frac{C_{\text{input}_\text{spike}}}{R_{\text{input}_\text{spike}} C_{\text{input}_\text{exp}}} &
    \alpha_\text{IP} &= \frac{C_{\text{input}_\text{spike}}}{R_{\text{IP}_\text{spike}} C_{\text{input}_\text{exp}}} \\
    \alpha_\text{input} &\propto \frac{R_{\text{input}_\text{spike}}}{R_{\text{input}_\text{spike}} R_{\text{input}_\text{exp}}} &
    \alpha_\text{IP} &\propto \frac{R_{\text{input}_\text{spike}}}{R_{\text{IP}_\text{spike}} R_{\text{input}_\text{exp}}} \\
    \alpha_\text{input} &\propto \frac{1}{R_{\text{input}_\text{exp}}} &
\end{align*}
Notice how $\alpha_\text{input}$ reduces down to normalizing by the experimental library size, with no dependence at all on the spike-in.
This makes sense because the input always represents the same state, regardless of how much spike-in is added to it.
The function of the spike-in in the input is only to allow us to estimate abundances in the corresponding IP library.
Rewriting $\alpha_\text{IP}$ in the form 
\begin{align*}
    \alpha_\text{IP} &\propto \frac{1}{R_{\text{IP}_\text{spike}} \frac{R_{\text{input}_\text{exp}}}{R_{\text{input}_\text{spike}}}}
\end{align*}
shows that $\alpha_\text{IP}$ will basically scale the experimental IP signal to the same scale as the experimental input signal, using the spike-in as a link between the two samples.
This makes it natural to subtract the normalized input signal from the normalized IP signal: since they are on the same scale, the resulting coverage can be interpreted as reporting how much more IP signal was detected than was expected based on the input.

\newpage
\bibliographystyle{apalike}
\begingroup
    \singlespacing
    \bibliography{references/spt5}
\endgroup

\cleardoublepage

\chapter{Stress-responsive intragenic transcription}
\label{chapter:stress}

\section{Abstract}

\lipsum[1]

\section{Collaborators}

\begin{description}[align=right, labelwidth=5cm, noitemsep, leftmargin=!]
    \item [Steve Doris] generated TSS-seq and ChIP-nexus libraries
    \item [Dan Spatt] polyribosome fractionation, fitness competitions,\\and other experiments
    \item [James Warner] fitness competitions and other experiments
\end{description}

\section{Possible functions for intragenic transcription in wild-type cells}

ASE1 \citep{mcknight2014}.
KAR4 \citep{gammie1999}.
ASP3 \citep{huang2010}.

\clearpage

\section{Discovery of stress-induced intragenic promoters by TFIIB ChIP-nexus and TSS-seq}

\begin{wrapfigure}[18]{r}{3in}
    \centering
    \includegraphics[width=3in]{figures/stress/stress_gasch_comparison.pdf}
    \caption[Scatterplots comparing change in genic TFIIB signal to change in RNA microarray signal, for oxidative and amino acid stresses.]{Scatterplots comparing change in genic TFIIB signal to change in RNA microarray signal from \citet{gasch2000}, for oxidative and amino acid stresses. The Pearson correlation coefficient is shown for each comparison.}
    \label{fig:stress_gasch_comparison}
\end{wrapfigure}
To discover cases of stress-induced intragenic transcription initiation, we performed ChIP-nexus of TFIIB in wild-type yeast in conditions of oxidative stress, amino acid stress, and nitrogen stress, along with controls of growth in rich YPD medium and defined SC medium.
The genic TFIIB response to each of the stresses either correlated well with the expected transcriptomic response to the stress (Figure \ref{fig:stress_gasch_comparison}), or was enriched for metabolic pathways consistent with the cellular response to the stress (Figure \ref{fig:stress_nitrogen_gene_ontology}), indicating that TFIIB ChIP-nexus is able to 
We identified 140 intragenic TFIIB peaks significantly induced at least 1.5-fold in at least one stress condition, with some peaks being induced in more than one stress (Figure \ref{fig:stress_tfiib_ridgelines}).
\begin{figure}[h]
    \centering
    \includegraphics[width=4in]{figures/stress/stress_nitrogen_gene_ontology.pdf}
    \caption[Gene ontology terms enriched in genes with upregulated genic TFIIB peaks in nitrogen stress.]{Gene ontology terms enriched in genes with significantly upregulated genic TFIIB peaks in nitrogen stress.}
    \label{fig:stress_nitrogen_gene_ontology}
\end{figure}


%k \begin{tabular}{l | l | l}
%                         & condition                 & control \\ \hline
%     oxidative stress    & 45 minutes in rich medium (YPD) + 1mM diamide & rich medium (YPD) \\
%     amino acid stress   & 30 minutes in defined medium lacking amino acids and adenine & defined medium (SC) \\
%     nitrogen stress     & 8 hours in defined medium lacking amino acids and adenine and with limiting concentrations of 
% \end{tabular}



% \begin{figure}
\begin{sidewaysfigure}
    \includegraphics[width=8.25in]{figures/stress/stress_tfiib_ridgelines.pdf}
    \caption[TFIIB ChIP-nexus protection over all genes with stress-induced intragenic TFIIB peaks.]{Relative TFIIB ChIP-nexus protection over all genes with an intragenic TFIIB peak significantly induced in one or more of the stress conditions tested, as depicted in the left panel. Genes are aligned by start codon, and are sorted within each group by the distance from the start codon to the summit of the induced intragenic TFIIB peak. Data are shown for each gene up to the stop codon of the gene. Regions where TFIIB peaks are called are shaded in the stress conditions according to the fold-change of the peak relative to the corresponding control condition.}
    \label{fig:stress_tfiib_ridgelines}
\end{sidewaysfigure}

\begin{figure}
\includegraphics[width=6in]{figures/stress/stress_tfiib_coverage.pdf}
\label{fig:stress_tfiib_coverage}
\caption[TFIIB ChIP-nexus protection over four genes with stress-induced intragenic TFIIB peaks.]{Caption asdflkj asldkfjlkj.}
\end{figure}

\begin{figure}
\includegraphics[width=3in]{figures/stress/stress_promoter_tss_diffexp_summary.pdf}
\caption[Bar plot of the number of promoters from various genomic classes differentially expressed in oxidative stress.]{Caption dsafklj asldkfjlkj.}
\label{fig:stress_promoter_tss_diffexp_summary}
\end{figure}

\begin{figure}
\includegraphics[width=6in]{figures/stress/stress_promoter_tss_expression.pdf}
\caption[TSS-seq expression levels in oxidative stress of oxidative-stress-induced genic and intragenic promoters.]{Caption dsafklj zzzz.}
\label{fig:stress_promoter_tss_diffexp_summary}
\end{figure}

\section{Chromatin landscape of oxidative-stress-induced promoters.}

\lipsum[1]

\begin{figure}
% \includegraphics[width=6in]{figures/stress/stress_promoter_tss_expression.pdf}
\caption[A figure showing TSS-seq, TFIIB ChIP-nexus, and MNase-ChIP-seq for the oxidative-stress-induced promoters.]{Caption dsafklj .}
% \label{fig:stress_promoter_tss_diffexp_summary}
\end{figure}

\section{Polysome enrichment of oxidative-stress-induced intragenic transcripts}

\lipsum[1]

\begin{figure}
\includegraphics[width=6in]{figures/stress/stress_promoter_tss_polyenrichment.pdf}
\caption[Polysome enrichment in oxidative stress, for oxidative-stress-induced genic and intragenic promoters.]{Caption wsdasdr zzzz.}
\label{fig:stress_promoter_tss_polyenrichment}
\end{figure}

\section{TSS-seq analysis of oxidative stress in \textit{Saccharomyces sensu stricto} species}

\lipsum[1]

\begin{figure}
% \includegraphics[width=6in]{figures/stress/stress_promoter_tss_expression.pdf}
\caption[A figure showing TSS-seq coverage over oxidative-stress-induced TSSs in the three species.]{Caption dsafklj .}
% \label{fig:stress_promoter_tss_diffexp_summary}
\end{figure}

\begin{figure}
% \includegraphics[width=6in]{figures/stress/stress_promoter_tss_expression.pdf}
\caption[A figure showing TSS-seq coverage over DSK2 in the three species, possibly with the corresponding northern blot.]{Caption dsafklj .}
% \label{fig:stress_promoter_tss_diffexp_summary}
\end{figure}

\section{Functions of intragenic DSK2 expression in oxidative stress}

\lipsum[1]

\begin{figure}
% \includegraphics[width=6in]{figures/stress/stress_promoter_tss_expression.pdf}
\caption[A figure showing TSS-seq, TFIIB ChIP-nexus, and MNase-ChIP-seq at DSK2.]{Caption dsafklj .}
% \label{fig:stress_promoter_tss_diffexp_summary}
\end{figure}

\begin{figure}
% \includegraphics[width=6in]{figures/stress/stress_promoter_tss_expression.pdf}
\caption[A figure showing DSK2 fitness competition results.]{Caption dsafklj .}
% \label{fig:stress_promoter_tss_diffexp_summary}
\end{figure}

\section{Discussion}

\lipsum[1]

\section{Methods}

\subsection{Yeast growth conditions}

\subsection{Genome builds}

\subsection{TFIIB ChIP-nexus data analysis}

\subsection{TSS-seq data analysis}

\subsection{MNase-ChIP-seq data analysis}

\subsection{Sucrose gradient fractionation}

\subsection{Polysome-associated TSS-seq analysis}

\subsection{Multiple genome alignment}

\subsection{Diamide competitive fitness assays}

\newpage
\bibliographystyle{apalike}
\begingroup
\singlespacing
\bibliography{references/stress}
\endgroup

\cleardoublepage

\newpage
\addcontentsline{toc}{chapter}{Bibliography}
\begingroup
    \singlespacing
    \bibliography{}
\endgroup
\cleardoublepage


\chapter*{Vita}
\addcontentsline{toc}{chapter}{Vita}

\textbf{James Chuang}

\begin{description}[align=right, labelwidth=4cm, noitemsep, leftmargin=!]
    \item [year of birth:] 1991
    \item [contact address:] 77 Avenue Louis Pasteur\\
                             Room 239\\
                             Boston, MA 02115
\end{description}

\noindent\hrulefill

\section*{Education}
\begin{description}[align=right, labelwidth=1.5cm, noitemsep, leftmargin=!]
    \item [2018] MSc, Biomedical Engineering, Boston University
    \item [2013] BSc, Biomedical Engineering, Johns Hopkins University
\end{description}


\section*{Publications}
\begingroup
\singlespacing
\begin{description}[align=right, labelwidth=1.5cm, itemsep=1em, leftmargin=!]
    \item [2018] Doris SM*, \textbf{Chuang J}*, Viktorovskaya O, Murawska M, Spatt D, Churchman LS, Winston F (2018). Spt6 is required for the fidelity of promoter selection. \textbf{Molecular Cell}, doi:\href{https://doi.org/10.1016/j.molcel.2018.09.005}{10.1016/j.molcel.2018.09.005}
    \item [2018] \textbf{Chuang J}, Boeke JD, Mitchell LA (2018). Coupling Yeast Golden Gate and VEGAS for Efficient Assembly of the Violacein Pathway in \textit{Saccharomyces cerevisiae}. \textbf{Synthetic Metabolic Pathways}, doi:\href{https://doi.org/10.1007/978-1-4939-7295-1_14}{10.1007/978-1-4939-7295-1\_14}
    \item [2017] Aquino P, Honda B, Suma Jaini, Lyubetskaya A, Hosur K, Chiu JG, Ekladious I, Hu D, Jin L, Sayeg MK, Stettner AI, Wang J, Wong BG, Wong WS, Alexander SL, Ba C, Bensussen SI, Chou K, \textbf{Chuang J}, Gastler DE, Grasso DJ, Greifenberger JS, Guo C, Hawes AK, Israni DV, Jain SR, Kim J, Lei J, Li H, Li D, Li Q, Mancuso CP, Mao N, Masud SF, Meisel CL, Mi J, Nykyforchyn CS, Park M, Peterson HM, Ramirez AK, Reynolds DS, Rim NG, Saffie JC, Su H, Su WR, Su Y, Sun M, Thommes MM, Tu T, Varongchayakul N, Wagner TE, Weinberg BH, Yang R, Yaroslavsky A, Yoon C, Zhao Y, Zollinger AJ, Stringer AM, Foster JW, Wade J, Raman S, Broude N, Wong WW, Galagan JE (2017). Coordinated regulation of acid resistance in \textit{Escherichia coli}. \textbf{BMC Systems Biology}, doi:\href{https://doi.org/10.1186/s12918-016-0376-y}{10.1186/s12918-016-0376-y}
    \item [2015] Mitchell, LA*, \textbf{Chuang J}*, Agmon N, Khunsriraksakul C, Phillips NA, Cai Y, Truong DM, Veerakumar A, Wang Y, Mayorga M, Blomquist P, Sadda P, Trueheart J, Boeke JD (2015). Versatile genetic assembly system (VEGAS) to assemble pathways for expression in \textit{S. cerevisiae}. \textbf{Nucleic Acids Research}, doi:\href{https://doi.org/10.1093/nar/gkv466}{10.1093/nar/gkv466}
    \item [2015] Agmon N, Mitchell LA, Cai Y, Ikushima S, \textbf{Chuang J}, Zheng A, Choi W, Martin JA, Caravelli K, Stracquadanio G, Boeke JD (2015). Yeast Golden Gate (yGG) for the Efficient Assembly of \textit{S. cerevisiae} Transcription Units. \textbf{ACS Synthetic Biology}, doi:\href{https://doi.org/10.1021/sb500372z}{10.1021/sb500372z}
    \item [2013] Mitchell LA, Cai Y, Taylor M, Noronha AM, \textbf{Chuang J}, Dai L, Boeke JD (2013). Multichange isothermal mutagenesis: a new strategy for multiple site-directed mutations in plasmid DNA. \textbf{ACS Synthetic Biology}, \\doi:\href{https://doi.org/10.1021/sb300131w}{10.1021/sb300131w}
\end{description}
\endgroup


\end{document}
