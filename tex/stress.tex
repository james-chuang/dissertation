\chapter{Stress-responsive intragenic transcription}
\label{chapter:stress}

\section{Abstract}

Intragenic transcription opens up interesting possibilities for gene regulation in wild-type cells, and individual cases of intragenic transcription have previously been shown to be important for different biological processes.
To discover uncharacterized instances of intragenic transcription, we applied high resolution genomic assays of transcription initiation to wild-type \textit{Saccharomyces cerevisiae} cells in three stress conditions.
Focusing on the condition of oxidative stress, we characterize the expression and chromatin state of oxidative-stress-induced intragenic transcripts, and show that many of these transcripts are likely translated.
Finally, we show that the expression of an oxidative-stress-induced intragenic transcript at the gene \textit{DSK2} is needed for survival in conditions of oxidative stress.

\section{Collaborators}

\begin{description}[align=right, labelwidth=5cm, noitemsep, leftmargin=!]
    \item [Steve Doris] generated TSS-seq and ChIP-nexus libraries
    \item [Dan Spatt] polyribosome fractionation and competitive growth assays
    \item [James Warner] Northern blots and competitive growth assays
\end{description}

\section{Functions of intragenic transcription in wild-type cells}

In chapters \ref{chapter:six} and \ref{chapter:five}, we presented examples in which mutation of certain transcription elongation factors leads to the expression of intragenic transcripts, i.e., transcripts initiating from within gene bodies.
This implies that wild-type cells have evolved to prevent the occurrence of spurious intragenic transcription.
Despite this, our studies and others \citep{malabat2015, pelechano2013} consistently detect some instances of intragenic transcription in wild-type cells, some of which could have biological functions.
% Intragenic transcription overlaps genic transcription, by definition, making it plausible that the function of an instance of intragenic transcription could be to regulate genic expression, in addition to possible functions that are independent of genic transcription.

One mechanism by which intragenic transcription can exert a function is by being translated into a functional polypeptide or protein.
Translation of a sense strand intragenic transcript is likely to lead to production of an N-terminally truncated protein, which may differ from the full-length protein in stability, subcellular localization, or function due to the absence of domains normally found in the full-length protein \citep{benanti2009, gammie1999, mcknight2014}.
N-terminally truncated proteins can even exert dominant negative effects: Under DNA replication stress, intragenic transcription of the \textit{S. cerevisiae} mitotic spindle microtubule bundling gene \textit{Ase1} leads to the expression of an Ase1 protein lacking an N-terminal dimerization domain \citep{mcknight2014}.
The intragenic Ase1 protein still possesses the normal C-terminal microtubule-binding domain, and evidence suggests that by binding microtubules but not dimerizing, intragenic Ase1 antagonizes the function of full-length Ase1.
This interaction is important for the cellular response to replicative stress.

In other cases, the process of intragenic transcription may be functionally important, rather than the product of intragenic transcription.
As transcription occurs, co-transcriptional processses including chromatin remodeling and histone modification also take place via the recruitment of various factors to the transcription complex.
These factors are often recruited in a manner dependent on the stage of transcription, which results in stereotypic patterns of chromatin state over transcribed regions.
For example, in \textit{S. cerevisiae}, the Set1 complex responsible for methylation of histone H3 lysine 4 (H3K4) is recruited to Pol II specifically during early elongation, resulting in a pattern where H3K4 trimethylation (H3K4me3) is primarily found over the first 500 base pairs of transcribed regions.
Since intragenic transcribed regions overlap genic transcribed regions, intragenic transcription has the potential to generate unususal patterns of nucleosome positioning and histone modification over a gene, which may influence expression from the genic promoter.
This occurs at the \textit{S. cerevisiae} asparagine catabolic gene \textit{ASP3}, where intragenic transcription is necessary for wild-type levels of H3K4me3 over the 5$^\prime$-end of the gene and full induction of the gene upon nitrogen stress \citep{huang2010}.

RNAi.
R-loops.
Transcription interference.
No function.

Linking transcript to function.

% ASE1 \citep{mcknight2014}.
% KAR4 \citep{gammie1999}.
% CIK1 \citep{benanti2009}.
% ASP3 \citep{huang2010}.

\clearpage

\section{Discovery of stress-induced intragenic promoters by TFIIB ChIP-nexus and TSS-seq}

\begin{wrapfigure}[20]{r}{3in}
    \centering
    \includegraphics[width=3in]{figures/stress/stress_gasch_comparison.pdf}
    \caption[Scatterplots comparing change in genic TFIIB signal to change in RNA microarray signal, for oxidative and amino acid stresses.]{Scatterplots comparing change in genic TFIIB signal to change in RNA microarray signal from \citet{gasch2000}, for oxidative and amino acid stresses. The Pearson correlation coefficient is shown for each comparison.}
    \label{fig:stress_gasch_comparison}
\end{wrapfigure}

To discover cases of stress-induced intragenic transcription initiation, we performed ChIP-nexus of TFIIB for wild-type yeast under conditions of oxidative stress, amino acid stress, and nitrogen stress, along with controls of growth in rich YPD medium and defined SC medium.
The genic TFIIB response to each of the stresses either correlates well with the expected transcriptomic response to the stress (Figure \ref{fig:stress_gasch_comparison}), or is enriched for metabolic pathways consistent with the cellular response to the stress (Figure \ref{fig:stress_nitrogen_gene_ontology}), confirming that TFIIB ChIP-nexus captures changes in transcription initiation and that the cells were stressed as intended.
In total, we identify 140 intragenic TFIIB peaks significantly induced at least 1.5-fold in at least one stress condition, with some peaks being induced in more than one stress (Figures \ref{fig:stress_tfiib_coverage}, \ref{fig:stress_tfiib_ridgelines}).
We also observe a slight negative correlation between stress-induced changes in intragenic TFIIB signal and changes in the corresponding genic TFIIB signal (Figure \ref{fig:stress_genic_vs_intra}).

\begin{figure}[h]
    \centering
    \includegraphics[width=4in]{figures/stress/stress_nitrogen_gene_ontology.pdf}
    \caption[Gene ontology terms enriched in genes with nitrogen-stress-induced genic TFIIB peaks]{Gene ontology terms enriched in genes with nitrogen-stress-induced genic TFIIB peaks.}
    \label{fig:stress_nitrogen_gene_ontology}
\end{figure}

\begin{figure}[h]
    \includegraphics[width=6in]{figures/stress/stress_tfiib_coverage.pdf}
    \caption[TFIIB ChIP-nexus protection over four genes with stress-induced intragenic TFIIB peaks.]{Relative TFIIB ChIP-nexus protection over four genes with an intragenic TFIIB peak significantly induced in one or more of the stress conditions.}
    \label{fig:stress_tfiib_coverage}
\end{figure}

\clearpage

\begin{sidewaysfigure}
    \includegraphics[width=8.25in]{figures/stress/stress_tfiib_ridgelines.pdf}
    \caption[TFIIB ChIP-nexus protection over all genes with stress-induced intragenic TFIIB peaks.]{Relative TFIIB ChIP-nexus protection over all genes with an intragenic TFIIB peak significantly induced in one or more of the stress conditions tested, as depicted in the left panel. Genes are aligned by start codon, and are sorted within each group by the distance from the start codon to the summit of the induced intragenic TFIIB peak. Data are shown for each gene up to the stop codon of the gene. Regions where TFIIB peaks are called are shaded in the stress conditions according to the fold-change of the peak relative to the corresponding control condition.}
    \label{fig:stress_tfiib_ridgelines}
\end{sidewaysfigure}

\clearpage

\begin{figure}[h]
    \includegraphics[width=4in]{figures/stress/stress_genic_vs_intra.pdf}
    \caption[Scatterplot of change in intragenic versus genic TFIIB ChIP-nexus signal, for all pairs of intragenic and genic TFIIB peaks in the three stress conditions.]{Scatterplot comparing change in intragenic TFIIB ChIP-nexus signal to the change in genic TFIIB signal at the same gene, for all pairs of intragenic and genic TFIIB peaks in the three stress conditions. Error bars indicate $\pm$ 1 standard error, and the Pearson correlation coefficient is shown.}
    \label{fig:stress_genic_vs_intra}
\end{figure}

Because the greatest changes to intragenic transcription initiation were detected in oxidative stress, we focused on this condition and performed TSS-seq to determine which intragenic initiation events produce stable RNAs and in which strand orientation these events occur.
Considering only TSS peaks with a TFIIB peak overlapping the window extending 200 base pairs upstream of the TSS summit, we find cases of both sense intragenic and antisense TSSs that are differentially expressed in oxidative stress (Figure \ref{fig:stress_promoter_tss_diffexp_summary}).
In general, intragenic TSSs are expressed at lower levels than genic TSSs: Among oxidative-stress-induced TSSs, the most abundant intragenic TSS in oxidative stress is present at levels comparable to the 54\textsuperscript{th} percentile of genic TSS abundances (Figure \ref{fig:stress_promoter_tss_expression}).

\begin{figure}[h]
    \includegraphics[width=3in]{figures/stress/stress_promoter_tss_diffexp_summary.pdf}
    \caption[Bar plot of the number of promoters in various genomic classes detected as differentially expressed in oxidative stress.]{Bar plot of the number of promoters in various genomic classes detected as differentially expressed in oxidative stress.}
    \label{fig:stress_promoter_tss_diffexp_summary}
\end{figure}

\begin{figure}[h]
    \includegraphics[width=6in]{figures/stress/stress_promoter_tss_expression.pdf}
    \caption[TSS-seq expression levels in oxidative stress of oxidative-stress-induced genic and intragenic promoters.]{Cumulative distributions of TSS-seq expression levels in oxidative stress, for all genic and intragenic promoters significantly induced in oxidative stress. Error bars indicate $\pm$ one standard deviation.}
    \label{fig:stress_promoter_tss_expression}
\end{figure}

\clearpage

\section{Polysome enrichment of oxidative-stress-induced intragenic transcripts}

Translation of a transcript requires that the transcript possess both a 5$^\prime$-cap and a poly-A tail.
Since the TSS-seq protocol enriches for both of these features, this implies that the oxidative-stress-dependent intragenic transcripts we detect by TSS-seq could potentially be translated.
For sense strand intragenic transcripts, this could generate N-terminally-truncated protein isoforms.
To see how likely oxidative-stress-dependent intragenic transcripts are to be targets for translation, we performed sucrose gradient fractionation, isolated the RNA from the fractions containing polysomes, and sequenced TSS-seq libraries of the polysome-associated RNA.
Among oxidative-stress-induced TSSs with corresponding TFIIB peaks, intragenic TSSs in oxidative stress are less somewhat less enriched in the polysome fraction compared to genic TSSs (Figure \ref{fig:stress_promoter_tss_polyenrichment}).
However, half of the oxidative-stress-induced intragenic TSSs are found in polysomes during oxidative stress at levels greater than that of the 25\textsuperscript{th} percentile of oxidative-stress-induced genic TSSs, indicating that many of the intragenic transcripts we identify are translated at some level.

\begin{figure}[h]
    \includegraphics[width=6in]{figures/stress/stress_promoter_tss_polyenrichment.pdf}
    \caption[Relative polysome enrichment in oxidative stress, for oxidative-stress-induced genic and intragenic promoters.]{Relative polysome enrichment in oxidative stress, for oxidative-stress-induced genic and intragenic promoters. Error bars indicate $\pm$ one standard error.}
    \label{fig:stress_promoter_tss_polyenrichment}
\end{figure}

\section{Functions of intragenic DSK2 expression in oxidative stress}

To investigate a particular case of oxidative-stress-induced intragenic transcription in more depth, we focused on the gene \textit{DSK2}, which is associated with the intragenic TSS we found to be most significantly induced upon oxidative stress.
The major \textit{DSK2} intragenic TSS is associated with a TFIIB peak that is similarly induced in oxidative stress (Figure \ref{fig:stress_dsk2_summary}), and the intragenic transcript is also found in polysomes at intermediate levels (Figure \ref{fig:stress_promoter_tss_polyenrichment}).
\begin{figure}[h]
    \includegraphics[width=6in]{figures/stress/stress_dsk2_summary.pdf}
    \caption[Sense TSS-seq signal, TFIIB ChIP-nexus protection, and MNase-ChIP-seq data at the \textit{DSK2} gene, over an oxidative stress timecourse.]{Sense TSS-seq signal (purple), TFIIB ChIP-nexus protection (grey), smoothed MNase-seq dyad signal, and relative H3K4me3 MNase-ChIP-seq enrichment over the gene \textit{DSK2}, over a timecourse of oxidative stress. MNase-seq and MNase-ChIP-seq data are from \citet{weiner2015}. The shaded bars on the \textit{DSK2} ORF indicate PACE-core elements on the sense (darker) or antisense (lighter) strands, which are potential binding sites for the transcription factor Rpn4 \citep{shirozu2015}. The lightly shaded region of the background indicates the boundaries of the intragenic TFIIB peak.}
    \label{fig:stress_dsk2_summary}
\end{figure}

\begin{wrapfigure}[10]{r}{3in}
    \centering
    \includegraphics[width=3in]{figures/stress/stress_dsk2_pace_northern.pdf}
    \caption[Northern blot for \textit{DSK2} transcripts in wild-type \textit{DSK2} and \textit{dsk2-pace} strains, in the absence or presence of oxidative stress.]{Northern blot for \textit{DSK2} transcripts in wild-type \textit{DSK2} and \textit{dsk2-pace} strains, in the absence or presence of oxidative stress induced by addition of diamide to the media.}
    \label{fig:stress_dsk2_pace_northern}
\end{wrapfigure}

Dsk2 is a member of a family of partially redundant ubiquitin receptors that shuttle polyubiquitinated proteins to the proteasome for degradation.
Upon inspection of the \textit{DSK2} DNA sequence, we discovered multiple PACE-core sequences within the \textit{DSK2} coding sequence, which are potential binding sites for the proteasome transcription factor Rpn4 \citep{shirozu2015}.
By making silent mutations to three PACE-core elements occurring just upstream of the \textit{DSK2} intragenic TFIIB peak (the more darkly shaded sense strand PACE-core elements in Figure \ref{fig:stress_dsk2_summary}, two of which are adjacent to each other), we generated a strain named \textit{dsk2-pace} in which intragenic \textit{DSK2} expression is eliminated.

\begin{figure}
    \includegraphics[width=6in]{figures/stress/stress_diamide_fitnesscomp.pdf}
    \caption[Percentage of \textit{dsk2-pace} cells over two days of competitive growth against wild-type \textit{DSK2} cells at various concentrations of diamide.]{Percentage of \textit{dsk2-pace} cells over two days of competitive growth against wild-type \textit{DSK2} cells at various concentrations of diamide. The lines and shading are the mean $\pm$ one standard deviation of six replicates: Three in which \textit{dsk2-pace} and \textit{DSK2} cells are respectively marked by expression of mCherry and YFP, and three in which the fluorophores are swapped between strains.}
    \label{fig:stress_diamide_fitnesscomp}
\end{figure}

\section{TSS-seq analysis of oxidative stress in \textit{Saccharomyces sensu stricto} species}

\lipsum[1]

\begin{figure}
    % \includegraphics[width=6in]{figures/stress/stress_promoter_tss_expression.pdf}
    \caption[A figure showing TSS-seq coverage over oxidative-stress-induced TSSs in the three species.]{Caption dsafklj .}
    % \label{fig:stress_promoter_tss_diffexp_summary}
\end{figure}

\begin{figure}
    \includegraphics[width=6in]{figures/stress/stress_dsk2_interyeast.pdf}
    \caption[Sense TSS-seq signal over the \textit{DSK2} gene in \textit{Saccharomyces cerevisiae, mikatae, and uvarum} in unstressed and oxidative stress conditions.]{Relative sense TSS-seq signal over the \textit{DSK2} gene in \textit{Saccharomyces cerevisiae, mikatae, and uvarum} in unstressed and oxidative stress conditions. The shaded bars on the \textit{DSK2} gene diagrams indicate PACE-core elements on the sense (darker) or antisense (lighter) strands. The lightly shaded regions in the background indicate the boundaries of regions homologous to the two \textit{DSK2} intragenic TSS-seq peaks called in \textit{S. cerevisiae}, for each species.}
    \label{fig:stress_dsk2_interyeast}
\end{figure}

\section{Discussion}

\lipsum[1]

\section{Methods}

\subsection{Yeast growth conditions}

\subsection{Genome builds}

\subsection{TFIIB ChIP-nexus data analysis}

\subsection{TSS-seq data analysis}

\subsection{MNase-ChIP-seq data analysis}

\subsection{Sucrose gradient fractionation}

\subsection{Polysome-associated TSS-seq analysis}

\subsection{Multiple genome alignment}

\subsection{Diamide competitive fitness assays}

\newpage
\bibliographystyle{apalike}
\begingroup
\singlespacing
\bibliography{references/stress}
\endgroup
