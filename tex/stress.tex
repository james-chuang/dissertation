\chapter{Stress-responsive intragenic transcription}
\label{chapter:stress}

\section{Collaborators}

\begin{description}[align=right, labelwidth=5cm, noitemsep]
    \item [Steve Doris] generated TSS-seq and ChIP-nexus libraries
    \item [Dan Spatt] polyribosome fractionation, fitness competitions,
    \item [] and other experiments
    \item [James Warner] fitness competitions and other experiments
\end{description}

\section{Possible functions for intragenic transcription in wild-type cells}

\section{Discovery of stress-induced intragenic promoters by TFIIB ChIP-nexus and TSS-seq}

\begin{figure}
    \includegraphics[width=6in]{figures/stress/stress_tfiib_ridgelines.pdf}
    \caption[TFIIB ChIP-nexus protection over all genes with stress-induced intragenic TFIIB peaks.]{Relative TFIIB ChIP-nexus protection over all genes with an intragenic TFIIB peak significantly induced in one or more of the stress conditions tested, as depicted in the left panel. Genes are aligned by start codon, and are sorted within each group by the distance from the start codon to the summit of the induced intragenic TFIIB peak. Data are shown for each gene up to the stop codon of the gene. Regions where TFIIB peaks are called are shaded in the stress conditions according to the fold-change of the peak relative to the corresponding control condition.}
    \label{fig:stress_tfiib_ridgelines}
\end{figure}

\begin{figure}
\includegraphics[width=6in]{figures/stress/stress_tfiib_coverage.pdf}
\label{fig:stress_tfiib_coverage}
\caption[TFIIB ChIP-nexus protection over four genes with stress-induced intragenic TFIIB peaks.]{Caption asdflkj asldkfjlkj.}
\end{figure}

\begin{figure}
\includegraphics[width=3in]{figures/stress/stress_promoter_tss_diffexp_summary.pdf}
\caption[Bar plot of the number of promoters from various genomic classes differentially expressed in oxidative stress.]{Caption dsafklj asldkfjlkj.}
\label{fig:stress_promoter_tss_diffexp_summary}
\end{figure}

\begin{figure}
\includegraphics[width=6in]{figures/stress/stress_promoter_tss_expression.pdf}
\caption[TSS-seq expression levels in oxidative stress of oxidative-stress-induced genic and intragenic promoters.]{Caption dsafklj zzzz.}
\label{fig:stress_promoter_tss_diffexp_summary}
\end{figure}

\section{Chromatin landscape of oxidative-stress-induced promoters.}

\begin{figure}
% \includegraphics[width=6in]{figures/stress/stress_promoter_tss_expression.pdf}
\caption[A figure showing TSS-seq, TFIIB ChIP-nexus, and MNase-ChIP-seq for the oxidative-stress-induced promoters.]{Caption dsafklj .}
% \label{fig:stress_promoter_tss_diffexp_summary}
\end{figure}

\section{Polysome enrichment of oxidative-stress-induced intragenic transcripts}

\begin{figure}
\includegraphics[width=6in]{figures/stress/stress_promoter_tss_polyenrichment.pdf}
\caption[Polysome enrichment in oxidative stress, for oxidative-stress-induced genic and intragenic promoters.]{Caption wsdasdr zzzz.}
\label{fig:stress_promoter_tss_polyenrichment}
\end{figure}

\section{TSS-seq analysis of oxidative stress in \textit{Saccharomyces sensu stricto} species}

\begin{figure}
% \includegraphics[width=6in]{figures/stress/stress_promoter_tss_expression.pdf}
\caption[A figure showing TSS-seq coverage over oxidative-stress-induced TSSs in the three species.]{Caption dsafklj .}
% \label{fig:stress_promoter_tss_diffexp_summary}
\end{figure}

\begin{figure}
% \includegraphics[width=6in]{figures/stress/stress_promoter_tss_expression.pdf}
\caption[A figure showing TSS-seq coverage over DSK2 in the three species, possibly with the corresponding northern blot.]{Caption dsafklj .}
% \label{fig:stress_promoter_tss_diffexp_summary}
\end{figure}

\section{Functions of intragenic DSK2 expression in oxidative stress}

\begin{figure}
% \includegraphics[width=6in]{figures/stress/stress_promoter_tss_expression.pdf}
\caption[A figure showing TSS-seq, TFIIB ChIP-nexus, and MNase-ChIP-seq at DSK2.]{Caption dsafklj .}
% \label{fig:stress_promoter_tss_diffexp_summary}
\end{figure}

\begin{figure}
% \includegraphics[width=6in]{figures/stress/stress_promoter_tss_expression.pdf}
\caption[A figure showing DSK2 fitness competition results.]{Caption dsafklj .}
% \label{fig:stress_promoter_tss_diffexp_summary}
\end{figure}

\section{Discussion}

\section{Methods}

\newpage
\bibliographystyle{apalike}
\begingroup
\singlespacing
\bibliography{references/stress}
\endgroup
