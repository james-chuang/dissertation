\chapter{Stress-responsive intragenic transcription}
\label{chapter:stress}

\section{Collaborators}

\begin{description}[align=right, labelwidth=5cm, noitemsep, leftmargin=!]
    \item [Steve Doris] generated TSS-seq and ChIP-nexus libraries
    \item [Dan Spatt] polyribosome fractionation, fitness competitions,\\and other experiments
    \item [James Warner] fitness competitions and other experiments
\end{description}

\section{Possible functions for intragenic transcription in wild-type cells}

ASE1 \citep{mcknight2014}.
KAR4 \citep{gammie1999}.
ASP3 \citep{huang2010}.

\clearpage

\section{Discovery of stress-induced intragenic promoters by TFIIB ChIP-nexus and TSS-seq}

\begin{wrapfigure}[18]{r}{3in}
    \centering
    \includegraphics[width=3in]{figures/stress/stress_gasch_comparison.pdf}
    \caption[Scatterplots comparing change in genic TFIIB signal to change in RNA microarray signal, for oxidative and amino acid stresses.]{Scatterplots comparing change in genic TFIIB signal to change in RNA microarray signal from \citet{gasch2000}, for oxidative and amino acid stresses. The Pearson correlation coefficient is shown for each comparison.}
    \label{fig:stress_gasch_comparison}
\end{wrapfigure}
To discover cases of stress-induced intragenic transcription initiation, we performed ChIP-nexus of TFIIB in wild-type yeast in conditions of oxidative stress, amino acid stress, and nitrogen stress, along with controls of growth in rich YPD medium and defined SC medium.
The genic TFIIB response to each of the stresses either correlated well with the expected transcriptomic response to the stress (Figure \ref{fig:stress_gasch_comparison}), or was enriched for metabolic pathways consistent with the cellular response to the stress (Figure \ref{fig:stress_nitrogen_gene_ontology}), indicating that TFIIB ChIP-nexus is able to 
We identified 140 intragenic TFIIB peaks significantly induced at least 1.5-fold in at least one stress condition, with some peaks being induced in more than one stress (Figure \ref{fig:stress_tfiib_ridgelines}).
\begin{figure}[h]
    \centering
    \includegraphics[width=4in]{figures/stress/stress_nitrogen_gene_ontology.pdf}
    \caption[Gene ontology terms enriched in genes with upregulated genic TFIIB peaks in nitrogen stress.]{Gene ontology terms enriched in genes with significantly upregulated genic TFIIB peaks in nitrogen stress.}
    \label{fig:stress_nitrogen_gene_ontology}
\end{figure}


%k \begin{tabular}{l | l | l}
%                         & condition                 & control \\ \hline
%     oxidative stress    & 45 minutes in rich medium (YPD) + 1mM diamide & rich medium (YPD) \\
%     amino acid stress   & 30 minutes in defined medium lacking amino acids and adenine & defined medium (SC) \\
%     nitrogen stress     & 8 hours in defined medium lacking amino acids and adenine and with limiting concentrations of 
% \end{tabular}



% \begin{figure}
\begin{sidewaysfigure}
    \includegraphics[width=8.25in]{figures/stress/stress_tfiib_ridgelines.pdf}
    \caption[TFIIB ChIP-nexus protection over all genes with stress-induced intragenic TFIIB peaks.]{Relative TFIIB ChIP-nexus protection over all genes with an intragenic TFIIB peak significantly induced in one or more of the stress conditions tested, as depicted in the left panel. Genes are aligned by start codon, and are sorted within each group by the distance from the start codon to the summit of the induced intragenic TFIIB peak. Data are shown for each gene up to the stop codon of the gene. Regions where TFIIB peaks are called are shaded in the stress conditions according to the fold-change of the peak relative to the corresponding control condition.}
    \label{fig:stress_tfiib_ridgelines}
\end{sidewaysfigure}

\begin{figure}
\includegraphics[width=6in]{figures/stress/stress_tfiib_coverage.pdf}
\label{fig:stress_tfiib_coverage}
\caption[TFIIB ChIP-nexus protection over four genes with stress-induced intragenic TFIIB peaks.]{Caption asdflkj asldkfjlkj.}
\end{figure}

\begin{figure}
\includegraphics[width=3in]{figures/stress/stress_promoter_tss_diffexp_summary.pdf}
\caption[Bar plot of the number of promoters from various genomic classes differentially expressed in oxidative stress.]{Caption dsafklj asldkfjlkj.}
\label{fig:stress_promoter_tss_diffexp_summary}
\end{figure}

\begin{figure}
\includegraphics[width=6in]{figures/stress/stress_promoter_tss_expression.pdf}
\caption[TSS-seq expression levels in oxidative stress of oxidative-stress-induced genic and intragenic promoters.]{Caption dsafklj zzzz.}
\label{fig:stress_promoter_tss_diffexp_summary}
\end{figure}

\section{Chromatin landscape of oxidative-stress-induced promoters.}

\lipsum[1]

\begin{figure}
% \includegraphics[width=6in]{figures/stress/stress_promoter_tss_expression.pdf}
\caption[A figure showing TSS-seq, TFIIB ChIP-nexus, and MNase-ChIP-seq for the oxidative-stress-induced promoters.]{Caption dsafklj .}
% \label{fig:stress_promoter_tss_diffexp_summary}
\end{figure}

\section{Polysome enrichment of oxidative-stress-induced intragenic transcripts}

\lipsum[1]

\begin{figure}
\includegraphics[width=6in]{figures/stress/stress_promoter_tss_polyenrichment.pdf}
\caption[Polysome enrichment in oxidative stress, for oxidative-stress-induced genic and intragenic promoters.]{Caption wsdasdr zzzz.}
\label{fig:stress_promoter_tss_polyenrichment}
\end{figure}

\section{TSS-seq analysis of oxidative stress in \textit{Saccharomyces sensu stricto} species}

\lipsum[1]

\begin{figure}
% \includegraphics[width=6in]{figures/stress/stress_promoter_tss_expression.pdf}
\caption[A figure showing TSS-seq coverage over oxidative-stress-induced TSSs in the three species.]{Caption dsafklj .}
% \label{fig:stress_promoter_tss_diffexp_summary}
\end{figure}

\begin{figure}
% \includegraphics[width=6in]{figures/stress/stress_promoter_tss_expression.pdf}
\caption[A figure showing TSS-seq coverage over DSK2 in the three species, possibly with the corresponding northern blot.]{Caption dsafklj .}
% \label{fig:stress_promoter_tss_diffexp_summary}
\end{figure}

\section{Functions of intragenic DSK2 expression in oxidative stress}

\lipsum[1]

\begin{figure}
% \includegraphics[width=6in]{figures/stress/stress_promoter_tss_expression.pdf}
\caption[A figure showing TSS-seq, TFIIB ChIP-nexus, and MNase-ChIP-seq at DSK2.]{Caption dsafklj .}
% \label{fig:stress_promoter_tss_diffexp_summary}
\end{figure}

\begin{figure}
% \includegraphics[width=6in]{figures/stress/stress_promoter_tss_expression.pdf}
\caption[A figure showing DSK2 fitness competition results.]{Caption dsafklj .}
% \label{fig:stress_promoter_tss_diffexp_summary}
\end{figure}

\section{Discussion}

\lipsum[1]

\section{Methods}

\subsection{Yeast growth conditions}

\subsection{Genome builds}

\subsection{TFIIB ChIP-nexus data analysis}

\subsection{TSS-seq data analysis}

\subsection{MNase-ChIP-seq data analysis}

\subsection{Sucrose gradient fractionation}

\subsection{Polysome-associated TSS-seq analysis}

\subsection{Multiple genome alignment}

\subsection{Diamide competitive fitness assays}

\newpage
\bibliographystyle{apalike}
\begingroup
\singlespacing
\bibliography{references/stress}
\endgroup
