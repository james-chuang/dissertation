

% Type of document prepared for this degree:
%   1 = Master of Science thesis,
%   2 = Doctor of Philosophy dissertation.
%   3 = Master of Science thesis and Doctor of Philisophy dissertation.
\degree=2

\prevdegrees{B.S., Johns Hopkins University, 2013\\
	M.S., Boston University, 2018}

\department{Department of Biomedical Engineering}

% Degree year is the year the diploma is expected, and defense year is
% the year the dissertation is written up and defended. Often, these
% will be the same, except for January graduation, when your defense
% will be in the fall of year X, and your graduation will be in
% January of year X+1
\defenseyear{2019}
\degreeyear{2019}

% For each reader, specify appropriate label {First, Second, Third},
% then name, and title. IMPORTANT: The title should be:
%   "Professor of Electrical and Computer Engineering",
% or similar, but it MUST NOT be:
%   Professor, Department of Electrical and Computer Engineering"
% or you will be asked to reprint and get new signatures.
% Warning: If you have more than five readers you are out of luck,
% because it will overflow to a new page. You may try to put part of
% the title in with the name.
\reader{First}{Fred Winston, PhD}{Professor of Genetics\\Harvard Medical School}
\reader{Second}{Ahmad Khalil, PhD}{Associate Professor of Biomedical Engineering}
\reader{Third}{L. Stirling Churchman, PhD}{Associate Professor of Genetics\\Harvard Medical School}
\reader{Fourth}{John T. Ngo, PhD}{Assistant Professor of Biomedical Engineering}
\reader{Fifth}{Wilson Wong, PhD}{Associate Professor of Biomedical Engineering}

% The Major Professor is the same as the first reader, but must be
% specified again for the abstract page. Up to 4 Major Professors
% (advisors) can be defined.
\numadvisors=2
\majorprof{Fred Winston, PhD}{{Professor of Genetics\\Harvard Medical School}}
\majorprofb{Ahmad Khalil, PhD}{{Associate Professor of Biomedical Engineering}}
%\majorprofc{First M. Last, PhD}{{Professor of Astronomy}}
%\majorprofd{First M. Last, PhD}{{Professor of Biomedical Engineering}}

\maketitle
\cleardoublepage

% The copyright page is blank except for the notice at the bottom. You
% must provide your name in capitals.
\copyrightpage
\cleardoublepage

% Now include the approval page based on the readers information
\approvalpage
\cleardoublepage

% % The acknowledgment page should go here. Use something like
% % \newpage\section*{Acknowledgments} followed by your text.
% \newpage
% \section*{\centerline{Acknowledgments}}
% \vskip 1in
% \noindent
% James Chuang
% \cleardoublepage

% The abstractpage environment sets up everything on the page except
% the text itself.  The title and other header material are put at the
% top of the page, and the supervisors are listed at the bottom.  A
% new page is begun both before and after.  Of course, an abstract may
% be more than one page itself.  If you need more control over the
% format of the page, you can use the abstract environment, which puts
% the word "Abstract" at the beginning and single spaces its text.

\begin{abstractpage}

Transcription of protein-coding genes in eukaryotic cells is carried out by the protein complex RNA polymerase II.
During the elongation phase of transcription, RNA polymerase II associates with transcription elongation factors which modulate the activity of the transcription complex and are needed to carry out co-transcriptional processes.
Chapters \ref{chapter:six} and \ref{chapter:five} of this dissertation describe studies of Spt6 and Spt5, two conserved transcription elongation factors.

Spt6 is a transcription elongation factor thought to replace nucleosomes in the wake of transcription.
\textit{Saccharomyces cerevisiae} \textit{spt6} mutants express elevated levels of intragenic transcripts, transcripts appearing to initiate from within gene bodies.
We applied high resolution genomic assays of transcription initiation to an \textit{spt6-1004} mutant, allowing us to catalog the full extent of intragenic transcription in \textit{spt6-1004} and show for the first time on a genome-wide scale that the intragenic transcripts observed in \textit{spt6-1004} are largely explained by new transcription initiation.
We also assayed chromatin structure genome-wide in \textit{spt6-1004}, finding a global depletion and disordering of nucleosomes.
In addition to increased intragenic transcription in \textit{spt6-1004}, our results also reveal an unexpected decrease in expression from most canonical genic promoters.
Comparing intragenic and genic promoters, we find that intragenic promoters share some features with genic promoters.
Altogether, we propose that the transcriptional changes in \textit{spt6-1004} are explained by a competition for transcription initiation factors between genic and intragenic promoters, which is made possible by a global decrease in nucleosome protection of the genome.

Spt5 is another transcription elongation factor, important for the processivity of the transcription complex and many transcription-related processes.
To study the requirement for Spt5 \textit{in vivo}, we applied multiple genomic assays to \textit{Schizosaccharomyces pombe} cells depleted of Spt5.
Our results reveal an accumulation of RNA polymerase II over the 5$^\prime$ ends of genes upon Spt5 depletion, and a progressive decrease in transcript abundance towards the 3$^\prime$ ends of genes.
This is consistent with a model in which Spt5 depletion causes transcription elongation defects and increases early termination.
We also unexpectedly discover that Spt5 depletion causes hundreds of antisense transcripts to be expressed across the genome, primarily initiating from within the first 500 base pairs of genes.

The expression of intragenic transcripts when transcription elongation factors are disrupted suggests that cells have evolved to prevent spurious intragenic transcription.
However, some cases of intragenic transcription are consistently detected in wild-type cells, and some of these cases are known to be important for different biological functions.
Chapter \ref{chapter:stress} of this dissertation describes our efforts to better understand the functions of intragenic transcription in wild-type cells by studying uncharacterized instances of intragenic transcription.
To discover uncharacterized instances of intragenic transcription, we applied high resolution genomic assays of transcription initiation to wild-type \textit{Saccharomyces cerevisiae} under three stress conditions.
For the condition of oxidative stress, we show that intragenic transcripts are generally expressed at lower levels than genic transcripts, and that many intragenic transcripts are likely to be translated at some level.
By comparing intragenic transcription in three yeast species, we find that most examples of oxidative-stress regulated intragenic transcription identified in \textit{S. cerevisiae} are not conserved.
Finally, we show that the expression of an oxidative-stress-induced intragenic transcript at the gene \textit{DSK2} is needed for \textit{S. cerevisiae} to survive in conditions of oxidative stress.

\end{abstractpage}
\cleardoublepage

\tableofcontents
\cleardoublepage

% \listoftables
% \addcontentsline{toc}{chapter}{List of Tables}
% \cleardoublepage

\listoffigures
\addcontentsline{toc}{chapter}{List of Figures}
\cleardoublepage

