\chapter{Genomics of transcription elongation factor Spt6}

\section{Collaborators}

\begin{description}[align=right, labelwidth=5cm, noitemsep]
    \item [Steve Doris] optimized TSS-seq and ChIP-nexus protocols
    \item [] generated TSS-seq and ChIP-nexus libraries
    \item [Olga Viktorovskaya] generated MNase-seq libraries
    \item [Magdalena Murawska] generated NET-seq libraries
    \item [Dan Spatt] various experiments for publication
\end{description}

\section{Introduction to Spt6 and intragenic transcription}

% The work described in this chapter relates to understanding how a eukaryotic cell specifies which sites in its genome are permitted to become sites of transcription initiation.
% To get a rough idea of the specificity of transcription initiation, it is useful to start with a simple back-of-the-envelope calculation of the proportion of the human genome at which transcription initiation occurs.
% The human genome is approximately three billion base pairs in length, and each base pair can potentially be transcribed from each of its two strands.
% Each gene in the genome can have multiple transcription start sites (TSSs), which I assume to be five in number for the average gene.
% At last count, the human genome contains about twenty thousand protein-coding genes.
% To be conservative in our estimate with regards to specificity, we assume that all twenty thousand genes are expressed, which leads to the following proportion:
% \begin{align*}
%     \frac{\left(2 \times 10^4 \; \text{genes}\right) \left(5 \; \frac{\text{TSS}}{\text{gene}} \right)}
%          {\left(3 \times 10^9 \; \text{base pairs} \right) \left(2 \; \frac{\text{TSS}}{\text{base pair}} \right)}.
% \end{align*}
% However, this expression underestimates the extent of transcription initiation by only considering protein-coding genes, neglecting the many classes of noncoding genes present in the genome.
% If we assume that there are five noncoding genes for each coding gene, the updated expression,
% \begin{align*}
%     \frac{\left(1.2 \times 10^5 \; \text{genes}\right) \left(5 \; \frac{\text{TSS}}{\text{gene}} \right)}
%          {\left(3 \times 10^9 \; \text{base pairs} \right) \left(2 \; \frac{\text{TSS}}{\text{base pair}} \right)}
%     &= 0.0001,
% \end{align*}
% says that when presented with a thousand positions to choose from, RNA polymerase chooses just one to start transcribing from!

% Where transcription initiates is determined in large part by DNA sequence: the presence of certain sequence motifs increases the probability that RNA polymerase binds to DNA along with numerous co-factors required for initiation.
% However, DNA sequence alone does not entirely account for the specificity of transcription initiation.
% Genetic studies in yeast first showed that some transcription \textit{elongation} factors, including histone chaperones and histone modification enzymes, play a role in restricting where transcription is allowed to initiate \citep{kaplan2003, cheung2008, hennig2013}.
% In this project, we study the role of a particular transcription elongation factor called \textbf{Spt6} in this process.
% Many years of research on Spt6 is summarised as follows \citep{doris2018}:

% \begin{itemize}[nosep, topsep=.5em]
% \item Spt6 interacts directly with:
% 	\begin{itemize}[nosep]
% 	\item RNA polymerase II (RNAPII) \citep{close2011, diebold2011, liu2011, sdano2017, sun2010, yoh2007}
% 	\item histones \citep{bortvin1996, mccullough2015}
% 	\item the essential factor Spn1 (IWS1) \citep{diebold2010b, li2018, mcdonald2010}
% 	\end{itemize}
% \item Spt6 is believed to function primarily as an elongation factor based on:
% 	\begin{itemize}[nosep]
% 	\item association with elongating RNAPII \citep{andrulis2000, ivanovska2011, kaplan2000, mayer2010}
% 	\item ability to enhance elongation in vitro \citep{endoh2004} and in vivo \citep{ardehali2009}
% 	\end{itemize}
% \item Spt6 has been shown to regulate initiation in some cases \citep{adkins2006, ivanovska2011}
% \item Spt6 regulates:
% 	\begin{itemize}[nosep]
% 	\item chromatin structure \citep{bortvin1996, degennaro2013, ivanovska2011, jeronimo2015, kaplan2003, perales2013, vanbakel2013}
% 	\item histone modifications, including:
% 		\begin{itemize}[nosep]
% 		\item H3K36 methylation \citep{carrozza2005, chu2006, yoh2008, youdell2008}
% 		\item in some organisms, H3K4 and H3K27 methylation \citep{begum2012, chen2012, degennaro2013, wang2017, wang2013}
% 		\end{itemize}
% 	\end{itemize}
% \item Spt6 is likely a histone chaperone required to reassemble nucleosomes in the wake of transcription \citep{duina2011}.
% \end{itemize}

\begin{wrapfigure}[18]{r}{3in}
    \centering
    \includegraphics[width=3in]{figures/six/six_spt6_western.pdf}
    \caption[Western blot showing Spt6 protein levels in wild-type and \textit{spt6-1004} cells, at 30\textdegree C and after 80 minutes at 37 \textdegree C.]{Western blot showing Spt6 protein levels in wild-type and \textit{spt6-1004} cells, at 30\textdegree C and after 80 minutes at 37 \textdegree C. Immunoblotting was performed using $\alpha$-FLAG antibody to detect Spt6 and $\alpha$-Myc antibody to detect Dst1 from a spike-in strain. The quantification shown is the mean $\pm$ standard deviation of three blots.}
    \label{fig:six_spt6_western}

    \vspace{0.5em}

    \centering
    \includegraphics[width=3in]{figures/six/six_gene_diagram.pdf}
    \caption[Diagram of transcript classes.]{Diagram of transcript orientation with respect to coding DNA sequences, for the categories of transcripts referred to in this document.}
    \label{fig:six_gene_diagram}
\end{wrapfigure}

Studies in the yeasts \textit{Saccharomyces cerevisiae} and \textit{Schizosaccharomyces pombe} have previously examined the requirement for Spt6 in normal transcription \citep{cheung2008, degennaro2013, kaplan2003, pathak2018, uwimana2017, vanbakel2013}.
As Spt6 is essential for viability in \textit{S. cerevisiae}, many of these studies use the same temperature-sensitive \textit{spt6} mutant used in this project, \textbf{\textit{spt6-1004}}, which encodes an in-frame deletion of a helix-hairpin-helix domain within Spt6 \citep{kaplan2003}.
When \textit{spt6-1004} cells are shifted from 30\textdegree C to 37\textdegree C for 80 minutes, bulk Spt6 protein levels are depleted to about 20\% of wild-type levels (Figure \ref{fig:six_spt6_western}).
The most notable phenotype of the \textit{spt6-1004} mutant is the appearance of \textbf{intragenic transcripts}, transcripts which appear to arise from within protein-coding sequences, in both sense and antisense orientations relative to the coding gene (Figure \ref{fig:six_gene_diagram}) \citep{cheung2008, degennaro2013, kaplan2003, uwimana2017}.

\begin{SCfigure}[40][h]
    \centering
    \includegraphics[width=4.4in]{figures/six/six_aat_assay_comparison.pdf}
    \caption[RNA-seq, TSS-seq, and TFIIB ChIP-nexus signal at the \textit{AAT2} gene, in \textit{spt6-1004} after 80 minutes at 37\textdegree C.]{Sense strand RNA-seq signal, sense strand TSS-seq signal, and TFIIB ChIP-nexus protection at the \textit{AAT2} gene, in \textit{spt6-1004} after 80 minutes at 37\textdegree C.}
    \label{fig:six_aat_assay_comparison}
\end{SCfigure}

Previous genome-wide measurements of transcript levels in \textit{spt6-1004} relied on tiled microarrays \citep{cheung2008} and RNA sequencing \citep{uwimana2017}.
Studying intragenic transcription is difficult with these methods, since the signal for an intragenic transcript in the same orientation as the gene it overlaps is convoluted with the signal from the full-length `genic' transcript (Figure \ref{fig:six_aat_assay_comparison}) \citep{cheung2008, lickwar2009}. Therefore, these methods can only discover intragenic transcripts which are highly expressed relative to the corresponding genic transcript, and are likely to find many false positives.
Additionally, these methods are assays of steady-state RNA levels, which makes them unable to distinguish whether the intragenic transcripts observed in \textit{spt6-1004} result from: A) new intragenic transcription initiation in the mutant, B) reduced decay of intragenic transcripts which are rapidly degraded in wild-type, or C) processing of full-length protein-coding RNAs.

To address these challenges to studying intragenic transcription, we applied two genomic assays to \textit{spt6-1004}: transcription start-site sequencing (\textbf{TSS-seq}), and \textbf{ChIP-nexus of TFIIB}, a component of the RNA polymerase II pre-initiation complex (PIC).
TSS-seq sequences the 5$^\prime$ end of capped and polyadenylated RNAs \citep{arribere2013, malabat2015}, allowing separation of intragenic from genic RNA signals and identification of intragenic transcript starts with single-nucleotide resolution (Figure \ref{fig:six_aat_assay_comparison}).
ChIP-nexus is a high-resolution chromatin immunoprecipitation technique, in which the immunoprecipitated DNA is exonuclease digested up to the bases crosslinked with the protein of interest before sequencing \citep{he2015}.
When applied to the PIC component TFIIB, ChIP-nexus reports where transcription initiation is occurring, thus allowing us to determine if intragenic transcripts in \textit{spt6-1004} result from new transcription initiation.

\section{Data analysis pipelines for TSS-seq and ChIP-nexus}

% In order to use TSS-seq and ChIP-nexus to answer questions about Spt6 and intragenic transcription, I developed analysis pipelines for TSS-seq and ChIP-nexus data.
% The pipelines are written using the Python-based Snakemake workflow specification language \citep{koster2012}, and perform steps including read cleaning \citep{martin2011}, various quality controls \citep{andrews2012}, read alignment \citep{kim2013, langmead2012}, data normalization, coverage track generation \citep{quinlan2010}, peak calling \citep{zhang2008}, differential expression/binding analyses \citep{love2014}, data visualization with clustering, motif enrichment analyses \citep{bailey2015}, and gene ontology analyses \citep{young2010}.
% The Snakemake framework allows these data analyses to be reproducible and scalable from workstations up to computing clusters.
% Updated versions of these pipelines with more details on their capabilities are available at \href{https://github.com/winston-lab}{github.com/winston-lab}.
% In the following subsections I will describe the thought behind only a few of the more novel pipeline steps before moving on to results relating to Spt6 and intragenic transcription.

\subsection{TSS-seq peak calling}

% TSS-seq data from a single region of transcription initiation tends to occur as a cluster of signal distributed over a range of positions, rather than a single nucleotide (Figure \ref{fig:tss_coverage}) \citep{arribere2013, malabat2015}.
% It is reasonable to consider such a cluster of TSS-seq signal as a single entity, because the signals within the cluster are usually highly correlated to one another across different conditions.
% Therefore, to identify TSSs from TSS-seq data and quantify them for downstream analyses such as differential expression, it is necessary to annotate these groups of signal by using the data to perform peak-calling.

% In its current state, the TSS-seq pipeline calls peaks using 1-D \href{https://en.wikipedia.org/wiki/Watershed_(image_processing)}{watershed segmentation}, followed by filtering for reproducibility by the Irreproducible Discovery Rate (IDR) method \citep{li2011}.
% First, a smoothed version of the TSS-seq coverage is generated for each sample using a discretized Gaussian kernel.
% Next, an initial set of peaks is generated by: 1) assigning all nonzero signal in the original, unsmoothed coverage to the nearest local maximum of the smoothed coverage in the direction of positive derivative, and 2) taking the minimum and maximum genomic coordinates of the original coverage assigned to each local maximum as the peak boundaries.
% The peaks are then trimmed to the smallest genomic window that includes 95\% of the original coverage, and the probability of the peak being generated by noise is estimated by a Poisson model where $\lambda$, the expected coverage, is the maximum of the expected coverage over the chromosome and the expected coverage in a window upstream of the peak (as for the ChIP-seq peak caller MACS2 \citep{zhang2008}).
% The influence of local read density on $\lambda$ is intended to reduce false positive peaks within gene bodies, especially for highly expressed genes: Since there are more fragments of RNA present for highly expressed genes, more fragments within the gene body will make it into the final library, even if they are not true 5' ends.
% To generate the final set of peaks, the peaks are ranked by significance under the Poisson model, and filtered by IDR.
% In brief, IDR attempts to separate true peaks from experimental noise based on the intuition that, when peaks in each replicate are independently ranked by a metric such as significance, true peaks will have more similar ranks between replicates than peaks representing noise \citep{li2011}.

% The IDR algorithm currently only works for two replicates.
% Future improvements could include expanding the IDR implementation to handle more replicates and improve the accuracy of peak calling with more data.

\subsection{A note on ChIP-nexus peak calling}

% A number of tools have been created specifically for peak-calling using data from high-resolution ChIP techniques such as ChIP-nexus and ChIP-exo \citep{wang2014, hansen2016}.
% When applied to our TFIIB ChIP-nexus data, these tools tended to split what appeared to be a single TFIIB binding event into multiple peaks.
% This may be because TFIIB has been observed to crosslink to DNA at multiple points (Figure \ref{fig:tfiib_tata}) \citep{rhee2012}, which suggests that while these tools may work well for factors that bind symmetrically with a single crosslinking point on either side, there is still room for improvement when it comes to factors with more complex binding patterns.
% For the purposes of this project, the standard ChIP-seq peak caller MACS2 was used \citep{zhang2008}.

% ChIP-seq peaks lack strand information, as DNA binding factors usually do not bind DNA in a strand-specific manner.
% Because of this, we could not separate intragenic TFIIB peaks into peaks associated with sense or antisense transcription.
% The distinctive shape of the aggregate TFIIB ChIP-nexus signal (Figure \ref{fig:tfiib_tata}) suggests that information about the strand of transcription may be present in the ChIP-nexus binding profile.
% Future work could include learning the direction of transcription from labeled ChIP-nexus training data.

\section{TSS-seq and TFIIB ChIP-nexus results for \textit{spt6-1004}}

To study the relationship between Spt6 and transcription, TSS-seq and TFIIB ChIP-nexus libraries were prepared from wild-type and \textit{spt6-1004} cells, both shifted from 30\textdegree C to 37\textdegree C for 80 minutes.
In wild-type cells, TSS-seq and TFIIB ChIP-nexus recapitulate their expected distributions over the genome: Most TSS signal is restricted to annotated genic TSSs, while most TFIIB signal is localized just upstream of the TSS (Figures \ref{fig:six_tss_seq_heatmaps}, \ref{fig:six_tfiib_heatmap}).
In \textit{spt6-1004}, the signal for both assays infiltrates gene bodies, already suggesting that new transcription initiation does contribute to the intragenic transcription phenotype.
Notably, sense strand TSS-seq signal in \textit{spt6-1004} tends to occur towards the 3$^\prime$ end of genes, while antisense strand TSS-seq signal tends to occur towards the 5$^\prime$ end of genes.

\begin{figure}[H]
\centering
\includegraphics[width=6in]{figures/six/six_tss_seq_heatmaps.pdf}
\caption[Heatmaps of sense and antisense TSS-seq signal from wild-type and \textit{spt6-1004} cells, over non-overlapping coding genes.]{Heatmaps of sense and antisense TSS-seq signal from wild-type and \textit{spt6-1004} cells, over 3522 non-overlapping genes aligned by wild-type genic TSS and sorted by annotated transcript length. Data are shown for each gene up to 300 nucleotides 3$^\prime$ of the cleavage and polyadenylation site (CPS), indicated by the white dotted line. Values are the mean of spike-in normalized coverage in non-overlapping 20 nucleotide bins, averaged over two replicates. Values above the 92nd percentile are set to the 92nd percentile for visualization.}
\label{fig:six_tss_seq_heatmaps}
\end{figure}

\begin{wrapfigure}[19]{R}{3in}
    \centering
    \includegraphics[width=3in]{figures/six/six_tfiib_heatmap.pdf}
    \caption[Heatmaps of TFIIB ChIP-nexus protection from wild-type and \textit{spt6-1004} cells, over non-overlapping coding genes]{Heatmaps of TFIIB binding measured by ChIP-nexus, over the same regions shown in Figure \ref{fig:six_tss_seq_heatmaps}. Values are the mean of library-size normalized coverage in non-overlapping 20 bp bins, averaged over two replicates. Values above the 85th percentile are set to the 85th percentile for visualization.}
    \label{fig:six_tfiib_heatmap}
\end{wrapfigure}

The TSS-seq data were quantified by peak calling and differential expression analysis, and classified into genomic categories based on their position relative to coding genes.
As suggested by the heatmap visualization (Figure \ref{fig:six_tss_seq_heatmaps}), we detect significant induction of over 4000 intragenic and antisense TSSs in \textit{spt6-1004} (Figure \ref{fig:six_tss_diffexp_summary}).
Compared to previous studies identifying \textit{spt6-1004} intragenic transcription by tiled microarray and RNA-seq, we identify intragenic transcription at over 1000 additional genes (Figure \ref{fig:six_intragenic_genes_bvenn}) and have the exact start sites of all identified TSSs.
The TSS-seq data also revealed an unexpected downregulation of most genic TSSs: In this experiment, we detected a significant downregulation to levels below 67\% of wild-type levels at 75\% (3579/4792) of genic TSSs (Figure \ref{fig:six_tss_diffexp_summary}).
As a result of intragenic/antisense induction and genic repression, expression levels in \textit{spt6-1004} of all classes of transcripts become similar to one another (Figure \ref{fig:six_tss_expression_levels}).

\begin{figure}[H]
    \centering
    \begin{minipage}[t]{2.875in}
        \centering
        \includegraphics[width=2.875in]{figures/six/six_tss_diffexp_summary.pdf}
        \caption[Bar plot of the number of TSS-seq peaks of various genomic classes differentially expressed in \textit{spt6-1004} versus wild-type.]{Bar plots of the number of TSS-seq peaks differentially expressed in \textit{spt6-1004} after 80 minutes at 37\textdegree C versus wild-type after 80 minutes at 37\textdegree C. The height of each bar is proportional to the total number of peaks in the category, including those not found to be significantly differentially expressed.}
        \label{fig:six_tss_diffexp_summary}
    \end{minipage}\hfill
    \begin{minipage}[t]{2.875in}
        \centering
        \includegraphics[width=2.875in]{figures/six/six_intragenic_genes_bvenn.pdf}
        \caption[Set diagram of the number of genes with \textit{spt6-1004}-induced intragenic transcripts reported in \citet{cheung2008}, \citet{uwimana2017}, and our TSS-seq data.]{Set diagram of the number of genes reported to have \textit{spt6-1004}-induced intragenic transcripts using tiled arrays \citep{cheung2008}, RNA-seq \citep{uwimana2017}, and TSS-seq (this work).}
        \label{fig:six_intragenic_genes_bvenn}
    \end{minipage}
\end{figure}

\begin{wrapfigure}[10]{r}{3in}
    \centering
    \includegraphics[width=3in]{figures/six/six_tss_expression_levels.pdf}
    \caption[Violin plots of expression level distributions for genomic classes of TSS-seq peaks in wild-type and \textit{spt6-1004} cells.]{Violin plots of expression level distributions for genomic classes of TSS-seq peaks in wild-type and \textit{spt6-1004}, both after 80 minutes at 37\textdegree C. Normalized counts are the mean of spike-in size factor normalized counts from two replicates.}
    \label{fig:six_tss_expression_levels}
\end{wrapfigure}

The changes in transcript levels in \textit{spt6-1004} observed by TSS-seq correspond with substantial differences in the pattern of TFIIB binding on the genome.
In contrast to the discrete peaks in promoter regions seen in wild-type, TFIIB in \textit{spt6-1004} binds much more promiscuously, with many loci having TFIIB signal spread over broad regions of the genome (Figure \ref{fig:six_tfiib_spreading_ssa4}).
This difference in binding pattern makes peak calling ineffective for quantifying TFIIB signal in this case: ChIP-seq peak callers generally use different algorithms for calling `narrow' peaks (e.g. for sequence-specific transcription factors) and `broad' peaks (e.g. for histone modifications), meaning that a single algorithm is unable to call peaks that are meaningful for differential binding analyses between wild-type and \textit{spt6-1004}.
Therefore, to see if changes in transcript levels in \textit{spt6-1004} correspond to changes in transcription initiation, we compared the change in TSS-seq signal at TSS-seq peaks in \textit{spt6-1004} to the change in TFIIB ChIP-nexus signal in the window extending 200 bp upstream of the TSS-seq peak.
Changes in TSS-seq signal in \textit{spt6-1004} are associated with a change in TFIIB signal of the same sign at over 82\% of TSSs of any genomic class, indicating that the increase in intragenic transcript levels and decrease in genic transcript levels observed in \textit{spt6-1004} are in large part explained by changes in transcription initiation.

\begin{SCfigure}[50][h]
\centering
\includegraphics[width=3.5in]{figures/six/six_tfiib_spreading_ssa4.pdf}
\caption[TFIIB ChIP-nexus protection over the 20 kb flanking the gene \textit{SSA4}, in wild-type and \textit{spt6-1004} cells.]{
    \begin{description}[align=right, nosep, itemindent=0pt, leftmargin=4.2em, font=\normalfont]
        \item [top)] TFIIB ChIP-nexus protection in wild-type and \textit{spt6-1004}, over 20 kb of chromosome II flanking the \textit{SSA4} gene.
        \item [bottom)] Expanded view of TFIIB protection over the \textit{SSA4} gene.
    \end{description}
}
\label{fig:six_tfiib_spreading_ssa4}
\end{SCfigure}

\begin{figure}[h]
\centering
\includegraphics[width=6in]{figures/six/six_tss_v_tfiib.pdf}
\caption[Scatterplots of fold-change in \textit{spt6-1004} over wild-type, comparing TSS-seq and TFIIB ChIP-nexus.]{Scatterplots of fold-change in \textit{spt6-1004} over wild-type, comparing TSS-seq and TFIIB ChIP-nexus. Each dot represents a TSS-seq peak paired with the window extending 200 bp upstream of the TSS-seq peak summit for quantification of TFIIB ChIP-nexus signal. Fold-changes are regularized fold-change estimates from DESeq2, with size factors determined from the \textit{S. pombe} spike-in (TSS-seq), or \textit{S. cerevisiae} counts (ChIP-nexus).}
\end{figure}

\section{MNase-seq results from \textit{spt6-1004}}

Because a primary function of Spt6 is to act as histone chaperone that reassembles nucleosomes in the wake of transcription \citep{duina2011}, it is reasonable to expect that the transcriptional changes seen in \textit{spt6-1004} would be associated with changes in chromatin structure.
The requirement for Spt6 in maintaining normal chromatin structure has been demonstrated in previous studies \citep{bortvin1996, ivanovska2011, jeronimo2015, kaplan2003, perales2013, vanbakel2013}.
To re-examine this requirement in higher resolution, we assayed nucleosome protection genome-wide using micrococcal nuclease digestion of chromatin followed by sequencing (MNase-seq).

\begin{figure}[H]
    \centering
    \begin{minipage}[t]{2.875in}
        \centering
        \includegraphics[width=2.875in]{figures/six/six_mnase_metagene.pdf}
        \caption[Average MNase-seq dyad signal in wild-type and \textit{spt6-1004}, over non-overlapping genes aligned by wild-type +1 nucleosome dyad.]{Average MNase-seq dyad signal in wild-type and \textit{spt6-1004}, over 3522 non-overlapping genes aligned by wild-type +1 nucleosome dyad. Values are the mean of spike-in normalized coverage in non-overlapping 20 bp bins, averaged over two replicates (\textit{spt6-1004}) or one experiment (wild-type). The solid line and shading are the median and the inter-quartile range.}
        \label{fig:six_mnase_metagene}
    \end{minipage}\hfill
    \begin{minipage}[t]{2.875in}
        \centering
        \includegraphics[width=2.875in]{figures/six/six_global_nuc_occ_fuzz.pdf}
        \caption[Contour plot of nucleosome occupancy and fuzziness in wild-type and \textit{spt6-1004}.]{Contour plot of the global distributions of nucleosome occupancy and fuzziness in wild-type and \textit{spt6-1004}. Dashed lines indicate median values.}
        \label{fig:six_global_nuc_occ_fuzz}
    \end{minipage}
\end{figure}

In wild-type, the MNase-seq data recapitulate the expected signature over genes, with a nucleosome-depleted region upstream of a strongly positioned `+1' nucleosome, and a regularly phased array of nucleosomes over the gene body (Figure \ref{fig:six_mnase_metagene}).
In \textit{spt6-1004}, nucleosome signal is severely reduced at canonical nucleosome positions and spreads into inter-nucleosome regions.
Changes in aggregate nucleosome signal such as those observed in Figure \ref{fig:six_mnase_metagene} are the combination of changes to nucleosome occupancy (the number of reads assigned to a nucleosome), fuzziness (the standard deviation of read positions for a nucleosome), and position (the coordinate with the maximum reads for a nucleosome) \citep{chen2013}.
Using DANPOS2 \citep{chen2013}, we called nucleosome positions and quantified these metrics for wild-type and \textit{spt6-1004}.
Wild-type nucleosomes span a relatively wide range of occupancy and fuzziness space, with highly occupied nucleosomes tending to be less fuzzy (i.e., more well-positioned) (Figure \ref{fig:six_global_nuc_occ_fuzz}).
In \textit{spt6-1004}, the population of nucleosomes is much more homogeneous: nucleosome occupancy is decreased globally, and nucleosome fuzziness is restricted to the high end of the wild-type distribution.

Previous studies observed two trends: 1) In wild-type cells, nucleosome positioning is weaker over highly transcribed genes than over moderately transcribed genes \citep{shivaswamy2008}, and 2) In \textit{spt6-1004} cells, the decrease in nucleosome occupancy is greater for highly transcribed genes \citep{ivanovska2011}.
To re-examine these trends, we looked at the MNase-seq data in the context of NET-seq data, which reports the position of actively transcribing RNAPII and reflects a gene's level of transcription (Figure \ref{fig:six_mnase_heatmaps}) \citep{churchman2012}.
The data support the first trend: in wild-type, genes with the strongest NET-seq signal have decreased MNase-seq signal.
However, there is no obvious relationship between transcription level and the nucleosome changes observed in \textit{spt6-1004} (Figure \ref{fig:six_mnase_heatmaps}).
The apparent discrepancy might be explained by the improved resolution and breadth of MNase-seq versus the MNase and microarray of chromosome III used in the previous study \citep{ivanovska2011}.

\begin{figure}[H]
    \centering
    \includegraphics[width=6in]{figures/six/six_mnase_heatmaps.pdf}
    \caption[Heatmaps of sense NET-seq signal, MNase-seq dyad signal, nucleosome occupancy changes, and nucleosome fuzziness changes over non-overlapping coding genes, aligned by genic TSS and arranged by sense NET-seq signal.]{
        \begin{description}[align=right, nosep, itemindent=0pt, leftmargin=4.2em, font=\normalfont]
            \item [left)] Heatmap of sense strand NET-seq signal for 3522 non-overlapping genes, aligned by genic TSS and sorted by total sense strand NET-seq signal in the window extending 500 nt downstream from the genic TSS. Values are the mean of library-size normalized coverage in non-overlapping 20 nt bins, averaged over two replicates.
            \item [middle)] Heatmaps of MNase-seq dyad signal in wild-type and \textit{spt6-1004} for the same genes, aligned by wild-type +1 nucleosome dyad and arranged by sense NET-seq signal as in the leftmost panel. Values are the mean of spike-in normalized coverage in non-overlapping 20 bp bins, averaged over two replicates (\textit{spt6-1004}) or one experiment (wild-type).
            \item [right)] Heatmaps of fold-change in nucleosome occupancy and fuzziness for the same genes, aligned by wild-type +1 nucleosome dyad and arranged by sense NET-seq signal as in the leftmost panel.
        \end{description}
    }
    \label{fig:six_mnase_heatmaps}
\end{figure}

\subsection{Clustering of MNase-seq profiles at \textit{spt6-1004}-induced intragenic TSSs}

The aggregate MNase-seq dyad signal around all \textit{spt6-1004} intragenic TSSs is aperiodic (Figure \ref{fig:six_intragenic_mnase_metagenes}, top left panel), which occurs as a result of destructive interference from offset nucleosome phasing patterns.
To discover these phasing patterns, we used the wild-type and \textit{spt6-1004} MNase-seq data flanking intragenic TSSs to train a self-organizing map to assign TSSs with similar MNase-seq patterns to nearby nodes in a rectangular grid (Figure \ref{fig:six_mnase_som}).
This allowed us to see that, although there is considerable diversity in the nucleosome pattern surrounding intragenic TSSs, most intragenic TSSs occur in areas between the positions of nucleosome dyads.
By hierarchically clustering the nodes of the self-organizing map, we further grouped intragenic TSSs into three major clusters differing primarily by the phasing of the nucleosome array relative to the TSS, as shown in Figure \ref{fig:six_intragenic_mnase_metagenes}.

\begin{sidewaysfigure}
    \centering
    \includegraphics[width=8.25in]{figures/six/six_mnase_som.pdf}
    \caption[Average MNase-seq dyad signal around all \textit{spt6-1004}-induced intragenic TSSs, grouped by a self-organizing map of the MNase-seq signal.]{Average MNase-seq dyad signal around all \textit{spt6-1004}-induced intragenic TSSs, grouped by assignment to nodes of a 6x8 super-organizing map (SOM). The number of TSSs assigned to each node is shown in the upper right of each panel, and is shaded by the node's assignment to a cluster determined by agglomerative hierarchical clustering of the nodes. The solid line and shading are the median and inter-quartile range of the mean spike-in normalized coverage over two replicates (\textit{spt6-1004}) or one experiment (wild-type), in non-overlapping 5 bp bins.}
    \label{fig:six_mnase_som}
\end{sidewaysfigure}

\begin{figure}[H]
\centering
\includegraphics[width=6in]{figures/six/six_intragenic_mnase_metagenes.pdf}
\caption[Average wild-type and \textit{spt6-1004} MNase-seq dyad signal and GC content for three clusters of \textit{spt6-1004}-induced intragenic TSSs, as well as wild-type genic TSSs.]{
    \begin{description}[align=right, nosep, itemindent=0pt, leftmargin=6.2em, font=\normalfont]
        \item [left column)] Average MNase-seq dyad signal for \textit{spt6-1004} intragenic TSSs, both aggregated and grouped into three clusters by the wild-type and \textit{spt6-1004} MNase-seq dyad signal flanking the TSS, as well as all genic TSSs detected in wild-type. Values are the mean of spike-in normalized dyad coverage in non-overlapping 10 bp bins, averaged over two replicates (\textit{spt6-1004}) or one experiment (wild-type). The solid line and shading are the median and inter-quartile range.
        \item [right column)] Average GC content of the DNA sequence in a 21 bp window, as above.
    \end{description}
}
\label{fig:six_intragenic_mnase_metagenes}
\end{figure}

\clearpage

\section{Other features of \textit{spt6-1004} intragenic promoters}

The resolution with which we were able to identify intragenic TSSs allowed us to closely examine their sequence features and compare them to genic TSSs.

\subsection{Information content and sequence preference of intragenic TSSs}

\begin{wrapfigure}[13]{R}{3in}
\centering
\includegraphics[width=3in]{figures/six/six_tss_seqlogos.pdf}
\caption[Sequence logos of TSS-seq reads overlapping genic and intragenic TSS-seq peaks in \textit{spt6-1004}.]{Sequence logos depicting information content and sequence preference of TSS-seq reads overlapping genic and intragenic TSS-seq peaks in \textit{spt6-1004}.}
\label{fig:six_tss_seqlogos}
\end{wrapfigure}

To examine the DNA sequence preference of intragenic and genic TSSs in \textit{spt6-1004}, we aligned the sequences of all TSS-seq reads overlapping TSS-seq peaks of each class, and calculated the information content and sequence distribution for each class (Figure \ref{fig:six_tss_seqlogos}).
Intragenic TSSs have a sequence preference almost identical to previously observed sequence preference of genic TSSs \citep{malabat2015}, suggesting that RNA polymerase initiates transcription similarly at genic and intragenic TSSs, and that the lack of intragenic initiation in wild-type is due to inaccessibility of the initiation sequence, possibly due to histones.

\subsection{Sequence motifs enriched at intragenic TSSs}

To examine whether sequence-specific transcription factors contribute to the expression of intragenic transcripts in \textit{spt6-1004}, we looked for enrichment or depletion of the DNA sequence motifs associated with these factors upstream of intragenic TSSs.
Exact matches to the TATA element consensus sequence TATAWAWR are enriched upstream between 100 and 150 nt upstream of intragenic TSSs, in the same position but to a lesser degree than the TATA enrichment observed upstream of genic TSSs (Figure \ref{fig:six_intragenic_tata}).
This further supports the model that \textit{spt6-1004} intragenic promoters are sequences similar to canonical genic promoters, which become accessible for transcription initiation when the normal chromatin state is disturbed.

\begin{SCfigure}[50][h]
    \centering
    \includegraphics[width=3in]{figures/six/six_intragenic_tata.pdf}
    \caption[Kernel density estimate of matches to a consensus TATA-box motif upstream of genic and \textit{spt6-1004}-induced intragenic TSSs.]{Scaled density of occurrences of exact matches to the motif TATAWAWR upstream of TSSs. For each category, a Gaussian kernel density estimate of the positions of motif occurrences is scaled by the number of motif occurrences per region.}
    \label{fig:six_intragenic_tata}
\end{SCfigure}

\begin{figure}[h]
    \centering
    \includegraphics[width=6in]{figures/six/six_meme_motifs.pdf}
    \caption[Sequence logos of motifs discovered by MEME upstream of \textit{spt6-1004}-induced intragenic and antisense TSSs.]{Sequence logos of motifs discovered by MEME \citep{bailey2015} in the window -100 to +30 bp relative to \textit{spt6-1004} intragenic and antisense TSSs. The number of motif occurrences and the E-value, indicating the expected number of motif occurrences if the input sequences were scrambled, are shown for each motif.}
    \label{fig:six_meme_motifs}
\end{figure}

\section{Discussion}

\clearpage
\section{Methods}

\subsection{Yeast strain construction and grown conditions}

All yeast strains were constructed by standard yeast transformation or crosses.
The \textit{spt6-1004} and wild-type strains were grown as previously described \citep{cheung2008}: Cells were grown in YPD at 30\textdegree C to a concentration of approximately $1 \times 10^7$ cells/ml (OD$_{600} = 0.6$), at which point an equal volume of YPD medium pre-warmed to 44\textdegree C was added, and the cultures were shifted to 37\textdegree C for 80 minutes.

\subsection{Sequencing library preparation\\(TSS-seq, ChIP-nexus, MNase-seq, NET-seq)}

All library preparation methods are detailed in \citet{doris2018}.

\subsection{Genome builds}

The genome build used for \textit{S. cerevisiae} was R64-2-1.
The genome build used for \textit{S. pombe} was ASM294v2.

\subsection{TSS-seq data analysis}

An up-to-date version of the Snakemake \citep{koster2012} workflow used to process TSS-seq libraries is maintained at \href{https://github.com/winston-lab/tss-seq}{github.com/winston-lab/tss-seq}.
At the time of writing, removal of adapter sequences and random hexamer sequences from the 3$^\prime$ end of the read and 3$^\prime$ quality trimming were performed using cutadapt \citep{martin2011}.
The random hexamer molecular barcode on the 5$^\prime$ end of the read was then removed and processed using a custom Python script (adapted from \citet{mayer2015}).
Reads were aligned to the combined \textit{S. cerevisiae} and \textit{S. pombe} reference genomes using Tophat2 \citep{kim2013} without a reference transcriptome, and uniquely mapping reads were selected using SAMtools \citep{li2009}.
Reads mapping to the same location as another read with the same molecular barcode were identified as PCR duplicates and removed using a custom Python script (adapted from \citet{mayer2015}).
Coverage of the 5$^\prime$-most base, corresponding to the TSS, was extracted using bedtools genomecov \citep{quinlan2010} and normalized to the total number of uniquely mapping, non-duplicate \textit{S. pombe} reads.
Quality statistics of raw, cleaned, non-aligning, and uniquely aligning non-duplicate reads were assessed using FastQC \citep{andrews2014}.

The pipeline additionally performs TSS-seq peak calling, differential expression, classification of peaks into genome categories, gene ontology analysis, motif enrichment analysis, \textit{de novo} motif discovery, sequence logo visualization, and data visualization with the option to separate data into clusters of similar signal.

\subsubsection{Reannotation of \textit{S. cerevisiae} TSSs using TSS-seq data}

TSS-seq coverage from two replicates of a wild-type \textit{S. cerevisiae} strain grown at 30\textdegree C in YPD was averaged and used to adjust the 5$^\prime$ ends of an annotation of major transcript isoforms based on TIF-seq data \citep{pelechano2013}.
The 5$^\prime$ end of the original annotation was changed to the position of maximum TSS-seq signal in a window $\pm$ 250 nt of the original 5$^\prime$ end if the maximum TSS-seq signal was greater than the 95th percentile of all non-zero TSS-seq signal.

\subsubsection{TSS-seq peak calling}
\label{subsubsec:tss_peak_calling}

TSS-seq data representing transcription from a single promoter tends to occur as a cluster of signal distributed over a range of positions, rather than a single nucleotide \citep{arribere2013, malabat2015}.
It is reasonable to consider such a cluster of TSS-seq signal as a single entity, because the signals within the cluster are usually highly correlated to one another across different conditions.
Therefore, to identify TSSs from TSS-seq data and quantify them for downstream analyses such as differential expression, it is necessary to annotate these groups of TSS-seq signal by using the data to perform peak-calling.

At the time of writing, TSS-seq peak calling for a given experimental group was performed by 1-D watershed segmentation of the data for each sample in the group, followed by filtering for reproducibility within the group by the Irreproducible Discovery Rate (IDR) method \citep{li2011}.
First, a smoothed version of the TSS-seq coverage is generated for each sample using an adaptive two-stage kernel density estimation with a discretized Gaussian kernel \citep{silverman1986}.
For a given nucleotide, the adaptive kernel bandwidth, $\sigma_{\text{adaptive}}$, is given by
\begin{align*}
    \sigma_{\text{adaptive}} &= \sigma_\text{pilot} \left( \frac{\rho_{\text{pilot}}}{g} \right)^{-\alpha},
\end{align*}
where $\sigma_\text{pilot}$ is the standard, fixed bandwidth of a Gaussian kernel used to calculate the pilot signal density $\rho_\text{pilot}$ at that nucleotide, $g$ is the geometric mean of $\rho_\text{pilot}$ over the whole genome, and $\alpha$ is a parameter in $[0,1]$ that determines the degree to which the pilot density $\rho_\text{pilot}$ affects $\sigma_\text{adaptive}$.
The adaptive kernel adjusts the kernel bandwidth to be smaller in regions of high signal density and larger in regions of lower signal density, allowing the smoother to better accommodate both `sharp' TSSs where the signal is distributed over a relatively small window, as well as `broad' TSSs where the signal is more dispersed.
For all analyses in this document, adaptive smoothing was performed with $\sigma_\text{pilot} = 10$ and $\alpha = 0.2$.

Following smoothing, an initial set of peaks is formed by assigning all nonzero signal in the original, unsmoothed coverage to the nearest local maximum of the smoothed coverage, and taking the minimum and maximum genomic coordinates of the original coverage as the peak boundaries for each local maximum of the smoothed coverage.
Peaks are then trimmed to the smallest genomic interval that includes 95\% of the original coverage, and the probability of the peak begin generated by noise is estimated by a Poisson model where $\lambda$, the expected coverage, is the maximum of the expected coverage over the chromosome and the expected coverage in the 2 kb window upstream of the peak (\`a la the ChIP-seq peak caller MACS2 \citep{zhang2008}).
Finally, peaks are ranked by their significance under the Poisson model, and a final list of peaks for the group is generated using the IDR method ($\text{IDR}=0.1$) \citep{li2011}.
In brief, IDR compares ranked lists of regions in order to set a cutoff, beyond which the regions are no longer consistent between replicates.

The python script used for 1-D watershed segmentation of TSS-seq data is \href{https://github.com/winston-lab/tss-seq/blob/master/scripts/tss_peakcalling.py}{available as part of the TSS-seq pipeline}, and the IDR implementation used in the pipeline is also \href{https://github.com/nboley/idr}{available on GitHub}.

\subsubsection{TSS differential expression analysis}

For TSS-seq differential expression analysis, TSS-seq peak-calling was performed \hyperref[subsubsec:tss_peak_calling]{as described above} for both \textit{S. cerevisiae} and the \textit{S. pombe} spike-in.
The read counts for each peak in each condition were used as the input to differential expression analysis by DESeq2 \citep{love2014}, with the alternative hypothesis $\allowbreak \left\lvert\log_2 \left(\text{fold-change}\right) \right\rvert > 1.5$ and a false discovery rate of 0.1.
To normalize by spike-in, the size factors of the \textit{S. pombe} spike-in counts were used as the size factors for \textit{S. cerevisiae}, although we note that due to the median of ratios normalization used in DESeq2, the major TSS-seq results of this work are still observed when \textit{S. cerevisiae} size factors are used.

\subsubsection{Classification of TSS-seq peaks into genomic categories}
\label{subsubsec:tss_peak_classification}

TSS-seq peaks were assigned to genomic categories based on their position relative to the transcript annotation \hyperref[subsubsec:tss_reannotation]{described above} and an annotation of all verified open reading frames (ORFs) and blocked reading frames in \textit{S. cerevisiae} \citep{crooks2004, engel2014}.
First, `genic' regions were defined as follows: If a gene was present in both the transcript and ORF annotations, the genic region was defined as the interval (annotated TSS-30 nt, start codon).
If gene was present in the transcript annotation but not the ORF annotation, the genic region was defined as the interval (annotated TSS - 30 nt, annotated TSS + 30 nt).
If a gene was present only in the ORF annotation, the genic region was defined as the interval (start codon - 30 nt, start codon).
For the purposes of peak classification, regions were considered overlapping if they had at least one base of overlap.
TSS-seq peaks were classified as genic if they overlapped a genic region on the same strand.
Peaks were classified as intragenic if they were not classified as as genic peak, and their summit position overlapped an open or closed reading frame on the same strand.
Peaks were classified as antisense if their summit position overlapped a transcript on the opposite strand.
Finally, peaks were classified as intergenic if they did not overlap a transcript, reading frame, or genic region on either strand.

\subsubsection{TSS information content and sequence composition}

TSS-seq alignments were pooled for all replicates in a condition, and the DNA sequence flanking the position of every read overlapping TSS-seq peaks of a particular genomic category was extracted using SAMtools \citep{li2009} and bedtools \citep{quinlan2010}.
The information content and sequence composition was quantified using WebLogo \citep{crooks2004}, with the zeroth-order Markov model of the \textit{S. cerevisiae} genomic sequence as the background composition.
Sequence logos were plotting using helper functions from ggseqlogo \citep{wagih2017}.

\subsection{ChIP-nexus data analysis}
An up-to-date version of the Snakemake \citep{koster2012} workflow used to process ChIP-nexus libraries is maintained at \href{https://github.com/winston-lab/chip-nexus}{github.com/winston-lab/chip-nexus}.
At the time of writing, filtering for reads containing the constant region of the adapter on the 5$^\prime$ end of the read, 3$^\prime$ adapter removal, and 3$^\prime$ quality trimming were performed using cutadapt \citep{martin2011}.
The random pentamer molecular barcode on the 5$^\prime$ end of the read was then removed and processed using a custom Python script modified from \citet{mayer2015}.
Reads were aligned to the combined \textit{S. cerevisiae} and \textit{S. pombe} genomes using Bowtie2 \citep{langmead2012}, and uniquely mapping reads were selected using SAMtools \citep{li2009}.
Reads mapping to the same location as another read with the same molecular barcode were identified as PCR duplicates and removed using a custom Python script modified from \citet{mayer2015}.
Coverage of the 5$^\prime$-most base, corresponding to the point of crosslinking, was extracted using bedtools genomecov \citep{quinlan2010}.
The median fragment size estimated by MACS2 \citep{zhang2008} over all samples was used to generate coverage of factor protection and fragment midpoints, by extending reads to the fragment size, or by shifting reads by half the fragment size, respectively.
Coverage was normalized to the total number of reads uniquely mapping to \textit{S. cerevisiae}.
Quality statistics of raw, cleaned, non-aligning, and uniquely aligning non-duplicate reads were assessed using FastQC \cite{andrews2014}.

\subsubsection{A note on ChIP-nexus peak calling}
\label{subsubsec:tfiib_peak_calling}

\subsubsection{TFIIB ChIP-nexus differential binding analysis}
For TFIIB ChIP-nexus differential binding analysis, TFIIB peaks were called by MACS2 and IDR filtering \hyperref[subsubsec:tfiib_peak_calling]{as described above}.
A non-redundant list of peaks called in the condition and control groups being compared was generated using bedtools multiinter \citep{quinlan2010}, and the counts of fragment midpoints for each peak in each sample were used as the input to differential binding analysis by DESeq2 \citep{love2014}, with the alternative hypothesis $\allowbreak \left\lvert\log_2 \left(\text{fold-change}\right) \right\rvert > 1.5$ and a false discovery rate of 0.1.
For estimation of change in TFIIB binding upstream of TSS-seq peaks, TFIIB fragment midpoint counts in the window extending 200 bp upstream of the TSS-seq peak summit were used as the input to DESeq2.
\textit{S. cerevisiae} counts were used for size factor calculation.

\subsubsection{Classification of TFIIB ChIP-nexus peaks into genomic categories}

As for TSS-seq peaks, TFIIB ChIP-nexus peaks were assigned to genomic categories based on their position relative to the transcript annotation \hyperref[subsubsec:tss_reannotation]{described above}, an annotation of all verified open reading frames (ORFs) and blocked reading frames \citep{crooks2004, engel2014}, and \hyperref[subsubsec:tss_peak_classification]{an annotation of `genic' regions derived from the transcript and ORF annotations}.
TFIIB ChIP-nexus peaks were classified as genic if they overlapped a genic region.
Peaks were classified as intragenic if they were not classified as a genic peak, and the entire peak overlapped an open or closed reading frame.
Finally, peaks were classified as intergenic if they did not overlap a transcript, reading frame, or genic region.

\subsection{MNase-seq data analysis}

\subsubsection{Nucleosome quantification}

\subsubsection{Clustering of MNase-seq signal at \textit{spt6-1004} intragenic TSSs}

\subsection{Motif enrichment}

\subsection{\textit{De novo} motif discover motif discovery}

\clearpage
\bibliographystyle{apalike}
\begingroup
    \singlespacing
    \bibliography{references/spt6}
\endgroup
